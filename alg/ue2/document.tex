\documentclass[]{article}

\usepackage{amsfonts} 
\usepackage{amsmath}
\usepackage{amssymb}
\usepackage[margin=2.5cm]{geometry}

%opening
\title{Algebra Übungsblatt 2}
\author{Ida Hönigmann}

\begin{document}

\maketitle

\begin{abstract}

\end{abstract}

\newpage
\section{68}

\texttt{*}, \texttt{+} ... 2-stellig
\texttt{cos}, \texttt{sin} ... 1-stellig
\texttt{i} ... 0-stellig
\texttt{a}, \texttt{b}, \texttt{c} ... Variablen 

\begin{verbatim}
	
(i)       (ii)       (iii)         (iv)
    *         sin          +             +
   / \         |          / \           / \
 sin  c        +        sin  b         *  cos
  |           / \        |            / \  |
  +          a   b       a           sin i a
 / \                                  |
a   b                                 b
\end{verbatim}


\begin{tabular}{r|ccc}
		  & Präfix               & Infix                    & Postfix              \\\hline
	(i)   & \texttt{*sin+abc}    & \texttt{sin(a+b)*c}      & \texttt{ab+sinc*}    \\
	(ii)  & \texttt{sin+ab}      & \texttt{sin(a+b)}        & \texttt{ab+sin}      \\
	(iii) & \texttt{+sinab}      & \texttt{sin(a)+b}        & \texttt{asinb+}      \\
	(iv)  & \texttt{+*sinbicosa} & \texttt{sin(b)*i+cos(a)} & \texttt{acosibsin*+} \\
\end{tabular}


\newpage

\section{72}
\textbf{Definition}

\begin{align*}
	C \text{ heißt Klon auf } A :\iff & (i) \forall n \in \mathbb{N}\setminus\{0\} \forall i \in \{1, ..., n\}: \pi_i^{(n)} \in C \\
	& (ii) f_1,f_2,...,f_k:A^n\rightarrow A, g:A^k \rightarrow A \in C \\
	& \implies g\circ_{n,k}(f_1,...,f_k) = g(f_1(a_1,...,a_n),...,f_k(a_1,...,a_n)) \in C
\end{align*}

\noindent
\textbf{gesucht:} 3 Klone $C$ auf $A:=\{1, ..., k\}$ mit $A^A\subseteq C$, wobei $k\geq 3$.

\begin{itemize}
	\item 
	
\begin{align*}
	C_a := \bigcup_{n\in\mathbb{N}\setminus\{0\}} \{f:A^n\rightarrow A \vert \exists i \in \{1, ..., n\}\exists \tilde{f}\in A^A : f(x_1, ..., x_n) = \tilde{f}(x_i)\}
\end{align*}

\begin{itemize}
	\item Sei $\pi_i^{(n)}$ eine beliebige Projektion. $\pi_i^{(n)} \in C_a$, da für $\tilde{f} = id$ gilt $f(x_1, ..., x_n) = x_i$.
	
	\item Sei $f_1,...,f_k,g \in C_c$ (mit Stelligkeiten wie oben beschrieben) beliebig. Nennen wir $h:=g\circ_{n,k}(f_1,...,f_k)$.
	
	\begin{align*}
		g(x_1, ..., x_k) &= \tilde{g}(x_i)\\
		f_1(x_1, ..., x_n) &= \tilde{f}_1(x_j)\\
		&\vdots\\
		f_k(x_1, ..., x_n) &= \tilde{f}_k(x_l)\\
		\implies h(x_1, ..., x_n) &= \tilde{g}(\tilde{f}_i(x_l)) \in C_a
	\end{align*}

	\item Für $n=1$ ist $f:A\rightarrow A$ mit $f=\tilde{f}$ beliebig in $C_a$.
\end{itemize}
	
	\item 
	
\begin{align*}
	C_b := \bigcup_{n\in\mathbb{N}\setminus\{0\}} \{f: A^n \rightarrow A\}
\end{align*}

\begin{itemize}
	\item Alle Projektionen $\pi_i^{(n)}$ liegen in der Menge aller Funktionen von $A^n$ nach $A$.
	
	\item Alle beliebigen Verknüpfungen von Funktionen liegen in der Menge alller Funktionen von $A^n$ nach $A\text{ Widerspruch!}$.
	
	\item Alle $A^A$ liegen in der Menge aller Funktionen von $A^1$ nach $A$.
\end{itemize}
	
	\item
	
\begin{align*}
	C_c := C_a \cup \bigcup_{n\in\mathbb{N}\setminus\{0\}} \{f: A^n \rightarrow A \vert f \text{ ist nicht surjektiv}\}
\end{align*}

\begin{itemize}
	\item Wir haben schon gezeigt, dass alle Projektionen $\pi_i^{(n)} \in C_a$. Gemeinsam mit $C_a \subseteq C_c$ ergibt das $\pi_i^{(n)} \in C_c$.
	
	\item Sei $f_1,...,f_k,g \in C_c$ (mit Stelligkeiten wie oben beschrieben) beliebig. Nennen wir $h:=g\circ_{n,k}(f_1,...,f_k)$.
	
	Falls $g$ eine nicht surjektive Funktion ist, ist klarerweise auch $h$ nicht surjektiv und somit $h \in C_c$.
	
	Sonst gilt $g \in C_a$ und somit $\exists \tilde{g}:A\rightarrow A \exists i\in \{1, ..., k\} \vert g(x_1,...,x_k) = \tilde{g}(x_i)$. $\implies h=\tilde{g}(f_i(x_1, ..., x_n))$. Falls auch $f_i \in C_a$ so haben wir bereits gezeigt, dass $h \in C_a$. Anderenfalls ist $h$ nicht surjektiv, da $f_i$ nicht surjektiv ist.
	
	\item Wir haben schon gezeigt, dass $A^A \subseteq C_a \subseteq C_c$.
\end{itemize}
	
\end{itemize}

Nun müssen wir zeigen, dass es sich um drei unterschiedliche Klone handelt.

\begin{align*}
	f:A^2 &\rightarrow A\\
	(a, b) &\mapsto ((a+b)mod |A|) + 1
\end{align*}

$f$ liegt (klarerweise) in $C_b$. Angenommen $f$ liegt in $C_a \implies f(a,b)=\tilde{f}(a) \lor f(a,b)=\tilde{f}(b)$. o.B.d.A $f(a,b)=\tilde{f}(a)$.

\begin{align*}
	f(1, 1) = 3 &\implies \tilde{f}(1) = 3\\
	f(1, 2) = 1 \text{ falls } |A| = 3 \text{ und } 4 \text{ falls } |A| > 3 &\implies \tilde{f}(1) \neq 3 \text{ Widerspruch!}
\end{align*}

Also $C_a \neq C_b$.

$f$ ist surjektiv, da man mit $\{(1, l): l \in \{1, ..., k\}\}$ alle Elemente in $A$ erreichen kann. Also $f \notin C_c$ und somit $C_b \neq C_c$.


\begin{align*}
	g:A^2 &\rightarrow A\\
	(a, b) &\mapsto 1 \text{ falls } a=b \text{ und } 2 \text{ sonst}
\end{align*}

Die Funktion $g$ ist offensichtlich nicht surjektiv ($3 \notin g(A)$) also $g \in C_c$.

Angenommen $g$ liegt in $C_a \implies g(a,b)=\tilde{g}(a) \lor g(a,b)=\tilde{g}(b)$. o.B.d.A $g(a,b)=\tilde{g}(a)$.

\begin{align*}
	g(1, 1) = 1 &\implies \tilde{g}(1) = 1\\
	g(1, 2) = 2 &\implies \tilde{g}(1) = 2 \text{ Widerspruch!}
\end{align*}

Also $C_a \neq C_b$.
\newpage

\section{77}
$A$ ... Menge, $\mathcal{O}_A$ ... Menge aller Klone auf $A$

\noindent
\textbf{zu zeigen: } $(\mathcal{O}_A, \subseteq)$ bildet einen vollständigen Verband

Sei $P \subseteq \mathcal{O}_A$ beliebig.

Für $P = \emptyset$ ist

\begin{align*}
	inf(P) &= \bigcup_{n\in\mathbb{N}\setminus\{0\}}\{f:A^n\rightarrow A\} \\
	sup(P) &= \bigcup_{n\in\mathbb{N}\setminus\{0\}}\{\pi_i^{(n)}: i\in \{1, ..., n\}\}.
\end{align*}
 
Was auch mit den unteren Definitionen übereinstimmt, wenn man die Vereinigung und den Schnitt über die leere Menge entsprechend definiert.
 
Wir wollen zeigen $\exists C \in \mathcal{O}_A: C = inf(P)$ also

\begin{align*}
	\forall D \in P: C \subseteq D & \text{ und}\\
	\forall \tilde{C} \in \mathcal{O}_A:(\forall D \in P: \tilde{C} \subseteq D) \implies \tilde{C} \subseteq C.
\end{align*}

\begin{align*}
	C:= \bigcap_{D\in P}D
\end{align*}

\begin{itemize}
	\item \textbf{zz: } $C$ ist ein Klon auf $A$
	
	Sei $\pi_i^{(n)}$ eine beliebige Projektion auf $A$. $\forall D \in P: \pi_i^{(n)} \in D$, da $D \in \mathcal{O}_A$. Das bedeutet aber, dass $\pi_i^{(n)} \in \bigcap_{D\in P}D = C$.
	
	Sei $f_1, ..., f_k : A^n \rightarrow A, g:A^k\rightarrow A$ aus $C$ beliebig. $\implies \forall D \in P: f_1, ..., f_k, g \in D$ und daher auch $\forall D \in P: h := g\circ_{n,k}(f_1, ..., f_k) \in D$. Das bedeutet aber, $h \in C$.
	
	Also ist $C \in \mathcal{O}_A$.
	
	\item \textbf{zz: } $\forall D \in P: C \subseteq D$
	
	gilt nach Definition von $C$.
	
	\item \textbf{zz: } $\forall \tilde{C} \in \mathcal{O}_A:(\forall D \in P: \tilde{C} \subseteq D) \implies \tilde{C} \subseteq C$
	
	Sei $\tilde{C} \in \mathcal{O}_A$ mit $\forall D \in P: \tilde{C} \subseteq D$ beliebig. Angenommen $C \subsetneq \tilde{C}$. Das bedeutet $\exists f \in \tilde{C}\setminus C$.
	
	Da $f \notin C$ gilt $\exists D \in P: f \notin D$ und somit $\neq (\tilde{C} \subseteq D)$ was ein Widerspruch ist. Also muss gelten $\tilde{C} \subseteq C$.
\end{itemize}

Insgesamt ist also $C = inf(P)$.

Da $(\mathcal{O}_A, \subseteq)$ eine Halbordnung ist und jede Teilmenge ein Infimum besitzt gilt nach Aufgabe 52 von letzer Woche, dass auch jede Teilmenge ein Supremum besitzt.

Alternativer Beweis des Supremums:

Zu zeigen: $\exists C \in \mathcal{O}_A: C = sup(P)$ also

\begin{align*}
	\forall D \in P: D \subseteq C & \text{ und}\\
	\forall \tilde{C} \in \mathcal{O}_A:(\forall D \in P: D \subseteq \tilde{C}) \implies C \subseteq \tilde{C}.
\end{align*}

\begin{align*}
	C:= \left[\bigcup_{D\in P}D\right] & \text{ wobei } [M] \text{ den Abschluss unter allen } \circ_{n,k} \text{ bezeichnet}
\end{align*}

\begin{itemize}
	\item \textbf{zz: } $C$ ist ein Klon auf $A$
	
	Sei $\pi_i^{(n)}$ eine beliebige Projektion auf $A$. $\forall D \in P: \pi_i^{(n)} \in D$, da $D \in \mathcal{O}_A$. Das bedeutet aber, dass $\pi_i^{(n)} \in \bigcup_{D\in P}D = C$.
	
	Die Abgeschlossenheit bezüglich aller $\circ_{n,k}$ gilt nach Definition.
	
	Also ist $C \in \mathcal{O}_A$.
	
	\item \textbf{zz: } $\forall D \in P: D \subseteq C$
	
	gilt nach Definition von $C$.
	
	\item \textbf{zz: } $\forall \tilde{C} \in \mathcal{O}_A:(\forall D \in P: D \subseteq \tilde{C}) \implies C \subseteq \tilde{C}$
	
	Sei $\tilde{C} \in \mathcal{O}_A$ mit $\forall D \in P: D \subseteq \tilde{C}$ beliebig. Angenommen $\tilde{C} \subsetneq C$. Das bedeutet $\exists f \in C \setminus \tilde{C}$.
	
	Da $f \in C$ gilt entweder $\exists D \in P: f \in D$ was aber ein Widerspruch zu $D \subseteq \tilde{C}$ ist, da $f \notin \tilde{C}$ oder
	$f$ entsteht durch $\circ_{n,k}$ auf $\bigcup_{D\in P}D$. Da $\forall D \in P: D \subseteq \tilde{C} \implies \bigcup_{D\in P}D \subseteq \tilde{C}$ kann $\tilde{C}$ kein Klon sein, da $f \notin \tilde{C}$ und somit $\tilde{C}$ nicht unter allen $\circ_{n,k}$ abgeschlossen ist.
\end{itemize}

Insgesamt also $C = sup(P)$.


\newpage

\section{102}
\begin{enumerate}
	\item $E := \{\{0\}, \{0, 1\}, \{0, 2\}, \{0, 3\}, \{0, 2, 3\}, \{0, 1, 2, 3\}\}$
	
	\textbf{gesucht: } Algebra $\mathcal{A}$ auf $A:=\{0, 1, 2, 3\}$, sodass $\forall M \in E: M$ ist Unteralgebra von $A$
	
	\begin{align*}
		\mathcal{A} = (A, w_1, w_2, w_3) \text{ (Typ (0,2,2))}
	\end{align*}
	\begin{align*}
			w_1:A^0 &\rightarrow A && w_2:A^3 &\rightarrow A && w_3:A^3 &\rightarrow A\\
			()&\mapsto 0 && (a,b,c) &\mapsto
			\begin{cases}
				2 & \text{, falls } \{a,b,c\} = \{0,1,3\}\\
				a & \text{, sonst}
			\end{cases} &&
			(a,b,c) &\mapsto
			\begin{cases}
				3 & \text{, falls } \{a,b,c\} = \{0,1,2\}\\
				a & \text{, sonst}
			\end{cases}
	\end{align*}

	\begin{align*}
		\mathcal{P}(A) = \{&&\emptyset, &&w_1! &&\{0\}, && &&\{1\}, &&w_1! &&\{2\}, &&w_1! && \\
		                   &&\{3\}, &&w_1! &&\{0, 1\}, && &&\{0, 2\}, && &&\{0, 3\}, && && \\
		                   &&\{1, 2\}, &&w_1! &&\{1, 3\}, &&w_1! &&\{2, 3\}, &&w_1! &&\{0, 1, 2\}, &&w_3! && \\
		                   &&\{0, 1, 3\}, &&w_2! &&\{0, 2, 3\}, && &&\{1, 2, 3\}, &&w_1! &&\{0, 1, 2, 3\} && &&\}
	\end{align*}
	Wobei $w_i!$ bedeutet, dass die Menge nicht unter $w_i$ abgeschlossen ist.

	\item Ist 1. für beliebiges $E \subseteq \mathcal{P}(A)$ lösbar?
	
	Nein, z.B. für $E = \emptyset$ gibt es keine Algebra $\mathcal{A}=(A,(w_i)_{i\in I})$ ohne Unteralgebren, da $A$ immer eine Unteralgebra von sich selbst ist!

	\item \textbf{gesucht: } Kriterium / Algorithmus um entscheiden zu können ob für gegebenes, endliches $A$ und $E \subseteq \mathcal{P}(A)$ eine Algebra $\mathcal{A} = (A,w_1,w_2,...)$ existiert mit $Sub(\mathcal{A}) = E$.

\end{enumerate}

	\begin{alignat*}{1}
		& \texttt{Algorithmus (A ... Menge, E ... gewünschte Unteralgebren)}\\
		(1) \hspace{0.5cm} &\texttt{for (} P \in \mathcal{P}(A) \setminus E \texttt{) \{}\\
		(2) \hspace{0.5cm} &\hspace{1cm} C := \bigcap_{B \in E, P\subseteq B}B\texttt{;}\\
		(3) \hspace{0.5cm} &\hspace{1cm} \texttt{if (}C\setminus P = \emptyset \texttt{) \{}\\
		(4) \hspace{0.5cm} &\hspace{2cm} \texttt{raise Error("nicht lösbar");}\\
		(5) \hspace{0.5cm} &\hspace{1cm} \texttt{\} else \{}\\
		(6) \hspace{0.5cm} &\hspace{2cm} w_P : A^{|P|} \rightarrow A, (x_1, ..., x_{|P|})\mapsto \begin{cases}
			y \in C \setminus P & \texttt{, falls } \{x_1, ..., x_{|P|}\} = P\\
			x_1 & \texttt{, sonst}
		\end{cases} \\
		(7) \hspace{0.5cm} &\hspace{2cm} \texttt{füge } w_P \texttt{ zur Algebra hinzu;}\\
		(8) \hspace{0.5cm} &\hspace{1cm} \texttt{\}}\\
		(9) \hspace{0.5cm} &\texttt{\}}\\
	\end{alignat*}

	Jede Operation $w$ erreicht (für unsere Zwecke), dass wenn $x_1, ..., x_n$ in einer Unteralgbra $U$ liegen, dass dann auch $y = w(x_1, ..., x_n) \in U$ sein muss.
	
	Damit ein $P \in \mathcal{P}(A)\setminus E$ nicht in $Sub(\mathcal{A})$ liegt muss also ein $w_P$ garantieren, dass wenn $P \subseteq U \implies \exists y \notin P: y \in U$. Natürlich muss das $y$ so gewählt werden, dass $\forall B \in E: P\subset B \implies y \in B$, da sonst $B \notin Sub(\mathcal{A})$. $\implies y \in \left(\bigcap_{B \in E, P\subseteq B}B\right)$.
	
	Falls aber $\left(\bigcap_{B \in E, P\subseteq B}B\right) = \emptyset$ kann nach der Überlegung von oben keine Lösung existieren. Sonst garantiert die Operation wie in (6) beschrieben, dass $P \notin Sub(\mathcal{A})$ für alle $P \notin E$ (wegen der Schleife in (1)).
	
	Für $Q \in E$ hat keine der Funktionen $(w_P)_{P\in \mathcal{P}(A)\setminus E}$ einen Effekt, da immer entweder gilt
	
	\begin{itemize}
		\item $P \subseteq Q$ und somit
		
		\begin{align*}
			\forall q_1, ..., q_{|P|} \in Q: w_P(q_1, ..., q_{|P|}) = \begin{cases}
				y & \text{, falls } \{q_1, ..., q_{|P|}\} = P\\
				q_1 & \text{, sonst}
			\end{cases}
		\end{align*}
	
		Da $Q \in E$ und $P \subseteq Q$ gilt nach Konstruktion, dass $y \in Q$ und $q_1 \in Q$ sowieso.
		
		\item $P\setminus Q \neq \emptyset$ und somit $\exists p \in P\setminus Q$ damit kann nicht der Fall $\{q1, ..., q_{|P|}\} = P$ eintreten. Also gilt
		
		\begin{align*}
			\forall q_1, ..., q_{|P|} \in Q: w_P(q_1, ..., q_{|P|}) = q_1 \in Q.
		\end{align*}
	\end{itemize}

	In beiden Fällen gilt also, dass $Q$ bezüglich allen $(w_P)_{P\in \mathcal{P}(A)\setminus E}$ abgeschlossen ist.

\end{document}
