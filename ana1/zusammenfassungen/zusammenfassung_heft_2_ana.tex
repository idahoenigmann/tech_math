\documentclass[twocolumn]{article}
\usepackage[utf8]{inputenc}
\usepackage[german]{babel}

\usepackage{amsthm}
\usepackage{amsmath}
\usepackage{amsfonts}
\usepackage{mathtools}
\usepackage{centernot}

\newtheorem{theorem}{Satz}[section]
\newtheorem{corollary}{Corollary}[theorem]
\newtheorem{lemma}[theorem]{Lemma}
\newtheorem{definition}{Definition}[section]
\newtheorem*{remark}{Bemerkung}
\newtheorem*{schreibweise}{Schreibweise}

%opening
\title{Zusammenfassung Heft 2 ANA}
\author{Ida Hönigmann}

\newcommand*{\logeq}{\Leftrightarrow}

\begin{document}
	
\maketitle

\section{Rationale Zahlen}
\begin{definition}
	Ein angeordneter Körper heißt vollständig angeordnet, falls $\emptyset \neq M \subseteq K$ und $M$ nach oben beschränkt ($\implies M$ hat Supremum).
\end{definition}

\begin{lemma}
	vollständig angeordnet $\implies$ archimedisch angeordnet
\end{lemma}

\begin{theorem}
	$K$ und $L$... vollständig angeordnete Körper $\implies \exists ! \phi:L\rightarrow K$ mit $+$ und $*$ verträglich, bijektiv und mit $\leq$ verträglich.
\end{theorem}
	
\begin{schreibweise}
	$\mathbb{R}=$ vollständig angeordneter Körper
\end{schreibweise}

\begin{theorem}
	$x\in \mathbb{R}$, $x\geq 0$, $n \in \mathbb{N}$
	
	Dann $\exists ! y \in \mathbb{R}, y \geq 0 : y^{n}=x$
\end{theorem}
	
\begin{definition}
	$x \in \mathbb{R}$, $x \geq 0$
	
	Sei $\sqrt[n]{x}$ die eindeutige Zahl $y \geq 0$ mit $y^{n}=x$.
\end{definition}
	
\begin{remark}
	$y \mapsto y^{n}$ ist bijektiv
\end{remark}
	
\begin{lemma}
	\begin{itemize}
		\item $\sqrt[q]{\frac{1}{x}}=\frac{1}{\sqrt[q]{x}}$
		\item $\sqrt[q]{xz}=\sqrt[q]{x}*\sqrt[q]{z}$
		\item $(\sqrt[q]{x})^{p}=\sqrt[q]{x^{p}}$
	\end{itemize}
\end{lemma}

\begin{definition}
	$x \in \mathbb{R}$, $x>0$, $r \in \mathbb{Q}$, $r=\frac{p}{q}$ mit $p \in \mathbb{Z}$ und $q \in \mathbb{N}$
	
	$x^{r}=(\sqrt[q]{x})^{p}$
\end{definition}
	
\begin{lemma}
	\begin{itemize}
		\item $x^{r+s}=x^{r}*x^{s}$
		\item $(x^{r})^{s}=x^{r*s}$
		\item $x^{-1}=\frac{1}{x^{r}}$
	\end{itemize}
\end{lemma}	

\begin{remark}
	$\mathbb{R} \supset \mathbb{Q}$, da $\sqrt{2} \notin \mathbb{Q}$, $\sqrt{2} \in \mathbb{R}$
\end{remark}

\begin{definition}
	$\mathbb{R}\setminus\mathbb{Q}$ heißen irrationale Zahlen, dafür schreibt man auch $\mathbb{I}$.
\end{definition}

\begin{lemma}
	$M$, $N \neq \emptyset$... Mengen $f:M\times N \rightarrow \mathbb{R}$
	
	Falls $f(M\times N)$ nach oben beschränkt, so gilt
	
	$sup(\{f(m,n) : (m,n) \in M \times N\})=sup(\{s_{q}:q \in N\})=sup(\{sup(\{f(m,n):m \in M\}) : n \in N\})$
	
	Falls $\forall n \in N \{f(m,n):m \in M\}$ nach oben beschränkt und falls $\{sup(\{f(m,n):m \in M\}) : n \in N\}$ nach oben beschränkt ist, dann ist $f(M\times N)$ nach oben beschränkt.
\end{lemma}

\begin{remark}
	$A,B \subset \mathbb{R} \implies sup(A+B)=sup(A)+sup(B)$
\end{remark}

\section{Komplexe Zahlen}
\begin{definition}
	$\mathbb{C}\coloneqq \mathbb{R}\times\mathbb{R}$
\end{definition}

\begin{schreibweise}
	$(a,b) \in \mathbb{C}$ werden in der Form $a+ib$ geschrieben. $a$ nennt man den Realteil, $b$ nennt man den Imaginärteil der Zahl.
\end{schreibweise}

\begin{definition}
	$a+ib,c+id \in \mathbb{C}$
	
	$(a+ib)+(c+id)=(a+c)+i(b+d)$
	
	$(a+ib)*(c+id)=(ac-bd)+i(ad+bc)$
\end{definition}

\begin{schreibweise}
	$a+ib \in \mathbb{C}$ mit $b=0$ heißt rein reell. $a\coloneqq (a+ib)$
	
	$a+ib \in \mathbb{C}$ mit $a=0$ heißt rein imaginär. $ib\coloneqq (a+ib)$
	
	$z=(a+ib)$ dann ist $Rez=a$ und $Imz=b$
\end{schreibweise}

\begin{remark}
	$i$ ist Lösung von $x^{2}+1=0$.
\end{remark}

\begin{theorem}
	$(\mathbb{C},+,*)$ ist Körper. Dabei ist für $a+ib \in \mathbb{C}$:
	
	\begin{itemize}
		\item $0+i0$ ist additiv neutrales Element
		\item $-a+i(-b)$ ist additiv inverses Element
		\item $1+i0$ ist multiplikativ neutrales Element
		\item $a+ib\neq 0 \implies \frac{a}{a^{2}+b^{2}}+i\frac{(-b)}{a^{2}+b^{2}}$ ist multiplikativ inverses Element
	\end{itemize}
\end{theorem}

\begin{remark}
	$\mathbb{C}$ ist kein angeordneter Körper.
\end{remark}

\begin{remark}
	$\mathbb{C}$ ist Vektorraum über $\mathbb{C}$ mit Dimension $1$.
	
	$\mathbb{C}$ ist Vektorraum über $\mathbb{R}$ mit Dimension $2$.
\end{remark}

\begin{definition}
	$z\coloneqq (a+ib) \in \mathbb{C}$
	
	$\bar{z}\coloneqq a+i(-b)$ heißt konjugiert komplexe Zahl.
	
	$|z|=\sqrt{a^{2}+b^{2}}$ mit $|z|=0 \logeq (a+ib)=(0+i0)$
\end{definition}

\begin{remark}
	"$\mathbb{R}\subseteq\mathbb{C}$"
\end{remark}

\begin{lemma}
	\begin{itemize}
		\item $|Rez|\leq |z|$
		\item $|Imz|\leq |z|$
		\item $|z*w|=|z|*|w|$
		\item $|z+w|\leq |z|+|w|$
		\item $||z|-|w||\leq |z-w|$
		\item $|\bar{z}|=|z|$
		\item $z*\bar{z}=|z|^{2}$
		\item $z\neq 0 \implies \frac{1}{z}=\frac{\bar{z}}{|z|^{2}}=\frac{a}{|z|^{2}}+i\frac{-b}{|z|^{2}}$
	\end{itemize}
\end{lemma}

\section{Grenzwerte}

\begin{definition}
	$M=\emptyset$, $d:M\times M \rightarrow \mathbb{R}$
	
	$(M,d)$ heißt metrischer Raum, falls
	
	\begin{itemize}
		\item (M1) $d(x,y)\geq 0$ und $d(x,y)=0\logeq x=y$
		\item (M2) $d(x,y)=d(y,x)$
		\item (M3) $x,y,z \in M \implies d(x,z) \leq d(x,y) + d(y,z)$
	\end{itemize}
\end{definition}

\begin{lemma}
	$p \in \mathbb{N}$, $a_{1},...,a_{p},b_{1},...,b_{p} \in \mathbb{R}$
	
	$(\sum_{j=1}^{n}a_{j}*b_{j})^{2}\leq(\sum_{j=1}^{n}a_{j}^{2})*(\sum_{j=1}^{n}b_{j}^{2})$
	
	$(\sum_{j=1}^{n}(a_{j}+b_{j})^{2})^{\frac{1}{2}} \leq (\sum_{j=1}^{n}a_{j}^{2})^{\frac{1}{2}}+(\sum_{j=1}^{n}b_{j}^{2})^{\frac{1}{2}}$
\end{lemma}

\begin{definition}
	$X$... Menge mit $X \neq \emptyset$
	
	Eine Folge $x$ ist eine Funktion $x:\mathbb{N}\rightarrow X$.
	
	Meist schreibt man $(x_{n})_{n \in \mathbb{N}}$ mit $x_{n}=x(n)$.
\end{definition}

\begin{definition}
	$(M,d)$... metrischer Raum, $(x_{n})_{n \in \mathbb{N}}$ ... Folge aus $M$, $x \in M$
	
	$(x_{n})_{n \in \mathbb{N}}$ heißt konvergent gegen $x$ falls
	
	$\forall \epsilon \in \mathbb{R}, \epsilon > 0 \exists N \in \mathbb{N} \forall n \geq N : d(x_{n},x)< \epsilon$
	
	In diesem Fall schreibt man auch
	
	$\lim\limits_{n\rightarrow \infty}x_{n}=x$ oder "bei $n \rightarrow \infty$ geht $x_{n}\rightarrow x$'' oder $(x_{n})_{n \in \mathbb{N}}\rightarrow x$.
\end{definition}

\begin{remark}
	$\lim\limits_{n\rightarrow\infty}x_{n}=x \logeq \lim\limits_{n\rightarrow\infty}d(x_{n},x)=0$
\end{remark}

\begin{remark}
	$k \in \mathbb{Z}$, $\mathbb{Z}_{\geq k}\coloneqq\{p \in \mathbb{Z}: p \geq k\}$
	
	$(x_{n})_{n \in \mathbb{Z}_{\geq K}}$... Folge
	
	konvergiert $\logeq \forall \epsilon > 0 \exists N \in \mathbb{Z}_{\geq K}:d(x_{n},x)<\epsilon\forall n \geq N$
	
\end{remark}

\begin{remark}
	Folgende Aussagen sind äquivalent:
	
	\begin{itemize}
		\item $\lim\limits_{n\rightarrow\infty}x_{n}=x$
		\item $\forall\epsilon >0\exists N \forall n \geq N : d(x_{n},x)\leq \epsilon$
		\item Für ein gewisses $K>0$ gilt: $\forall \epsilon > 0 \exists N \forall n \geq N : d(x_{n},x)< K*\epsilon$
		\item Für ein gewisses $K>0$ gilt: $\forall \epsilon > 0 \exists N \forall n \geq N: d(x_{n},x)\leq K*\epsilon$
	\end{itemize}
\end{remark}

\begin{definition}
	Sei $(x_{n})_{n \in \mathbb{N}}$ eine Folge in $M$. Ist $n:\mathbb{B}\rightarrow\mathbb{N}$ streng monoton wachsend, so heißt $(x_{n(j)})_{j\in \mathbb{N}}$ eine Teilfolge von $(x_{n})_{n \in \mathbb{N}}$.
\end{definition}

\begin{theorem}
	$(M,d)$... metrischer Raum, $(x_{n})_{n \in \mathbb{N}}$... Folge in $M$, $x \in M$
	
	\begin{itemize}
		\item $(x_{n})_{n \in \mathbb{N}}$ hat höchstens einen Grenzwert
		\item $k \in \mathbb{N}$ beliebig. $(x_{n})_{n \in \mathbb{N}}\rightarrow x \logeq (x_{n})_{n \in \mathbb{Z}_{\geq k}}\rightarrow x$
		\item $k \in \mathbb{N}$ beliebig. $(x_{n})_{n \in \mathbb{N}}\rightarrow x \logeq (x_{n}+k)_{n \in \mathbb{N}}\rightarrow x \logeq (x_{n}-k)_{n \in \mathbb{Z}_{\geq k+1}}\rightarrow x$
		\item $(x_{n})_{n \in \mathbb{N}}\rightarrow x \land (x_{n(j)})_{j \in \mathbb{N}}$ ist Teilfolge von $(x_{n})_{n \in \mathbb{N}} \implies \lim\limits_{j\rightarrow \infty}(x_{n(j)})_{j \in \mathbb{N}}=x$
	\end{itemize}
\end{theorem}

\begin{remark}
	Teilfolge konvergiert $\centernot\implies$ Folge konvertiert
\end{remark}

\begin{lemma}
	$(x_{n})_{n \in \mathbb{N}}\rightarrow x$, $(y_{n})_{n \in \mathbb{N}}\rightarrow y$ ... Folgen in $(M,d)$
	
	$\implies \lim\limits_{n\rightarrow \infty}d(x_{n},y_{n})=d(x,y)$
\end{lemma}

\begin{lemma}
	$(M,d)$... metrischer Raum, $a_{1},a_{2},b_{1},b_{2} \in M$
	
	$|d(a_{1},b_{1})-d(a_{2},b_{2})|\leq d(a_{1},a_{2})+d(b_{1},b_{2})$
\end{lemma}

\begin{definition}
	$(M,d)$... metrischer Raum
	
	\begin{itemize}
		\item $Y \subseteq M$ heißt beschränkt, falls $Y \neq \emptyset$ und $\exists z \in M \exists c > 0 \forall y \in Y : d(z,y)\leq c$
		\item $f:E\rightarrow M$ heißt beschränkt wenn $f(E)\subseteq M$ beschränkt ist.
		\item $(x_n)_{n \in \mathbb{N}}$... Folge aus $M$ heißt beschränkt, falls $x:\mathbb{N}\rightarrow M$ beschränkt ist.
	\end{itemize}
\end{definition}

\begin{remark}
	$\emptyset \neq Y \subseteq M$ ist beschränkt $\logeq \forall x \in M : \exists C_x \geq 0 \forall y \in Y: d(y,x)\leq C_x$
\end{remark}

\begin{lemma}
	$(x_n)_{n \in \mathbb{N}}... Folge in (M,d)$
	
	$\lim\limits_{n\rightarrow\infty}x_n =x \implies (x_n)_{n \in \mathbb{N}}$ ist beschränkt.
\end{lemma}

\begin{lemma}
	$(x_n)_{n \in \mathbb{N}}$, $(y_n)_{n \in \mathbb{N}}$ ... Folgen in $(\mathbb{R},d_2)$ mit $x_n\rightarrow x$ und $y_n\rightarrow y$
	
	Dann gelten folgende Aussagen:
	\begin{itemize}
		\item $\exists c \in \mathbb{R} : x < c \implies \exists N \in \mathbb{N}; x_n < c \forall n \geq N$.
		\item $x<y\implies \exists N \forall n \geq N : x_n < y_n$
		\item $\exists N \in \mathbb{N} \forall n \geq N: x_n \leq y_n \implies x \leq y$
	\end{itemize}
\end{lemma}

\begin{theorem}
	$(x_n)_{n \in \mathbb{N}}$, $(y_n)_{n \in \mathbb{N}}$, $(a_n)_{n \in \mathbb{N}}$... Folgen in $(\mathbb{R},d_2)$
	
	$x_n\rightarrow a \land y_n \rightarrow a \land x_n \leq a_n \leq y_n$ für alle $n \in \mathbb{N}$ bis auf endlich viele $\implies a_n \rightarrow a$
\end{theorem}

\begin{theorem}
	Sei $(z_n)_{n \in \mathbb{N}}$, $(w_n)_{n \in \mathbb{N}}$ Folgen aus $\mathbb{C}$ mit $z_n \rightarrow z$ und $w_n \rightarrow w$, dann gelten folgende Aussagen:
	
	\begin{itemize}
		\item $\lim\limits_{n\rightarrow \infty}|z_n|=|z|$ und $\lim\limits_{n\rightarrow \infty}\bar{z_n}=\bar{z}$
		
		\item $\lim z_n + w_n = z+w$ und $\lim (-z_n)=-z$
		
		\item $z = 0$ und $(u_n)_{n \in \mathbb{N}}$... beschränkte Folge $\implies z_n * u_n \rightarrow 0$
		
		\item $\lim z_n * w_n = z*w$
		
		\item $\lim z_n^k = z^k$, falls $k \in \mathbb{N}$ fest
		
		\item $z \neq 0 \implies \lim \frac{1}{z_n}=\frac{1}{z}$
		
		\item $z_n \in \mathbb{R}, z_n \geq 0, z \in \mathbb{R}$, $z \geq 0$, $k \in \mathbb{N}$ fest $\implies \sqrt[k]{z_n}\rightarrow \sqrt[k]{z}$
	\end{itemize}
\end{theorem}

\begin{definition}
	Eine Folge $(a_n)_{n \in \mathbb{N}}$ in $\mathbb{R}$ heißt
	
	\begin{itemize}
		\item monoton wachsend, falls $\forall n \in \mathbb{N} : a_n \leq a_{n+1}$
		\item monoton fallend, falls $\forall n \in \mathbb{N} : a_n \geq a_{n+1}$
		\item streng monoton wachsend, falls $\forall n \in \mathbb{N} : a_n < a_{n+1}$
		\item streng monoton fallend, falls $\forall n \in \mathbb{N} : a_n > a_{n+1}$
	\end{itemize}
\end{definition}

\begin{theorem}
	Falls $(x_n)_{n \in \mathbb{N}}$ eine monoton wachsende Folge in $\mathbb{R}$ ist nach oben beschränkt (d.h. $\exists c > 0 \forall n \in \mathbb{N} : x_n \leq c$), dann ist $(x_n)_{n \in \mathbb{N}}$ konvergent.
	
	$\lim\limits_{n \rightarrow \infty}x_n=sup(\{x_n : n \in \mathbb{N}\})$
	
	Falls $(x_n)_{n \in \mathbb{N}}$ eine monoton fallende Folge in $\mathbb{R}$ ist nach unten beschränkt (d.h. $\exists c > 0 \forall n \in \mathbb{N} : x_n \geq c$), dann ist $(x_n)_{n \in \mathbb{N}}$ konvergent.
	
	$\lim\limits_{n \rightarrow \infty}x_n=inf(\{x_n : n \in \mathbb{N}\})$
\end{theorem}

\begin{definition}
	$(x_n)_{n \in \mathbb{N}}$ ... beschränkte Folge in $\mathbb{R}$, $N \in \mathbb{N}$
	
	$y_N \coloneqq inf(\{x_n : n \geq N\})$
	
	$z_N \coloneqq inf(\{x_n : n \geq N\})$
	
	$\liminf\limits_{n \rightarrow \infty}x_n\coloneqq \lim\limits_{N \rightarrow \infty}y_N = sup\{y_N : N \in \mathbb{N}\}$
	
	$\limsup\limits_{n \rightarrow \infty}x_n\coloneqq \lim\limits_{N \rightarrow \infty}z_N = inf\{z_N : N \in \mathbb{N}\}$
\end{definition}

\begin{lemma}
	$(x_n)_{n \in \mathbb{N}}$... beschränkte Folge in $\mathbb{R}$
	
	$\implies \exists$ Teilfolge $(x_{n(j)})_{j \in \mathbb{N}}$ mit $\lim\limits_{j \rightarrow \infty}x_{n(j)}=\liminf\limits_{n \rightarrow \infty}x_n$
	
	$\implies \exists$ Teilfolge $(x_{n(j)})_{j \in \mathbb{N}}$ mit $\lim\limits_{j \rightarrow \infty}x_{n(j)}=\limsup\limits_{n \rightarrow \infty}x_n$
\end{lemma}

\begin{lemma}
	$(x_n)_{n \in \mathbb{N}}$, $(y_n)_{n \in \mathbb{N}}$... beschränkte Folge in $\mathbb{R}$
	
	\begin{itemize}
		\item $\liminf\limits_{n \rightarrow \infty}x_n \leq \limsup\limits_{n \rightarrow \infty}x_n$
		\item $\forall n : x_n \leq y_n \implies \liminf a_n \leq \liminf b_n \land \limsup a_n \leq \limsup b_n$
		\item $\limsup\limits_{n \rightarrow \infty}(-x_n)=-\liminf\limits_{n \rightarrow \infty}x_n$ und $\liminf\limits_{n \rightarrow \infty}(-x_n)=-\limsup\limits_{n \rightarrow \infty}x_n$
		\item $\lim\limits_{n \rightarrow \infty}x_n=x \logeq \liminf\limits_{n \rightarrow \infty}x_n = x \land \limsup\limits_{n \rightarrow \infty}x_n = x$
	\end{itemize}
\end{lemma}

\subsection{Cauchy-Folgen}

\begin{definition}
	$(M,d)$... metrischer Raum
	
	$(x_n)_{n \in \mathbb{N}}$ in $M$ heißt Cauchy-Folge, falls $\forall \epsilon > 0 \exists N \in \mathbb{N} \forall n, m \geq N : d(x_m,x_n)< \epsilon$
\end{definition}

\begin{remark}
	Ob $\epsilon$ aus $\mathbb{R}$ oder $\mathbb{Q}$ stammt ist egal.
	
	$(x_n)\rightarrow x \logeq \forall \epsilon > 0 \in \mathbb{Q} \forall n \geq N : d(x_n,x) < \epsilon$
\end{remark}

\begin{lemma}
	$(x_n)_{n \in \mathbb{N}}$ ist Cauchy-Folge $\implies (x_n)$ ist beschränkt.
\end{lemma}

\begin{lemma}
	$(M,d)$... metrischer Raum, $(x_n)_{n \in \mathbb{N}}$... Folge in $M$
	
	Falls $(x_n)$ konvergent ist, dann ist $(x_n)$ auch eine Cauchy-Folge.
\end{lemma}

\begin{remark}
	Cauchy-Folge impliziert nicht, das die Folge konvergiert.
\end{remark}

\begin{definition}
	Ein metrischer Raum heißt vollständig, falls jede Cauchy-Folge auch gegen einen Punkt aus diesem metrischen Raum konvergiert.
\end{definition}

\begin{lemma}
	$(M,d)$... metrischer Raum
	
	Falls $(x_n)_{n \in \mathbb{N}}$ eine Cauchy-Folge ist und falls $\exists$ Teilfolge $(x_{n(j)})_{j \in \mathbb{N}}$ mit $\lim\limits_{j \rightarrow \infty}x_{n(j)}=x \in M$
	
	$\implies \lim\limits_{n \rightarrow \infty}x_n = x$
\end{lemma}

\begin{theorem}
	$(\mathbb{R},d_2)$ ist ein vollständig metrischer Raum.
\end{theorem}

\begin{lemma}
	$(x_n)_{n \in \mathbb{N}}$ ... Folge aus $\mathbb{R}^p$, $x_n=(\xi_{n,1},...,\xi_{n,p})$
	
	$x=(\xi_1,...,\xi_p) \in \mathbb{R}^p$
	
	Falls $x_n \rightarrow x$ bzgl. $d_1,d_2,d_{\infty}$, dann gilt $x_n \rightarrow x$ bzgl. aller dieser Metriken. Weiters gilt $x_n \rightarrow x$ bzgl. $d_1,d_2,d_{\infty} \logeq \forall j \in \{1,...,p\} : \xi_{n,j} \rightarrow \xi_j in (\mathbb{R}, d_2)$.
\end{lemma}

\begin{lemma}
	$(\mathbb{R}^p,d_{1, 2 oder \infty})$ ist vollständig metrischer Raum.
	
	$(\mathbb{C}^p,d_{1, 2 oder \infty})$ ist vollständig metrischer Raum.
\end{lemma}

\begin{lemma}
	Eine komplexe Folge konvergiert, wenn ihr Realteil und Imaginärteil konvergieren.
\end{lemma}

\begin{definition}
	$(x_n)_{n \in \mathbb{N}}$ ... Folge in $\mathbb{R}$
	
	$\lim\limits_{n \rightarrow \infty}x_n = +\infty \logeq \forall M > 0 \exists N \in \mathbb{N} \forall n \geq N : x_n > M$
	
	$\lim\limits_{n \rightarrow \infty}x_n = -\infty \logeq \forall M < 0 \exists N \in \mathbb{N} \forall n \geq N : x_n < M$
\end{definition}

\begin{remark}
	Eine Folge kann nicht gleichzeitig gegen $x \in \mathbb{R}$ und gegen $+\infty$ oder $-\infty$ konvergieren.
\end{remark}

\begin{theorem}
	$(x_n), (y_n)$ ... Folgen in $\mathbb{R}$, wobei $\lim\limits_{n \rightarrow \infty}x_n = +\infty$
	
	Dann gelten folgende Aussagen:
	
	\begin{itemize}
		\item $\lim\limits_{n \rightarrow \infty}-x_n = -\infty$
		\item $\{y_n:n\in \mathbb{N}\}$ ... nach unten beschränkt $\implies \lim\limits_{n \rightarrow \infty}(x_n+y_n)=+\infty$
		\item $\exists C >0 \forall n \in \mathbb{N} : y_n \geq C$, dann $\lim\limits_{n\rightarrow\infty}(x_n*y_n)=+\infty$
		\item Falls $\forall n \in \mathbb{N} : x_n \leq y_n \implies \lim\limits_{n \rightarrow \infty}+\infty$
		\item $\forall n \in \mathbb{N} : y_n > 0 \implies \lim\limits_{n \rightarrow \infty}y_n = +\infty \logeq \lim\limits_{n \rightarrow \infty} \frac{1}{y_n}=0$
		\item Wenn $y_n$ monoton wachsend ist, dann gilt 
		\begin{itemize}
			\item $\lim\limits_{n\rightarrow \infty}=sup(y_n)$, falls $\{y_n : n \in \mathbb{N}\}$ nach oben beschränkt.
			\item $\lim\limits_{n\rightarrow\infty}+\infty$, falls $\{y_n : n \in \mathbb{N}\}$ nicht nach oben beschränkt ist.
		\end{itemize}
	\end{itemize}
\end{theorem}

\begin{remark}
	Gleicher Satz gilt in gleicher Form auch für $-\infty$.
\end{remark}

\subsection{Reihen}

\begin{definition}
	$a_k$ ... Folge aus $\mathbb{R}$ oder aus $\mathbb{C}$, $n\in\mathbb{N}$
	
	$S_n=\sum_{j=1}^{n}a_j$ heißt die j-te Partialsumme
	
	$(S_n)_{n \in \mathbb{N}}$ heißt Reihe mit Summanden $a_n$.
	
	Falls $\lim\limits_{n\rightarrow\infty}S_n$ existiert, so heißt die Reihe konvergent.
\end{definition}

\begin{schreibweise}
	Wenn eine Reihe konvergent ist schreiben wir für $\lim\limits_{n\rightarrow\infty}S_n=x$ auch $\sum_{k=1}^{\infty}a_k=x$.
	
	Für $\lim\limits_{n\rightarrow\infty}S-n=+\infty$ schreiben wir auch $\sum_{k=1}^{\infty}=+\infty$. Gleiches für $-\infty$.
	
	Für $(S_n)$ schreibt man auch $\sum_{k=1}^{\infty}a_k$.
\end{schreibweise}

\begin{lemma}
	$\sum_{k=1}^{\infty}a_k$, $\sum_{k=1}^{\infty}b_k$ ... konvergente Reihen in $\mathbb{R}$ oder $\mathbb{C}$.
	
	Dann gilt:
	\begin{itemize}
		\item $\sum_{k=1}^{\infty}(a_k+b_k)$... konvergiert gegen $(\sum_{k=1}^{\infty}a_k)+(\sum_{k=1}^{\infty}b_k)$
		\item $\sum_{k=1}^{\infty}(\lambda*a_k)=\lambda*\sum_{k=1}^{\infty}a_k$
		\item $\sum_{k=1}^{\infty}\overline{a_k}=\overline{\sum_{k=1}^{\infty}a_k}$
	\end{itemize}
\end{lemma}

\begin{lemma}
	\begin{itemize}
		\item $\sum_{k=1}^{\infty}a_k$, $\sum_{k=1}^{\infty}a'_k$... Reihen, wobei $\exists l \in \mathbb{N} \forall k \geq l : a_k = a'_k$
		
		$\implies \sum_{k=1}^{\infty}a_k$ konvergent $\logeq \sum_{k=1}^{\infty}a'_k$ konvergiert
	
		\item $(k(j))_{j \in \mathbb{N}}$... streng monoton wachsende Folge in $\mathbb{N}$
		
		$\sum_{n=1}^{\infty}a_n$ konvergent $\implies \sum_{n=1}^{\infty}A_n$ konvergiert mit $A_1 = a_1+...+a_{k(1)}$, ... $A_k=a_{k(n-1)+1}+...+a_{k(n)}$
		
		\item $\sum_{k=1}^{\infty}a_k$, $\sum_{k=1}^{\infty}b_k$ ... konvergieren mit $a_k,b_k \in \mathbb{R}$
		
		\begin{itemize}
			\item $\forall k \in \mathbb{N} : a_k \leq b_k \implies \sum_{k=1}^{\infty}a_k \leq \sum_{k=1}^{\infty}b_k$
			\item Falls zusätzlich gilt $\exists l \in \mathbb{N} : a_l < b_l$, dann ist $\sum_{k=1}^{\infty}a_k < \sum_{k=1}^{\infty}b_k$
		\end{itemize}
	\end{itemize}
	
\end{lemma}

\begin{remark}
	Eine komplexe Reihe ist konvergent, falls die Reihe der Realteile und die Reihe der Imaginärteile konvergiert.
\end{remark}

\begin{lemma}
	$\sum_{k=1}^{\infty}a_k$ ... konvergent $\implies \lim\limits_{n \rightarrow \infty} (a_k)_{k \in \mathbb{N}} = 0$
\end{lemma}

\begin{lemma}
	\begin{itemize}
		\item $\sum_{k=1}^{\infty}a_k$ ... konvergent $\logeq S_n = \sum_{k=1}^{n}a_k$ ... beschränkt
		\item Majorantenkriterium: $\forall k \in \mathbb{N} : a_k \leq b_k$. Falls $\sum_{k=1}^{\infty}b_k$ ... konvergiert, dann gilt $\sum_{k=1}^{\infty}a_k$ ... konvergiert.
		\item Minorantenkriterium: $\forall k \in \mathbb{N} : a_k \leq b_k$. Falls $\sum_{k=1}^{\infty}a_k = +\infty$, dann gilt $\sum_{k=1}^{\infty}b_k = +\infty$.
	\end{itemize}
\end{lemma}

\begin{definition}
	$\sum_{k=1}^{\infty}a_k$ heißt absolut konvergent, falls $\sum_{k=1}^{\infty}|a_k|$ ... konvergent.
\end{definition}

\begin{lemma}
	$\sum_{k=1}^{\infty}a_k$ ... absolut konvergent $\implies \sum_{k=1}^{\infty}a_k$ ... konvergent
\end{lemma}

\begin{remark}
	Die Umkehrung ist falsch.
\end{remark}

\subsection{Konvergenzkriterien}

\begin{theorem}(Wurzelkriterium)
	$\sum_{k=1}^{\infty}a_k$ ... Reihe aus $\mathbb{R} / \mathbb{C}$
	
	\begin{itemize}
		\item Falls $\exists q \in [0,1) \exists N \in \mathbb{N} \forall n \geq N : \sqrt[n]{|a_n|}\leq q$ dann ist die Reihe absolut konvergent.
		\item Falls $\exists$ Teilfolge $(a_{n(j)})_{j \in \mathbb{N}}$ mit $\forall j \in \mathbb{N} : \sqrt[n(j)]{|a_{n(j)|}\geq 1}$ dann ist die Reihe divergent.
	\end{itemize}
\end{theorem}

\begin{theorem}(Quotientenkriterium)
	$\sum_{k=1}^{\infty}a_k$ ... Reihe aus $\mathbb{R} / \mathbb{C}$
	
	\begin{itemize}
		\item Falls $\exists q \in [0,1) \exists N \in \mathbb{N} \forall n \geq N : \frac{|a_{n+1}|}{|a_n|} \leq q$, dann ist die Reihe absolut konvergent.
		\item Falls $\exists N \in \mathbb{N} \forall n \geq N : \frac{|a_{n+1}|}{|a_n|} \geq 1$, dann ist die Reihe divergent.
	\end{itemize}
\end{theorem}

\begin{lemma}
	$a_1,...,a_m,b_1,...,b_m \in \mathbb{R}$ oder $\mathbb{C}$
	
	Dann gilt $\sum_{n=1}^{m}a_nb_n=a_m*\beta_m-\sum_{n=1}^{m-1}(a_{n+1}-a_n)*\beta_n$ mit $\forall n \in \{1,...,m\}: \beta_n = \sum_{j=1}^{n}b_j$
\end{lemma}

\begin{theorem}(Dirichletsches Kriterium)
	$(a_n)$ monotone Nullfolge aus $\mathbb{R}$, $(b_n)$ ... Folge aus $\mathbb{R}$ oder $\mathbb{C}$
	
	$\exists C \geq 0 \forall n \in \mathbb{N}: |\sum_{j=1}^{n}b_j|\leq C \implies \sum_{n=1}^{\infty}a_nb_n$ konvergiert
\end{theorem}

\begin{theorem}(Leibnitz-Kriterium)
	$(a_n)$ ... monotone Nullfolge $\implies \sum_{n=1}^{\infty}(-1)^n*a_n$ konvergiert.
\end{theorem}
\end{document}
