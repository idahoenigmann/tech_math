\documentclass[]{article}
\usepackage[a4paper, margin=2cm]{geometry}
\usepackage{amsmath}
\usepackage{amsfonts}
\usepackage{amssymb}
\usepackage{mathtools}
\usepackage{amsthm}
\usepackage[many]{tcolorbox}
\usepackage{listings}
\usepackage{cancel}

\newtheorem{lemma}{Lemma}
\newtheorem{theorem}{Theorem}
\newtheorem{definition}{Definition}
\newtheorem{algorithm}{Algorithm}
\newtheorem{example}{Example}
\newtheorem{remark}{Remark}
\newtheorem{recap}{Recap}
\newtheorem{corollary}{Corollary}


\tcolorboxenvironment{lemma}{
	colback=yellow!5!white,
	boxrule=0pt,
	boxsep=1pt,
	left=2pt,right=2pt,top=2pt,bottom=2pt,
	oversize=2pt,
	sharp corners,
	before skip=\topsep,
	after skip=\topsep,
}

\tcolorboxenvironment{definition}{
	colback=green!5!white,
	boxrule=0pt,
	boxsep=1pt,
	left=2pt,right=2pt,top=2pt,bottom=2pt,
	oversize=2pt,
	sharp corners,
	before skip=\topsep,
	after skip=\topsep,
}
\begin{document}
	
	\title{Computer Aided Geometric Design Compendium WS2023}
	\author{Ida Hönigmann}
	
	\maketitle

\section*{Orginazation}
Lecture each Thursday 12:00 to 14:00 (full 2 hours).
\\
Oral exam. Write email to fix date and time.
\\
Problem session each Thursday 14:00 to 16:00. Mandatory attendance!
\\
Kreuzerlübung.

\section{Bezier curves}

\begin{example}
	\textbf{Linear combination} $\lambda a + \mu b$
	
	TODO image
	
	\textbf{Affine combination} $\lambda a + \mu b$ and $\lambda + \mu = 1$
	
	TODO image
	
	What is $\mu$ so that $\lambda a + \mu B$ is on the line?
	\begin{align*}
		\lambda a + \mu b = a + t(b-a) \implies a (\underbrace{\lambda - 1 + t}_{=0}) + b (\underbrace{\mu - t}_{=0}) = 0
	\end{align*}
	If $a,b$ are linearly independent $\implies \mu = t \land \lambda + \mu = 1$

	\textbf{Convex combination} $\lambda a + \mu b$ and $\lambda + \mu = 1$ and $\lambda,\mu \geq 0$
	
	TODO image
	
	Line is $a + t(b-a)$ with $t\in[0,1]$ $\implies \mu,\lambda \in [0,1]$
\end{example}

\begin{definition}[combinations]
	linear combination $\sum_{i=1}^{n} \lambda_i v_i$ with $v_1,...,v_n \in \mathbb{R}^d, \lambda_1,...,\lambda_n \in \mathbb{R}$
	
	affine combination $\sum_{i=1}^{n} \lambda_i v_i$ with $\sum_{i=1}^{n} \lambda_i = 1$
	
	convex combination $\sum_{i=1}^{n} \lambda_i v_i$ with $\sum_{i=1}^{n} \lambda_i = 1$ and $\forall i: \lambda_i \geq 0$
\end{definition}

\begin{algorithm}[of de Casteljou, Bezier curve]
	Given: $b_0,...,b_n \in \mathbb{R}^d$ (called control points / Kontrollpunkte), $t \in \mathbb{R}$
	\\Recursion: $b_i^0(t) := b_i$
	\\$b_i^j(t) := (1-t)b_i^{j-1}(t) + tb_{i+1}^{j-1}(t)$ for $j=1,...,n$ and $i=0,...,n-j$
	\\Result: $b(t):=b_0^n(t)$ (called Bezier curve)
	
\end{algorithm}

\begin{remark}
	In the algorithm above often we choose $t\in[0,1]$.
\end{remark}

\begin{example}
	TODO images
\end{example}

\begin{remark}
	In this course $\mathbb{N}=\{1,2,3,...\}$ and $\mathbb{N}_0=\{0,1,2,3,...\}$
\end{remark}

\begin{recap}
	$0! := 1, n! := n(n-1)(n-2) \cdots 1$ for $n \geq 1$.
	
	\begin{align*}
		\binom{n}{k} := \begin{cases}
			\frac{n!}{k!(n-k)!}&, n\geq k \geq 0\\
			0&, k > n
		\end{cases} \text{ for } n,k \in \mathbb{N}_0
	\end{align*}
\end{recap}

\begin{definition}[Bernstein polynomials]
	For $n,i \in \mathbb{N}_0$ we define $B_i^n(t) := \binom{n}{i} t^i (1-t)^{n-i} \in \mathbb{R}[t]$
\end{definition}

\begin{remark}
	Special cases of Bernstein polynomials
	
	\begin{align*}
		i > n \implies B_i^n(t) = 0 && B_i^n(0) = \begin{cases}0, i\not= 0\\ 1, i=0 \end{cases} \\
		B_i^n(1) = \begin{cases}0, i\not= n\\ 1, i=n \end{cases} && B_0^0(t) = 1
	\end{align*}
\end{remark}

\begin{theorem}
	$b_i^j(t) = \sum_{l=0}^{j} B_l^j(t) b_{i+l}$
\end{theorem}

\begin{proof}
	Induction over $j$: $j=0$:
	\begin{align*}
		j=0: && b_i^0(t):= b_i = 1 \cdot b_i = B_0^0(t) \cdot b_i \hspace{1cm} \checkmark\\
		j-1 \rightarrow j: && b_i^j(t) := (1-t)b_i^{j-1}(t) + t b_{i+1}^{j-1}(t) \overset{\mathrm{IA}}{=}
		(1-t) \sum_{l=0}^{j-1} B_l^{j-1}(t) b_{i+l} + t \sum_{l=0}^{j-1} B_l^{j-1}(t) b_{i+1+l} =\\
		&& (1-t) \sum_{l=0}^{\overset{j}{\cancel{j-1}}} B_l^{j-1}(t) b_{i+l} + t \sum_{l=\underset{0}{\cancel{1}}}^{j} B_{l-1}^{j-1}(t) b_{i+l} =
		\sum_{l=0}^{j} (\underbrace{(1-t) B_l^{j-1}(t) + t B_{l-1}^{j-1}(t)}_{=B_l^j(t) \text{ using the following lemma}}) b_{i+l} =\\
		&& \sum_{l=0}^{j} B_l^j(t) b_{i+l} \hspace{1cm} \checkmark
	\end{align*}
\end{proof}

\begin{corollary}
	The Bezier curve equals $b(t) = b_0^n(t) = \sum_{l=0}^{n} B_l^j(t) b_{i+l}$, which is called the Bernstein representation of the Bezier curve.
\end{corollary}

\begin{remark}
	As $b(t) = \sum_{l=0}^{n} B_l^n(t) b_l \in C^\infty$ it is a polynomial curve of degree $n$, which is in $C^\infty$ and therefore ''very smooth''.
\end{remark}

\begin{lemma}
	$B_l^j(t) = (1-t)B_l^{j-1}(t) + t B_{l-1}^{j-1}(t)$
\end{lemma}

\begin{proof}
	\begin{align*}
		(1-t)B_l^{j-1}(t) + t B_{l-1}^{j-1}(t) = (1-t) \binom{j-1}{l} t^l (1-t)^{j-1-l} + t \binom{j-1}{l-1} t^{l-1} (1-t)^{j-1-l+1} =\\
		\binom{j-1}{l} t^l (1-t)^{j-l} + \binom{j-1}{l-1} t^{l} (1-t)^{j-l} = \left(\binom{j-1}{l} + \binom{j-1}{l-1}\right) t^l (1-t)^{j-l} = \binom{j}{l} t^l (1-t)^{j-l} = B_l^j(t)
	\end{align*}
\end{proof}

\begin{remark}
	What is $b(0)$? \hspace{1cm} $b(0)=\sum_{i=0}^{n} B_i^n(0) b_i = b_0 + 0 + 0 + \cdots + 0 = b_0$\\
	What is $b(1)$? \hspace{1cm} $b(1)=\sum_{i=0}^{n} B_i^n(1) b_i = 0 + \cdots + 0 + b_n = b_n$
\end{remark}

\begin{definition}[end-point-interpolating]
	Curves which pass through the first and last point are called end-point-interpolating (Endpunktinterpolierend).
\end{definition}

\begin{remark}
	Bezier curves are end-point-interpolating.
\end{remark}

\begin{remark}
	How many intersection points are there between a planar (i.e. in $\mathbb{R}^2$) Bezier curve and a straight line?
	\begin{align*}
		\text{Straight line: } p + t(q-p) && \text{ Bezier curve: } b(t) = \sum_{i=0}^{n} B_i^n(t) \underbrace{b_i}_{\in \mathbb{R}^2}
	\end{align*}
	
	Solving $p+t(q-p) = \sum_{i=0}^{n} B_i^n(t) b_i$ results in at most $n$ solutions.
\end{remark}

\begin{lemma}
	$\frac{d}{dt}B_i^n(t) = n(B_{i-1}^{n-1}(t) - B_i^{n-1}(t))$
\end{lemma}

\begin{proof}
	\begin{align*}
		\frac{d}{dt}B_i^n(t) = \frac{d}{dt} \binom{n}{i} t^i (1-t)^{n-i} = \binom{n}{i} i t^{i-1} (1-t)^{n-i} - \binom{n}{i} t^i (n-i) (1-t)^{n-i-1} =\\
		\frac{n!}{\underset{(i-1)!}{\cancel{i!}}(n-i)!} \cancel{i} t^{i-1} (1-t)^{n-i} - \frac{n!}{i!\underset{(n-i-1)!}{\cancel{(n-i)!}}}\cancel{(n-i)}t^i (1-t)^{n-i-1} =\\
		n \left(\frac{(n-1)!}{(i-1)!(n-i)!}t^{i-1}(1-t)^{n-i} - \frac{(n-1)!}{i!(n-i-1)!} t^i (1-t)^{n-i-1}\right) =\\
		n \left(\binom{n-1}{i-1}t^{i-1}(1-t)^{n-i} - \binom{n-1}{i} t^i (1-t)^{n-i-1}\right) = n(B_{i-1}^{n-1}(t) - B_i^{n-1}(t))
	\end{align*}
\end{proof}

\begin{theorem}
	$\dot{b}(t) := \frac{d}{dt} b(t) = n \sum_{i=0}^{n-1} B_i^{n-1}(t)(b_{i+1} - b_i) = n (b_1^{n-1}(t) - b_0^{n-1}(t))$
\end{theorem}

\begin{corollary}
	\begin{itemize}
		\item $\dot{b}(0) = n (b_1 - b_0)$
		\item $\dot{b}(1) = n (b_n - b_{n-1})$
		\item The last segment in the algorithm of de Casteljou is the tangent of the Bezier curve in $b(t)$.
		\item The derivative of a bezier curve of degree $n$ is a bezier curve of degree $n-1$ with control points $(b_1,b_0), (b_2 - b_1), \cdots, (b_n - b_{n-1})$.
	\end{itemize}
\end{corollary}

\begin{remark}
	Different applications using these curves are Rhino, OpenSCAD, Autocad, Geogebra, ...
\end{remark}

\end{document}
