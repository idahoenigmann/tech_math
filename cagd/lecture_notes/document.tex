\documentclass[]{article}
\usepackage[a4paper, margin=2.5cm]{geometry}
\usepackage{amsmath}
\usepackage{amsfonts}
\usepackage{amssymb}
\usepackage{mathtools}
\usepackage{amsthm}
\usepackage[many]{tcolorbox}
\usepackage{listings}

\newtheorem{lemma}{Lemma}
\newtheorem{theorem}{Theorem}
\newtheorem{definition}{Definition}
\newtheorem{algorithm}{Algorithm}
\newtheorem{example}{Example}
\newtheorem{remark}{Remark}
\newtheorem{recap}{Recap}

\tcolorboxenvironment{lemma}{
	colback=yellow!5!white,
	boxrule=0pt,
	boxsep=1pt,
	left=2pt,right=2pt,top=2pt,bottom=2pt,
	oversize=2pt,
	sharp corners,
	before skip=\topsep,
	after skip=\topsep,
}

\tcolorboxenvironment{definition}{
	colback=green!5!white,
	boxrule=0pt,
	boxsep=1pt,
	left=2pt,right=2pt,top=2pt,bottom=2pt,
	oversize=2pt,
	sharp corners,
	before skip=\topsep,
	after skip=\topsep,
}
\begin{document}
	
	\title{Computer Aided Geometric Design Compendium WS2023}
	\author{Ida Hönigmann}
	
	\maketitle

\section*{Orginazation}
Lecture each Thursday 12:00 to 14:00 (full 2 hours).
\\
Oral exam. Write email to fix date and time.
\\
Problem session each Thursday 14:00 to 16:00. Mandatory attendance!
\\
Kreuzerlübung.

\section{Bezier curves}

\begin{example}
	\textbf{Linear combination} $\lambda a + \mu b$
	
	TODO image
	
	\textbf{Affine combination} $\lambda a + \mu b$ and $\lambda + \mu = 1$
	
	TODO image
	
	What is $\mu$ so that $\lambda a + \mu B$ is on the line?
	\begin{align*}
		\lambda a + \mu b = a + t(b-a) \implies a (\underbrace{\lambda - 1 + t}_{=0}) + b (\underbrace{\mu - t}_{=0}) = 0
	\end{align*}
	If $a,b$ are linearly independent $\implies \mu = t \land \lambda + \mu = 1$

	\textbf{Convex combination} $\lambda a + \mu b$ and $\lambda + \mu = 1$ and $\lambda,\mu \geq 0$
	
	TODO image
	
	Line is $a + t(b-a)$ with $t\in[0,1]$ $\implies \mu,\lambda \in [0,1]$
\end{example}

\begin{definition}[combinations]
	linear combination $\sum_{i=1}^{n} \lambda_i v_i$ with $v_1,...,v_n \in \mathbb{R}^d, \lambda_1,...,\lambda_n \in \mathbb{R}$
	
	affine combination $\sum_{i=1}^{n} \lambda_i v_i$ with $\sum_{i=1}^{n} \lambda_i = 1$
	
	convex combination $\sum_{i=1}^{n} \lambda_i v_i$ with $\sum_{i=1}^{n} \lambda_i = 1$ and $\forall i: \lambda_i \geq 0$
\end{definition}

\begin{algorithm}[of de Casteljou, Bezier curve]
	Given: $b_0,...,b_n \in \mathbb{R}^d$ (called control points / Kontrollpunkte), $t \in \mathbb{R}$
	\\Recursion: $b_i^0(t) := b_i$
	\\$b_i^j(t) := (1-t)b_i^{j-1}(t) + tb_{i+1}^{j-1}(t)$ for $j=1,...,n$ and $i=0,...,n-j$
	\\Result: $b(t):=b_0^n(t)$ (called Bezier curve)
	
\end{algorithm}

\begin{remark}
	In the algorithm above often we choose $t\in[0,1]$.
\end{remark}

\begin{example}
	TODO images
\end{example}

\begin{remark}
	In this course $\mathbb{N}=\{1,2,3,...\}$ and $\mathbb{N}_0=\{0,1,2,3,...\}$
\end{remark}

\begin{recap}
	$0! := 1, n! := n(n-1)(n-2) \cdots 1$ for $n \geq 1$.
	
	\begin{align*}
		\binom{n}{k} := \begin{cases}
			\frac{n!}{k!(n-k)!}&, n\geq k \geq 0\\
			0&, k > n
		\end{cases} \text{ for } n,k \in \mathbb{N}_0
	\end{align*}
\end{recap}

\begin{definition}[Bernstein polynomials]
	For $n,i \in \mathbb{N}_0$ we define $B_i^n(t) := \binom{n}{i} t^i (1-t)^{n-i} \in \mathbb{R}[t]$
\end{definition}

\begin{remark}
	Special cases of Bernstein polynomials
	
	\begin{align*}
		i > n \implies B_i^n(t) = 0 && B_i^n(0) = \begin{cases}0, i\not= 0\\ 1, i=0 \end{cases} \\
		B_i^n(1) = \begin{cases}0, i\not= n\\ 1, i=n \end{cases} && B_0^0(t) = 1
	\end{align*}
\end{remark}

\begin{theorem}
	$b_i^j(t) = \sum_{l=0}^{j} B_l^j(t) b_{i+l}$
\end{theorem}

\end{document}
