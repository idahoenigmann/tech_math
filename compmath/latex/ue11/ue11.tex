\documentclass[]{article}

\usepackage{latexsym}
\usepackage{amssymb}
\usepackage{amsmath}
\usepackage{enumerate}

%opening
\title{Homework 1}
\author{Ida Hönigmann}

\newtheorem{theorem}{Theorem}
\newenvironment{proof}{\textit{\textbf{Proof}}}{\hfill$\blacksquare$}

\begin{document}

\maketitle

\section*{Problem 1}

\begin{align*}
	p(t)=\textrm{det}(A-t \cdot \textrm{Id}) =
	\begin{array}{|cccc|}
		a_{11}-t & a_{12} & \cdots & a_{1n} \\
		a_{21} & a_{22} - t & \cdots & a_{2n} \\
		\vdots & \vdots & & \vdots \\
		a_{n1}-t & a_{n2} & \cdots & a_{nn} - t \\
	\end{array}
\end{align*}


\section*{Problem 2}

The Gamma function is defined as

\begin{align*}
	\Gamma (x) := \lim\limits_{n \to \infty} \frac{n! n^x}{x(x+1)\cdots (x+n)}
\end{align*}

\noindent
There holds the Weierstraß product representation

\begin{align*}
	\frac{1}{\Gamma (x)} = x \cdot e^{Cx} \cdot \prod_{k=1}^{\infty} (1 + \frac{x}{k}) e^{-k/k} && \textrm{with} && C := \lim\limits_{n \to \infty} (\sum_{k=1}^{n} \frac{1}{k} - \ln n)
\end{align*}


\section*{Problem 3}

Let $f,g : \mathbb{R} \to \mathbb{R}$ be given functions given by

\begin{align*}
	f(x) := \left\{
	\begin{array}{ll}
		-1 & \textrm{if}\ x < - \frac{\pi}{2}, \\
		\sin (x) & \textrm{if}\ - \frac{\pi}{2} \leq x \leq \frac{\pi}{2}, \\
		1 & \textrm{if}\ x > \frac{\pi}{2}. \\
	\end{array}
	\right. & \textrm{and} & 
	g(x) := \left\{
	\begin{array}{ll}
		1 & \textrm{if}\ x \in \mathbb{Q}, \\
		0 & \textrm{if}\ x \in \mathbb{R} \setminus \mathbb{Q}. \\
	\end{array}
	\right.
\end{align*}

\newpage
\section*{Problem 4}

For $q \in \mathbb{R}$, it holds that

\begin{align*}
	\lim\limits_{n \to \infty} q^n = 
	\left\{
	\begin{array}{ll}
		+\infty & \textrm{if}\ q > 1, \\
		1 & \textrm{if}\ q = 1, \\
		0 & \textrm{if}\ -1 < q < 1, \\
		\nexists & \textrm{if}\ q \leq -1. \\
	\end{array}
	\right.
\end{align*}


\section*{Problem 5}

\begin{align*}
	A := \left(
	\begin{array}{ccccc}
		\alpha & 2\alpha & 3\alpha & \cdots & n\alpha \\
		0      & \alpha  & 2\alpha & \ddots & \vdots  \\
		0      & 0       & \ddots  & \ddots & 3\alpha \\
		\vdots & \ddots  & \ddots  & \ddots & 2\alpha \\
		0      & \cdots  & 0       & 0      & \alpha  \\
	\end{array}
	\right) \in \mathbb{R}^{n \times n}_{\textrm{tria}}
\end{align*}


\section*{Problem 6}

\begin{align*}
	\textrm{det}(V) =
	\begin{array}{|cccc|}
		1 & a & a^2 & a^3 \\
		1 & b & b^2 & b^3 \\
		1 & c & c^2 & c^3 \\
		1 & d & d^2 & d^3 \\
	\end{array}
	\\ = a^{3} b^{2} c - a^{3} b^{2} d - a^{3} b c^{2} + a^{3} b d^{2} \\ + a^{3} c^{2} d - a^{3} c d^{2} - a^{2} b^{3} c + a^{2} b^{3} d \\ + a^{2} b c^{3} - a^{2} b d^{3} - a^{2} c^{3} d + a^{2} c d^{3} \\ + a b^{3} c^{2} - a b^{3} d^{2} - a b^{2} c^{3} + a b^{2} d^{3} \\ + a c^{3} d^{2} - a c^{2} d^{3} - b^{3} c^{2} d + b^{3} c d^{2} \\ + b^{2} c^{3} d - b^{2} c d^{3} - b c^{3} d^{2} + b c^{2} d^{3}
	\\ = \left(a - b\right) \left(a - c\right) \left(a - d\right) \left(b - c\right) \left(b - d\right) \left(c - d\right)
\end{align*}

\section*{Problem 7}

\begin{theorem}
	For $a,b \in \mathbb{R}$ and a continuous function $f : (a, b) \to \mathbb{R}$, the following two assertions are equivalent:
	
	\begin{enumerate}[(i)]
		\item $f$ is uniformly continuous.
		\item $f$ has a continuous extension onto the compact interval $[a,b]$, i.e., there exists a function $\hat{f} : [a, b] \to \mathbb{R}$ with $\hat{f}=f(x)$ for all $x \in (a, b)$.
	\end{enumerate}

	In this case the continuous extension $\hat{f}$ is even unique.
\end{theorem}

\begin{proof}
	\textit{(i)} $\implies$ \textit{(ii)}	
	Uniform continuity of $f$ is defined as
	
	\[\exists \delta\ \forall \epsilon\ \forall x, y \in (a, b) : |x - y| < \delta \implies |f(x) - f(y)| < \epsilon \]
	
	Let $(x_n)_{n \in \mathbb{N}}$ be a sequence in $(a, b)$ which converges to $a$. Since every convergent sequence is a Cauchy-sequence it holds that
	
	\[\forall \delta\ \exists N \in \mathbb{N}\: \forall n, m > N : |x_n - x_m| < \delta\]
	
	The uniform continuity now gives us
	
	\[\forall n, m > N : |f(x_n) - f(x_m)| < \epsilon\]
	
	which indicates, that $(f(x_n))_{n \in \mathbb{N}}$ is a Cauchy-Series in $\mathbb{R}$. Now we define $\hat{f}(a):=\lim\limits_{n \to \infty}f(x_n)$.
	
	The same can be done for $(x_n)_{n \in \mathbb{N}}$ converging towards $b$.
	\newline
	
	\noindent
	\textit{(ii)} $\implies$ \textit{(i)} Proof by contradiction. Suppose $\hat{f}$ and therefore $f$ is not uniformly continuous on $(a, b)$. Then there exist sequences $(x_n)_{n \in \mathbb{N}}$, $(y_n)_{n \in \mathbb{N}}$ in $(a, b)$ with
	
	\begin{equation}
		\label{equ:not_uniform_continous}
		\begin{array}{cccc}
			\lim\limits_{n \to \infty} |x_n - y_n| = 0 &
			\textrm{and} &
			|\hat{f}(x_n)-\hat{f}(y_n)|>\epsilon &
			\forall n \in \mathbb{N}
		\end{array}
	\end{equation}
	
	Since $[a, b]$ is compact a subsequence $(x_{n(k)})_{k \in \mathbb{N}}$ with
	
	\[\lim\limits_{k \to \infty}x_{n(k)} = x\]
	
	exists. The subsequence $(y_{n(k)})$ also converges to $x$, since $(y_n - x_n) \to 0$ and
	
	\[\lim\limits_{k \to \infty}y_{n(k)} = \lim\limits_{k \to \infty}(y_{n(k)} - (y_{n(k)} - x_{n(k)})) = \lim\limits_{k \to \infty}x_{n(k)} = x\]
	
	Since $\hat{f}$ is continuous
	
	\[\lim\limits_{k \to \infty}|\hat{f}(x_{n(k)}) - \hat{f}(y_{n(k)})| = |\lim\limits_{k \to \infty} \hat{f}(x_{n(k)}) - \lim\limits_{k \to \infty} \hat{f}(y_{n(k)})| = |\hat{f}(x) - \hat{f}(x)| = 0\]
	
	which contradicts (\ref{equ:not_uniform_continous}).
	
\end{proof}

\section*{Problem 8}

\begin{theorem}
	For real numbers $x, y \in \mathbb{R}$ and $n \in \mathbb{N}$ we have
	
	\begin{align*}
		(x+y)^n = \sum_{k=0}^{n} \binom{n}{k} x^{n-k} y^k.
	\end{align*}
\end{theorem}

\begin{proof}
	Let $x, y \in \mathbb{R}$ be arbitrary. Proof by induction:
	
	\noindent$n = 0:$
	
	\[(x+y)^0 = 1 = \binom{0}{0} x^0 y^0\]
	
	\noindent$n+1:$
	
	\begin{align*}
		(x+y)^{n+1} &= (x+y)^n \cdot (x+y) \\
		&= \left( \sum_{k=0}^{n} \binom{n}{k} x^{n-k} y^k \right) (x+y) \\
		&= \left( \sum_{k=0}^{n} \binom{n}{k} x^{n-k} y^k \right)x +
		   \left( \sum_{k=0}^{n} \binom{n}{k} x^{n-k} y^k \right)y \\
		&= \left( \sum_{k=0}^{n} \binom{n}{k} x^{n-k+1} y^k \right) +
		   \left( \sum_{k=0}^{n} \binom{n}{k} x^{n-k} y^{k+1} \right) \\
		&= \left( \sum_{k=0}^{n} \binom{n}{k} x^{n-k+1} y^k \right) +
		   \left( \sum_{k=1}^{n+1} \binom{n}{k-1} x^{n-(k-1)} y^{k} \right) \\
		&= \left( \sum_{k=1}^{n} \binom{n}{k} x^{n-k+1} y^{k} \right) +
		   \left( \sum_{k=1}^{n} \binom{n}{k-1} x^{n-(k-1)} y^{k} \right) \\
		   &\ \quad + \binom{n}{0} x^{n-0+1} y^0 - \binom{n}{(n+1)-1} x^{n-(n+1)+1)} y^{n+1} \\
		&= \left( \sum_{k=1}^{n} \binom{n}{k} x^{n-k+1} y^{k} \right) +
		   \left( \sum_{k=1}^{n} \binom{n}{k-1} x^{n-k+1} y^{k} \right) \\
		   &\ \quad + \binom{n}{0} x^{n+1} y^0 - \binom{n}{n} x^{0} y^{n+1} \\
		&= \left( \sum_{k=1}^{n} \left(\binom{n}{k} + \binom{n}{k-1}\right) x^{n-k+1} y^{k} \right) + \binom{n}{0} x^{n+1} y^0 - \binom{n}{n} x^{0} y^{n+1} \\
		&= \left( \sum_{k=1}^{n} \binom{n+1}{k} x^{n-k+1} y^{k} \right) + \binom{n+1}{0} x^{n+1} y^0 - \binom{n+1}{n+1} x^{0} y^{n+1} \\
		&= \left( \sum_{k=0}^{n+1} \binom{n+1}{k} x^{(n+1)-k} y^{k} \right) \\
	\end{align*} 

\end{proof}

\end{document}
