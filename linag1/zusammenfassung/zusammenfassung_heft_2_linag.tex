\documentclass[twocolumn]{article}
\usepackage[utf8]{inputenc}
\usepackage[german]{babel}

\usepackage{amsthm}
\usepackage{amsmath}
\usepackage{amsfonts}
\usepackage{mathtools}

\newtheorem{theorem}{Theorem}[section]
\newtheorem{corollary}{Corollary}[theorem]
\newtheorem{lemma}[theorem]{Lemma}
\newtheorem{definition}{Definition}[section]
\newtheorem*{remark}{Bemerkung}
\newtheorem*{schreibweise}{Schreibweise}

%opening
\title{Zusammenfassung Heft 2 LINAG}
\author{Ida Hönigmann}

\newcommand*{\logeq}{\Leftrightarrow}

\begin{document}

\maketitle

\section{Vektorraum}

\begin{definition}
	$m_1, m_2, ..., m_n \in V$, $x_1,...,x_n \in K$
	
	Dann ist $x_1*m_1+x_2*m_2+...+x_n*m_n$ eine Linearkombination des Vektors $m_1,...,m_n$
\end{definition}

\begin{remark}
	Wenn $m_1=m_2$, dann Linearkombination über $m_1,...,m_n$ auch Linearkombination über $m_2,...,m_n$.
\end{remark}

\begin{definition}
	$M \subseteq V$
	
	$[M]\coloneqq\{v\in V \exists n \geq 0 \exists x_1, ..., x_n \in K \exists m_1, ..., m_n \in M : v=\sum_{i=1}^{n}x_i*m_i\}$ heißt die Hülle von M.
	
	Kurz auch: $[M]=\{v \in V : v$ ist Linearkombination von Elementen aus $M\}$.
\end{definition}

\begin{remark}
	$\emptyset \neq M \subseteq V \implies [M]$ ist Unterraum von $V$.
\end{remark}

\begin{lemma}
	$(U_i)_{i \in I}$... Familie von Unterräumen von $V$
	
	Dann ist $\bigcap_{i \in I} U_i$ ein Unterraum von $V$.
\end{lemma}

\begin{lemma}
	$[M]$ kann auch als $\cap\{U:U$ ist Unterraum von $V$, $M \in U\}$ definiert werden.
\end{lemma}

\begin{lemma}
	$V$... Vektorraum, $M$... Menge, $U$... Unterraum, $M \subset U \subset V$
	
	\begin{itemize}
		\item $M \subseteq [M] \subseteq U$
		\item $M_1 \subseteq M_2 \subseteq V \implies [M_1]\subseteq [M_2]$
		\item $[U]=U$
		\item $[[M]]=[M]$
	\end{itemize}
\end{lemma}

\subsection{Basis}

\begin{definition}
	$V$... Vektorraum
	
	$M \subseteq V$ heißt Erzzeugnissystem von $V \logeq [M]=V$ 
\end{definition}

\begin{definition}
	$V$... Vektorraum
	
	$M \subseteq V$ heißt linear abhängig $\logeq \exists a \in M : a \in [M\setminus \{a\}]$
	
	$M \subseteq V$ heißt linear unabhängig $\logeq \forall a \in M : a \notin [M\setminus \{a\}]$
\end{definition}

\begin{definition}
	$V$... Vektorraum, $M \subseteq V$
	
	$M$ ist Basis von $V \logeq M$ ist linear unabhängig $\land M$ ist Erzzeugnissystem.
\end{definition}

\begin{lemma}
	$V$... Vektorraum, $M \subseteq V$
	
	$M$ ist linear abhängig $\logeq \exists \sum_{i=1}^{n}x_i*a_i=0_V$ nicht trivial
\end{lemma}

\begin{lemma}
	$M \subseteq V$
	
	$m \in [M] \logeq [M] = [M\cup\{m\}]$
\end{lemma}

\begin{theorem}
	$V$... Vektorraum, $B \subseteq V$... Basis
	
	$\implies \forall x \in V\setminus\{0_V\}\exists b_1,...,b_n \in B$ verschieden, $\exists x_1,...,x_n \in K \neq 0 : x=\sum_{i=1}^{n}x_i*b_i$ mit $x_i*b_i$ eindeutig.
\end{theorem}

\begin{theorem}
	$V$... Vektorraum, $B \subseteq V$ ... Teilmenge
	
	Folgende Aussagen sind äquivalent:
	
	\begin{itemize}
		\item $B$ ist Basis
		\item $B$ ist ein minimales Erzzeugnissystem
		\item $B$ ist maximal linear unabhängig
	\end{itemize}
\end{theorem}

\begin{theorem}
	$A \subseteq M \subseteq V$ mit $V$... Vektorraum und $A$... linear unabhängig
	
	$\implies \exists Y (A \subseteq Y \subseteq M$ mit $Y$ maximal linear unabhängig $)$.
\end{theorem}

\subsection{Maximalitätsprinzip}

\begin{definition}
	$K$ ist eine Kette, wenn gilt $\forall x,y \in K : x \leq y \lor y \leq x$
\end{definition}

\begin{definition}
	$K$ ist eine maximale Kette, wenn gilt $\forall K' \supset K : K'$ ist keine Kette
\end{definition}

\begin{theorem}
	$(H,\leq)$... Halbordung
	
	$\implies \exists K \subseteq H : K$ ist eine maximale Kette.
\end{theorem}

\subsection{Basis}

\begin{theorem}
	$V$... Vektorraum
	
	$\implies \exists B \subseteq V : B$ ist Basis
\end{theorem}

\begin{lemma}
	Sei $A \subseteq V$ linear unabhängig $\implies \exists B \supseteq A : B$ ist Basis.
	
	Sei $M \subset V$ ein Erzeugungssystem $\implies \exists B \subset M : B$ ist Basis
\end{lemma}

\begin{theorem}
	Jeder Vektorraum $V$ hat eine Basis $B$.
	
	Wenn $A \subseteq V$ linear unabhängig ist $\implies \exists B \supset A$ mit $B$ ist Basis von $V$.
	
	Wenn $M \subseteq V$ Erzeugungssystem von $V$, dann $\exists B \supseteq M$ mit $B$ ist Basis.
\end{theorem}

\begin{lemma}
	$V$... Vektorraum über $K$, $M \subseteq V$, $m \in V$, $m=\sum_{i=1}^{n}x_i*m_i$ mit $x_1,...,x_n \in K^{\times }$ und $m_1,...,m_n \in M$ verschieden
	
	\begin{itemize}
		\item $M$ Erzeugungssystem $\implies \forall i (M\setminus \{m_i\})\cup \{m\}$ ist Erzeugungssystem
		\item $M$ linear unabhängig $\implies \forall i (M \setminus \{m_i\})\cup \{m\}$ ist linear unabhängig
		\item $M$ Basis $\implies \forall i (M \setminus \{m_i\})\cup \{m\}$ ist Basis
	\end{itemize}
\end{lemma}

\begin{theorem}[Austauschsatz von Steinitz]
	Sei $M$ ein Erzeugungssystem von $V$. Sei $A$ eine linear unabhängige Teilmenge von V. Dann gilt:
	
	$\exists \phi : A \rightarrow M$ injektiv, sodass $(M\setminus \phi(A))\cup A$ ein Erzeugungssystem ist.
\end{theorem}

\begin{lemma}
	$V$... Vektorraum über $K$, $B$, $B'$... Basen von $V$
	
	$\implies \exists \phi: B \rightarrow B'$ injektiv $\land \exists \phi':B'\rightarrow B$ injektiv
	
	Alle Basen zu einem Vektorraum sind gleich groß:
	
	$|B| = |B'|, d.h. \exists \psi: B\rightarrow B'$ bijektiv
\end{lemma}

\begin{theorem}[Satz von Cantor-Schröder-Bernstein]
	$C$, $D$... Mengen, $\exists \phi : C\rightarrow D$ injektiv, $\exists \phi' : D \rightarrow C$ injektiv
	
	$\implies \exists \psi : C \rightarrow D$ bijektiv
\end{theorem}

\begin{definition}
	$dimV\coloneqq$ Größe einer beliebigen Basis von $V$
\end{definition}

\begin{remark}
	$dimV=\infty \logeq \neg(dimV endlich)$
\end{remark}

\begin{lemma}
	Sei $V$, $V'$... Vektorräume über $K$
	
	$dimV=dimV'\logeq V \cong V'$, wobei $\cong$ bedeutet, dass $V$ und $V'$ isomorph sind
\end{lemma}

\begin{theorem}
	$V$... Vektorraum über $K$
	
	$V \cong K^{<B>}$, wobei $B \supseteq V$ eine beliebige Basis ist.
\end{theorem}

\begin{definition}
	$V$... Vektorraum über $K$ heißt endlich erzeugt
	
	\begin{itemize}
		\item $\logeq \exists M \subseteq V : [M]=V$
		\item $\logeq V$ hat endliches Erzeugungssystem
		\item $\logeq V$ hat endliche Basis
		\item $dimV$ ist endlich
	\end{itemize}
\end{definition}

\begin{lemma}
	$V$ endlich erzeugt
	
	\begin{itemize}
		\item $M \subseteq V$ Erzeugungssystem $\implies |M| \geq dimV$
		\item $M \subseteq V$ linear unabhängig $\implies |M| \leq dimV$
	\end{itemize}

	Wenn Gleichheit gilt ist $M$ sogar eine Basis.
\end{lemma}

\begin{lemma}
	$V$... Vektorraum, $U_1 \subseteq U_2 \subseteq V$, $U_2$... endlich erzeugt
	
	\begin{itemize}
		\item $dimU_1 \leq dimU_2$
		\item $U_1 \neq U_2 \logeq dimU_1 < dimU_2$
	\end{itemize}
\end{lemma}

\subsection{Elementare Spaltenumformungen}

\begin{definition}
	Es gibt folgende elementare Spaltenumformungen:
	
	\begin{itemize}
		\item Multiplikation einer Spalte mit $c\in K^{\times}$
		\item Vertauschen zweier beliebiger Spalten
		\item Addition eines vielfachen einer Spalte zu einer anderen.
	\end{itemize}
\end{definition}

\begin{lemma}
	$(a_1,...,a_m)$ lässt sich durch elementare Spaltenumformungen zu $(a_1',...a_m')$ umformen $\logeq [\{a_1,...,a_m\}]=[\{a_1',...,a_m'\}]$
\end{lemma}

\begin{definition}
	$n\geq 1$
	
	$Er\coloneqq \begin{pmatrix}
		1 & 0 & \cdots & 0 \\
		0 & 1 & \cdots & 0 \\
		\vdots & \vdots & \ddots & \vdots \\
		0 & 0 & \cdots & 1
	\end{pmatrix} \in K^{r\times r}$
\end{definition}

\begin{lemma}
	$A \in K^{n\times m}$ beliebig.
	
	$\implies \exists r \leq min(n,m)$, sodass $A$ durch elementare Spaltenumformungen zu $\begin{pmatrix}
		Er & 0 & \cdots & 0\\
		\vdots & \vdots & & \vdots\\
		* & 0 & \cdots & 0\\
	\end{pmatrix}$ umgeformt werden kann (bis auf die Reihenfolge der Zeilen).

	Die entstandene Matrix nennt man auch Normalform.
\end{lemma}

\begin{remark}
	$K$... Körper, $n\geq 1$, $a_1,...,a_n \in K^n$ verschieden
	
	$b=\sum_{i=1}^{n}x_i*a_i$ mit $x_1,...,x_m \in K$
	
	Sei $j$ mit $1 \leq j \leq n$, sodass $x_j \neq 0$.
	
	$\implies (a_1,...,a_n)$ lässt sich durch elementare Spaltenumformungen nach $(a_1,...,a_{j-1},b,a_{j+1},...,a_n)$ umformen.
\end{remark}

\begin{lemma}
	$K$... Körper, $n\geq 1$, $a_1,...,a_n,b_1,...,b_n \in K^n$
	
	Falls $[\{a_1,...,a_n\}] = [\{b_1,...,b_n\}]$, dann lässt sich $(a_1,...,a_n)$ durch elementare Spaltenumformungen zu $(b_1,...,b_n)$ umformen.
\end{lemma}

\subsection{Dimenssionssatz}

\begin{definition}
	$V$... Vektorraum, $(U_i|i \in I)$... Familie von Unterräumen von V
	
	$\sum_{i \in I}U_i=\{\sum_{i \in I}u_i|\forall i\in I: (u_i \in U_i) u_i=0$ für fast alle $i \in I\}$
\end{definition}

\begin{remark}
	$\bigcup_{i\in I}U_i \subseteq [\bigcup_{i\in I}U_i]=\sum_{i \in I}U_i$
\end{remark}

\begin{definition}
	$\sum_{i \in I}U_i$ heißt direkt $\logeq \forall j \in I : U_j \cap \sum_{i \in I \setminus \{j\}}U_i = \{0\}$
\end{definition}

\begin{schreibweise}
	Für direkte Summen schreibt man auch $\bigoplus$.
\end{schreibweise}

\begin{lemma}
	$(U_i)_{i \in I}$... Unterraume von V
	
	$S\coloneqq \sum_{i \in I}U_i$ direkt $\logeq \forall s \in S \exists!$ Darstellung $s=\sum_{i \in I}u_i$ wobei $\forall i \in I(u_i \in U_i)$
\end{lemma}

\begin{definition}
	$U$ ... Unterraum von V
	
	$T \subseteq V$ heißt komplementärer Unterraum zu $U \logeq V = U \bigoplus T$ also $U \cap T = \{0\}$.
\end{definition}

\begin{remark}
	$V\setminus U$ ist kein Unterraum von $V$., da $0_V \notin V \setminus U$.
\end{remark}

\begin{theorem}
	$\forall U$ ... Unterraum von $V$, $\exists T$ komplementär zu $U$ (im Allgemeinen nicht eindeutig).
\end{theorem}

\begin{theorem}(Dimenssionssatz)
	$U$, $T$ ... Unterraum von $V$ mit $dim(U) < \infty \land dim(T) < \infty$
	
	$\implies dim(U+T)=dim(U)+dim(T)-dim(U\cap T)$
\end{theorem}

\begin{lemma}
	$U$... Unterraum von $V$, $dim(V) < \infty$, $T$ ... Komplementärraum von $U$
	
	$dim(T)=dim(V)-dim(U)$
\end{lemma}

\section{Lineare Abbildungen}

\begin{definition}
	$V$, $W$... Vektorräume
	
	$f:V\rightarrow W$ heißt linear $\logeq$
	
	\begin{itemize}
		\item $\forall x,y \in V : f(x+y)=f(x)+f(y)$
		\item $\forall c \in K \forall x \in V : f(c*x)=c*f(x)$
	\end{itemize}
\end{definition}

\begin{definition}
	$V$... Vektorraum, $f$... lineare Abbildung
	
	$kerf \coloneqq \{v \in V : f(v) = 0\}$
\end{definition}

\begin{lemma}
	$kerf$ ist ein Unterraum von $V$.
	
	$f$ injektiv $\logeq kerf=\{0\}$
\end{lemma}

\begin{schreibweise}
	$L(V,W)$ ist die Menge aller linearer Abbildungen von $V$ zu $W$.
\end{schreibweise}

\begin{lemma}
	$f \in L(V,W)$
	
	\begin{itemize}
		\item $f(V)$ ist Unterraum von $W$
		\item $T \leq W$ Unterraum von $W \implies f^{-1}(T)$ ist Unterraum von $V$
		\item $\forall M \subseteq V : f([M]) = [f(M)]$
		\item $f$ bijektiv $\implies f^{-1} \in L(W,V)$
	\end{itemize}
\end{lemma}

\begin{lemma}
	$f \in L(V,W)$
	\begin{itemize}
		\item $f$ injektiv $\logeq dim(V) = dim(f(V))$
		\item $f$ surjektiv $dim(W)=dim(f(V))$
	\end{itemize}
\end{lemma}

\begin{lemma}
	$f \in L(V,W)$, $B$... Basis von $V$
	
	\begin{itemize}
		\item $f$ injektiv $\logeq f|_B$ injektiv $\land f(B)$ linear unabhängig
		\item $f$ surjektiv $\logeq f(B)$ Erzeugungssystem von $W$
	\end{itemize}
\end{lemma}

\begin{definition}
	$f \in L(V,W)$
	
	$def(f)\coloneqq dim(kerf)$ heißt der Defekt.
	
	$rg(f)\coloneqq dim(f(V))$ heißt der Rang.
\end{definition}


\begin{theorem}
	$f \in L(V,W)$, $dim(V)<\infty$
	
	$rg(f)+def(f)=dim(V)$
\end{theorem}

\begin{lemma}
	$f \in L(V,W)$, $dim(V)=dim(W)<\infty$
	
	$f$ injektiv $\implies f$ surjektiv
	
	$f$ surjektiv $\implies f$ injektiv
\end{lemma}

\section{Fortsetzungssatz}

\begin{remark}
	$f,g \in L(V,W)$, $M \subseteq V$... Erzeugungssystem
	
	$f=g \logeq f|_M = g|_M$
\end{remark}

\begin{theorem}
	$V,W$... Vektorräume, $B$... Basis von $V$, $f:B\rightarrow W$... Funktion
	
	$\implies \exists! \hat{f}\in L(V,W)$ mit $\hat{f}|_B = f$
\end{theorem}
\end{document}
