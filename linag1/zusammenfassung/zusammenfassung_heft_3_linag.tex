\documentclass[twocolumn]{article}
\usepackage[utf8]{inputenc}
\usepackage[german]{babel}

\usepackage{amsthm}
\usepackage{amsmath}
\usepackage{amsfonts}
\usepackage{mathtools}
\usepackage{tikz}
\usetikzlibrary{matrix}

\newtheorem{theorem}{Theorem}[section]
\newtheorem{corollary}{Corollary}[theorem]
\newtheorem{lemma}[theorem]{Lemma}
\newtheorem{definition}{Definition}[section]
\newtheorem*{remark}{Bemerkung}
\newtheorem*{schreibweise}{Schreibweise}

%opening
\title{Zusammenfassung Heft 3 LINAG}
\author{Ida Hönigmann}

\newcommand*{\logeq}{\Leftrightarrow}

\begin{document}

\maketitle

\begin{lemma}
	Zu jeder linearen Abbildung existiert eine passende Matrix und umgekehrt.
	
	$A \in K^{m\times n}$ ($m$ ... Dimension des Zielraums, $n$ ... Dimension des Urraums)
	
	$\phi_A: K^n \rightarrow K^m$, dann ist $\phi_A \in L(K^n, K^m)$
	
	$f \in L(K^n, K^m) \implies \exists A'\in K^{m\times n} : f = \phi_{A'}$
\end{lemma}

\begin{remark}
	$\Phi : K^{m\times n} \rightarrow L(K^n, K^m), A\mapsto \phi_A$ ist surjektiv, injektiv und linear
\end{remark}

\begin{definition}
	$A \in K^{m \times n}$
	
	$rg(A)\coloneqq dim[\{s_1,...,s_n\}] = dim[\{\phi_A(e_1),...,\phi_A(e_n)\}] = dim \phi_A(K^n) = rg(\phi_A)$ heißt der Spaltenrang von $A$.
	
	$\{e_1,...,e_n\}$... Erzeugnissystem von $K^n \implies \{\phi_A(e_1), ...,\phi_A(e_n)\}$ ... Erzeugnissystem von $\phi_A(K^n)$
	 
\end{definition}

\begin{lemma}
	$\{b_1,...,b_n\}$... Basis von $K^n$, $d_1,...,d_n \in K^m$
	
	$\exists! g \in L(K^n, K^m) : \forall i : g(b_i) = d_i$
	
	Um die Matrix $A \in K^{m \times n}$ mit $g = \phi_A$ zu finden:
	
	$\begin{pmatrix}
		b_1 & b_2 & \cdots & b_n\\
		d_1 & d_2 & \cdots & d_n\\
	\end{pmatrix}
	\rightsquigarrow
	\begin{pmatrix}
		E_n \\
		A \\
	\end{pmatrix}$
	
\end{lemma}

\begin{lemma}
	$f\in L(K^n, K^m)$, $g\in L(K^m, K^l)$
	
	$\implies g \circ f$ linear
	
	$A$ ... Matrix der Abbildung $f$, $B$ ... Matrix der Abbildung $g$
	
	$\implies \phi_B \circ \phi_A = \phi_{B*A}$
\end{lemma}

\begin{remark}
	$A \in K^{n \times m}$, $B \in K^{m \times l}$, $C \in K^{l\times j}$
	
	$\implies C*(B*A) = (C*B)*A$
\end{remark}

\begin{remark}
	$(K^{n \times m},*)$ bildet eine Halbgruppe ($E_n$ ist neutrales Element, unter $*$ abgeschlossen, aber für manche Matrizen existiert kein inverses Element).
\end{remark}

\section{Koordinaten}
\begin{definition}
	$V$... Vektorraum, $B$... Basis von $V$, $x \in V$
	
	$x = \sum_{b \in B}x_b * b$
	
	$B* : V \rightarrow K^{<B>}, x \mapsto <B*,x> : B \rightarrow K, b \mapsto x_b$ (Isomorphismus von $V$ nach $K^{<B>}$)
	
	$b_0 \in B$
	
	$b_0* : V \rightarrow K, x \mapsto x_{b_0}$
	
	Dann ist $b_i$ die i-te Koordinate von $x$.
	
\end{definition}

\begin{remark}
	Die Reihenfolge von $B$ ist bei den Koordinaten wichtig.
	
	$b_0*$ hängt von $B$ ab.
\end{remark}

\begin{schreibweise}
	$b*(x) = <b*,x> \in K$
	
	$B*(x) = <B*,x> \in K^{<B>}$
\end{schreibweise}

\begin{remark}
	$b_1,b_2 \in B$, $b_1 \neq b_2$
	
	$b_1*(b_1)=1$ und $b_1*(b_2)=0$
\end{remark}

\begin{lemma}
	Wenn Funktionen, wie im folgenden Diagramm gelten
\begin{tikzpicture}
	\matrix (m) [matrix of math nodes,row sep=3em,column sep=4em,minimum width=2em]
	{
		V & W \\
		K^{<B>} & K^{<C>} \\};
	\path[-stealth]
	(m-1-1) edge node [left] {$B*$} (m-2-1)
	edge [double] node [below] {$f$} (m-1-2)
	(m-2-1.east|-m-2-2) edge [dashed,-] node [below] {$f'$}
	(m-2-2)
	(m-1-2) edge node [right] {$C*$} (m-2-2);
\end{tikzpicture}

	dann gilt $\exists f'$ ... linear (also $\exists$ Matrix die $f'$ beschreibt).
\end{lemma}

\begin{remark}
	$f = (C*)^{-1}\circ f' \circ B*$
\end{remark}

\begin{schreibweise}
	Die Matrix die $f'$ beschreibt wird auch $<C*,f(B)>$ gennant.
\end{schreibweise}

\begin{remark}
	$f \in L(V,W)$, $B$... Basis von $V$, $C$... Basis von $W$
	
	$rg(<C*,f(B)>)=rg(f)$
\end{remark}

\begin{remark}
	$<D*,id(B)>$ wandelt zwischen Koordinatendarstellungen um.
\end{remark}

\section{Lineare Selbstabbildungen}
\begin{lemma}
	$(\{f:f\in L(V,V) | f$ ... bijektiv$, \circ\})$ bildet eine Gruppe.
\end{lemma}

\begin{schreibweise}
	Für die Menge aller bijektiven Funktionen aus $L(V,V)$ schreibt man $GL(V)$.
\end{schreibweise}

\begin{definition}
	$A \in K^{n\times n}$ heißt regulär $\logeq \exists B \in K^{n \times n} : B * A = E_n$
\end{definition}

\begin{theorem}
	$A \in K^{n \times n}$ regulär $\logeq \phi_a : K^n \rightarrow K^n$ bijektiv
	
	$(\{A \in K^{n \times n} | A $ regulär$\}, *)$ bildet eine Gruppe $\cong GL(V)$
\end{theorem}

\begin{definition}
	Wenn $A$ eine Matrix ist, dann heißt $A^{-1}$ die inverse Matrix, wenn gilt $A*A^{-1} = A^{-1} * A = E_n$
\end{definition}

\begin{lemma}
	Um von $A$ zu $A^{-1}$ zu kommen:
	
	$\begin{pmatrix}
		E_n \\
		A \\
	\end{pmatrix}
	\rightsquigarrow
	\begin{pmatrix}
		A^{-1} \\
		E_n \\
	\end{pmatrix}$
\end{lemma}

\begin{remark}
	$(GL_n(K),*)$ ist eine nicht kommutative Gruppe $\forall n \geq 2$.
\end{remark}

\section{Vektorräume linearer Abbildungen}

\begin{lemma}
	$V$, $W$ ... Vektorräume
	
	$W^V\coloneqq \{f : V \rightarrow W\}$ ist ein Vektorraum über $K$
	
	$L(V,W)$ ist ein Unterraum von $W^V$.
\end{lemma}

\begin{lemma}
	$V$, $W$ ... Vektorraum, $B=\{b_1,...,b_n\}$ ... Basis von $V$, $C=\{c_1,...,c_m\}$
	
	$\Phi : L(V,W)\rightarrow K^{m\times n}, f \mapsto <C*,f(B)>$
	
	$\Phi$ ist ein Isomorphismus.
	
	Insbesondere ist $dim(L(V,W)) = dim(K^{m \times n})$
\end{lemma}

\begin{definition}
	$c*A=\begin{pmatrix}
		c*a_{11} & \cdots & c*a_{1n} \\
		\vdots &  & \vdots \\
		c*a_{m1} & \cdots & c*a_{mn} \\
	\end{pmatrix}$
\end{definition}

\begin{remark}
	
	Für lineare Abbildungen $f$, $g$ und $h$ gilt:
	
	$g \circ (c*f) = c*(g \circ f) = (c*g) \circ f$
	
	$(g+h)\circ f = (g \circ f) + (h \circ f)$
	
	$g \circ (f+h) = (g \circ f) + (g \circ h)$
	
	Für die dazugehörigen Matrizen $A$, $B$ und $C$ gilt:
	
	$B*(c*A) = c*(B*A) = (c*B) * A$
	
	$(B+C)*A = (B*A)+(C*A)$
	
	$A*(B+C) = (A*B) + (A*C)$
\end{remark}

\end{document}
