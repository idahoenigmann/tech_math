\documentclass[]{article}
\usepackage[a4paper, margin=2cm]{geometry}
%\usepackage[ngerman]{babel}
\usepackage{amsmath}
\usepackage{amsfonts}
\usepackage{amssymb}
\usepackage{mathtools}
\usepackage{amsthm}
\usepackage[many]{tcolorbox}
\usepackage{listings}
\usepackage{cancel}
\usepackage{graphicx}
\usepackage{subfigure}
\usepackage{hyperref}
\usepackage{biblatex}
\usepackage{wasysym}

\newtheorem{lemma}{Lemma}
\newtheorem{theorem}{Satz}
\newtheorem{definition}{Definition}
\newtheorem{algorithm}{Algorithmus}
\newtheorem{example}{Beispiel}
\newtheorem{remark}{Bemerkung}
\newtheorem{recap}{Wiederholung}
\newtheorem{corollary}{Korollar}

%opening
\title{Numerik WS 2021/22}
\author{Ida Hönigmann}

\begin{document}

\maketitle
\tableofcontents
\newpage

\section{Grundbegriffe}
\subsection{Gegenstand der Numerischen Mathematik}

Von der Realität bis zur Interpretation einer Simulation ist es ein langer Weg.

\begin{itemize}
	\item \textbf{Mathematisches Modell} versucht mit Hilfe von mathematischen Formeln (idR. Differentialgl.) die Realität zu beschreiben.
	\item Die wenigsten Lösungen dieser mathematischen Modelle kann man exakt berechnen, d.h. man approximiert die exakte Lösung mittels \textbf{numerischer Simulation} am Rechner.
	\item Diese \textbf{numerische Lösung} wird dann interpretiert und man hofft, dass diese Interpretation die Realität beschreibt.
\end{itemize}

Jede numerische Simulation zerfällt in kleinere \textbf{numerische Probleme}, die geeignet zu lösen sind. Die elementarsten numerischen Probleme sind Gegenstand dieser Vorlesung.

\begin{example}
	\begin{itemize}
		\item Wie approximiert man komplizierte Funktionen mittels einfacher Funktionen (z.B. stückweise Polynome)?
		\item Wie berechnet man Grenzwerte (z.B. Integral, Differential)?
		\item Wie löst man lineare / nichtlineare Gleichungen?
	\end{itemize}
\end{example}

Jede numerische Simulation ist fehlerbehaftet.

\begin{itemize}
	\item \textbf{Modellfehler:} Das mathematische Modell vereinfacht die Realität.
	\item \textbf{Datenfehler:} Die Eingangsdaten einer Simulation stammen meistens aus physikalischen Messungen und haben daher eine gewisse Mess(un-)genauigkeit.
	\item \textbf{Rundungsfehler:} Auf Rechnern ersetzt die endliche Menge an Gleitkommazahlen das kontinuierliche $\mathbb{R}$, d.h. sowohl die Daten als auch die Rechnungen sind rundungsfehlerbehaftet.
	\item \textbf{Verfahrensfehler:} Viele Probleme werden mathematisch in unendlich dimensionalen Räumen oder mit Limiten formuliert. Beides steht im Rechner nicht zur Verfügung und muss diskretisiert werden.
\end{itemize}

In der Vorlesung liegt unser Hauptaugenmerk auf dem Verfahrensfehler und dem Aufwand zugehöriger Algorithmen.


\subsection{Kondition und Stabilität}

Betrachte ein abstraktes Problem. Werte $\Phi: X \rightarrow Y$ bei $x\in X$ aus, wobei $X,Y$ geeignete normierte Räume sind. Die \textbf{Kondition eines Problems} besagt, wie stark Änderungen in $x$ (z.B. Rundungsfehler) sich auf $\Phi(x)$ auswirken.

\begin{definition}
	Das Problem ist \textbf{schlecht konditioniert bzgl. absolutem Fehler}, wenn es eine kleine Störung $\tilde{x}$ von $x$ gibt mit $||\Phi(x)-\Phi(\tilde{x})|| \gg ||x-\tilde{x}||$.
	
	Das Problem ist \textbf{schlecht konditioniert bzgl. relativem Fehler}, falls $x \neq 0 \neq \Phi(x)$ und es ex. eine kleine Störung $\tilde{x}$ von $x$ gibt mit $\frac{||\Phi(x)-\Phi(\tilde{x})||}{||\Phi(x)||} \gg \frac{||x-\tilde{x}||}{||x||}$.
	
	Andernfalls bezeichnet man das Problem als \textbf{gut konditioniert (bzgl. abs./rel. Fehler)}.
\end{definition}

\begin{remark}
	Ist $\Phi$ stetig differenzierbar, d.h. $\Phi(x) - \Phi(\tilde{x}) = D\Phi(x)(x-\tilde{x}) + o(||x-\tilde{x}||)$ für $\tilde{x}\rightarrow x$ so beschreibt die Ableitung $D\Phi(x) \in L(X,Y)$ wie stark sich Änderungen in $x$ auf den Fehler auswirken.
	
	Deshalb bezeichnet man $\kappa_{abs}(x) = ||D\Phi(x)||$, $\kappa_{rel}(x) = \frac{||D\Phi(x)|| \cdot ||x||}{||\Phi(x)||}$ als \textbf{Konditionszahlen (bzgl. abs./rel. Fehler)}, d.h. man ist gut konditioniert für $\kappa_{abs}, \kappa_{rel}$ vergleichsweise klein.
\end{remark}

\begin{definition}
	Es sei $\tilde{\Phi}$ eine algorithmische Umsetzung von $\Phi$. Der Algorithmus $\tilde{\Phi}$ ist \textbf{instabil}, wenn es eine kleine Störung $\tilde{x}$ von $x$ gibt, sodass $\underbrace{||\Phi(x)-\tilde{\Phi}(\tilde{x})||}_{\text{tatsätlicher Fehler im Rechner}} \gg \underbrace{||\Phi(x)-\Phi(\tilde{x})||}_{\text{unvermeidlicher Fehler}}$.
	
	Andernfalls ist der Algorithmus stabil.
\end{definition}

\begin{remark}
	Mir ist bewusst, dass die Symbolik $\gg$ (''wesentlich größer'') ungenauer ist, als Sie es aus anderen Vorlesungen kennen. Aber ''schlecht konditioniert'' und ''instabil'' hängt halt an der Genauigkeit der Daten und den Erfordernissen des Nutzers!
\end{remark}

In der Vorlesung geht es primär um die Asymptotik, d.h. was könnte im Worst-Case passieren.

\textbf{Erinnerung/Warnung:} Die Arithmetik im Rechner erfüllt weder Assoziativität noch Distributivgesetz, d.h. die Reihenfolge (und Formulierung) der Rechenoperatoren spielt eine Rolle für Stabilität.

\begin{example}[schlechte Kondition bei Auslöschung]
	Als \textbf{Auslöschung} bezeichnet man das Phänomen, dass bei Subtraktion zweier annähernd gleicher Zahlen im Rechner die hinteren Ziffern (welche rundungsbehaftet sind) signifikant werden. Der relative Fehler kann sogar beliebig groß werden, d.h. $\Phi: \mathbb{R}^2 \rightarrow \mathbb{R}, (x,y) \mapsto x-y$ hat $\kappa_{rel}(x,y) = \frac{\sqrt{2} ||(x,y)||_2}{|x-y|} \gg 0$ für $x\approx y$.
\end{example}

\textbf{Achtung:} Oft ist ein Problem gut konditioniert, wird aber in Teilprobleme zerlegt (im Algorithmus) sodass der resultierende Algorithmus instabil wird.

\begin{example}
	Werte $\Phi(x) = \frac{1}{x+1} - \frac{1}{x}$ für $x \gg 0$ aus.
	
	\begin{align*}
		\Phi'(x) = - \frac{1}{(x+1)^2} + \frac{1}{x^2} = \frac{(x+1)^2 - x^2}{x^2(x+1)^2} = \frac{2x+1}{x^2(x+1)^2}\\
		\Phi(x) = \frac{x-(x+1)}{x(x+1)} = \frac{-1}{x(x+1)}\\
		\kappa_{rel}(x) = \frac{|\Phi'(x)| \cdot |x|}{|\Phi(x)|} = \frac{(2x+1)}{x^2(x+1)^2} x^2 (x+1) = 1 + \frac{x}{x+1} \leq 2
	\end{align*}
	
	$\implies$ gut konditioniert!	
\end{example}

\begin{example}
	Es sei $||.||$ eine Norm auf $\mathbb{K}^n$ und wir verwenden dieselbe Notation für die induzierte Operatornorm $||A|| := \sum_{x\in \mathbb{K}^n, x \neq 0}\frac{||Ax||}{||x||}$ für $A \in \mathbb{K}^{n\times n}$.
	
	Ist $A$ invertierbar, so bezeichnet $cond(A) := ||A|| \cdot ||A^{-1}||$ die \textbf{Konditionszahl von $A$ (bzgl. $||.||$)}. Betrachtet man das Lösungsproblem $\Phi:\mathbb{K}^n \rightarrow \mathbb{K}^n, \Phi(b) = A^{-1}b$, so gilt für die relative Konditionszahl (mit $x=A^{-1}b$)
	\begin{align*}
		\kappa_{rel}(b) = \frac{||D\Phi(b)|| \cdot ||b||}{||\Phi(b)||} = \frac{||A^{-1}|| \cdot ||Ax||}{||x||} \leq cond(A),
	\end{align*}
	wobei die letzte Abschätzung \textbf{scharf ist}, d.h. es gilt Gleichheit für mindestens ein $b$ (und ein $x$).
\end{example}

\subsection{Verfahrensfehler}

Im Wesentlichen gibt es zwei Arten von Verfahrensfehlern

\begin{itemize}
	\item \textbf{Abbruchfehler}, wenn ein konvergenter (aber unendlicher) Algorithmus nach endlich vielen Schritten abgebrochen wird.
	\item \textbf{Diskretisierungsfehler}, wenn eine kontinuierliche Größe durch eine diskrete vereinfacht wird, z.B. Differenzenquotienten statt Differenzialquotient.
\end{itemize}

\begin{example}[Abbruchfehler Heron-Verfahren]
	Für $x>0$ def. $y_1 := \frac{1}{2} (1+x), y_{n+1} := \frac{1}{2}(y_n + \frac{x}{y_n})$.
	
	\begin{align*}
		\implies y_{n+1}^2 - x = \frac{1}{4} (y_n^2 + 2x + \frac{x^2}{y_n}) - x = \frac{1}{4} (y_n - \frac{x}{y_n})^2 \geq 0\\
		\implies y_{n+1}^2 \geq x > 0 \text{ und } y_{n+1} - y_n = \frac{1}{2} (y_n + \frac{x}{y_n}) - y_n = \frac{x}{2y_n} - \frac{y_n}{2} = \frac{x-y_n^2}{2y_n} \leq 0\\
		\implies 0 < \sqrt{x} \leq y_{n+1} \leq y_n \implies y_n \rightarrow y\\
		\implies y = \frac{1}{2}(y + \frac{x}{y}) \implies \frac{1}{2}y^2 = \frac{1}{2}y^2 + \frac{1}{2}x \implies y^2 = x \implies y = \sqrt{x}
	\end{align*}
	
	$\implies (y_n)$ konvergiert monoton fallend gegen $\sqrt{x}$. Sobald $y_n = \sqrt{x}$, würde auch $y_{n+1} = \sqrt{x}$ gelten, d.h. endkonstante Folge.
\end{example}

\textbf{Später}: Heron-Verfahren ist tatsächlich \textbf{quadratisch konvergent}, d.h. ex. $C>0$ mit $|\sqrt{x}-y_{n+1}| \leq C |\sqrt{x}-y_n|^2$. $\implies$ schnelle konvergenz, weil sich korrekte Ziffern pro Schritt verdoppeln!

\begin{example}[Diskretesierungsfehler einseitiger Differenzenquotient]
	Sei $f:\mathbb{R} \rightarrow \mathbb{R}$ differenzierbar, $x\in \mathbb{R}$
	
	Ziel: Approximiere $\Phi:=f'(x)$ durch den einseitigen Diffquot. $\Phi_h = \frac{f(x+h)-f(x)}{h}$
	
	klar: $\Phi_h \rightarrow \Phi$ für $h \rightarrow 0$, aber die Konvergenz kann beliebig langsam sein. Man interessiert sich in der Numerik auch für Konvergenzraten bzgl. des Diskretisierungsparameters.
	
	Für $f \in \mathcal{C}^2$ (lokal um $x$) gilt nach Mittelwertsatz
	\begin{align*}
		f'(x) - \frac{f(x+h)-f(x)}{h} = f'(x) - f'(\zeta) = f''(\xi)(x-\zeta)
	\end{align*}
	mit Zwischenstellen $x \leq \xi \leq \zeta \leq x+h$
	\begin{align*}
		\implies |\Phi - \Phi_n| \leq ||f''||_{L^{\infty}(x,x+h)} h = \mathcal{O} (h)
	\end{align*}
	d.h. hier Konvergenzrate $1$ in $h$.
\end{example}

\begin{definition}
	Es sei $\Phi$ eine kontinuierliche Große mit Diskretisierung $\Phi_h$ für $h>0$. Dann bezeichnet man eine Abschätzung der Form $|\Phi - \Phi_h| = \mathcal{O}(h^\alpha)$ als \textbf{a-priori Fehlerabschätzung} mit \textbf{Konvergenzrate} $\alpha > 0$ (auch \textbf{Konvergenzordung}).
\end{definition}

Natürlich interessiert sich die Numerik für Verfahren, bei denen $\alpha > 0$ möglich groß ist.

\begin{example}[zentraler Differenzenquotient]
	$f:\mathbb{R} \rightarrow \mathbb{R}$ diffbar, $x\in \mathbb{R}$, $\Phi := f'(x)$ und $\Phi_h := \frac{1}{2} \left(\frac{f(x+h)-f(x)}{h} + \frac{f(x)-f(x-h)}{h}\right)$
	
	klar: $\Phi_h \rightarrow \Phi$ für $h \rightarrow 0$, $|\Phi - \Phi_h| = \mathcal{O}(h)$ sofern $f \in \mathcal{C}^2$ (lokal um $x$).
	
	Für $f \in \mathcal{C}^3$ (lokal um $x$) gilt mit Taylor $f(x+h) = f(x) + hf'(x) + \frac{h^2}{2}f''(x) + \frac{h^3}{6}f'''(\zeta)$ für geeignete $x-h \leq \zeta_- \leq x \leq \zeta_+ \leq x+h$
	\begin{align*}
		\implies \frac{f(x+h)-f(x)}{h} = f'(x) + \frac{h}{2} f'(x) + \frac{h^2}{6} f'''(\zeta_-)\\
		\frac{f(x)-f(x-h)}{h} = f'(x) - \frac{h}{2} f'(x) + \frac{h^2}{6} f'''(\zeta_+)\\
		\implies |\Phi - \Phi_h| = \frac{h^2}{6} \frac{|f'''(\zeta_+) + f'''(\zeta_-)|}{2} = \mathcal{O}(h^2)
	\end{align*}
	d.h. Konvergenzrate $\alpha = 2$.
	
	$\implies$ höhere Genauigkeit für gleiches $h$ bzw. gleiche Genauigkeit für größeres $h$.
\end{example}

\begin{remark}
	Auslöschung tritt immer auf (insb. bei Diffquot.), aber sie wird abgemildert durch Verfahren höherer Ordnung. Eine andere Möglichkeit für Verfahren höherer Ordnung zur Approximation von $f'(x)$ ist die Verwendung von Polynomapproximation, d.h. $f \approx p$ Polynom und berechne $p'(x) \approx f'(x)$.
\end{remark}

\section{Interpolation}

Bei einem \textbf{Interpolationsproblem} sind im einfachsten Fall Paare $(x_j, y_j)$ gegeben und eine ''einfache'' Funktion $p$ mit $p(x_j)=y_j \forall j$ gesucht, z.B. Polynome, Splines (= stückweise Polynome), rationale Funktionen (= Quotienten von Polynomen). Verwandt, aber mathematisch schwieriger sind \textbf{Approximationsprobleme}. Dabei ist eine Funktion $f$ und eine Norm $||.||$ gegeben, und es wird eine einfache Funktion $p$ gesucht, die $||f-p||$ in dieser Klasse einfacher Fkt. minimiert. Oft ist dabei die Funktion $f$ nur implizit gegeben, d.h. unbekannt.

\subsection{Lagrange-Polynominterpolation}

\textbf{Problemstellung:} Gegeben sind $n+1$ reelle \textbf{Stützstellen} $a \leq x_0 < ... < x_n \leq b$ und \textbf{Funktionswerte} $y_0, .., y_n \in \mathbb{K}$. Die \textbf{Lagrange-Interpolationsaufgabe} sucht ein Polynom $p \in \mathbb{P}_n = \{p(x)=\sum_{j=0}^{n}a_j x^j | a_0, ..., a_n \in \mathbb{K}\}$ vom Grad $n$ mit $p(x_j) = y_j \forall j=0, ..., n$

\begin{lemma}
	\begin{enumerate}
		\item $\mathbb{P}_n$ ist $\mathbb{K}$-Vektorraum mit $\dim \mathbb{P}_n = n+1$.
		\item Die \textbf{Monome} $p_j(x)=x^j, j=0, ..., n$ sind eine Basis von $\mathbb{P}_n$.
		\item Die \textbf{Lagrange-Polynome} $L_j(x) = \prod_{k=0, k\neq j}^{n} \frac{x-x_k}{x_j-x_k} \in \mathbb{P}_n$ erfüllen $L_j(x_k) = \delta_{jk}$ für alle $j,k=0, ..., n$ und bilden eine Basis von $\mathbb{P}_n$.
		\item Die \textbf{Newton-Polynome} $q_j(x) = \prod_{k=0}^{j-1} (x-x_k) \in \mathbb{P}_j$ für $j=0, ..., n$ bilden eine Basis von $\mathbb{P}_n$.
	\end{enumerate}
\end{lemma}

\begin{proof}
	klar: $\mathbb{P}_n$ ist $\mathbb{K}$-Vektorraum, $\dim \mathbb{P}_n \leq n+1$,
	
	zz: $\{L_0, .., L_n\} \subseteq \mathbb{P}_n$ lin. unab.
	
	Sei $\mu_0, ..., \mu_n \in \mathbb{K}$ mit $\sum_{j=0}^{n} \mu_j L_j(x) = 0 \forall x$
	
	Für $x=x_k$ folgt
	
	\begin{align*}
		0 = \sum_{j=0}^{n} \mu_j \underbrace{L_j(x_k)}_{=\delta_{jk}} = \mu_k
	\end{align*}
	
	$\implies$ lin. unab. laut Def. $\implies \dim \mathbb{P}_n \geq n+1 \implies$ Monome + Lagrange Pol. bilden Basis von $\mathbb{P}_n$.
	
	zz. $\{q_0, ..., q_n\} \subseteq \mathbb{P}_n$ lin. unab.
	
	Seien $\mu_0, ..., \mu_n \in \mathbb{K}$ mit $\sum_{j=0}^{n} \mu_j \underbrace{q_j(x)}_{=\prod_{k=0}^{j-1}(x-x_k)} = 0$.
	
	Für $x=x_0$ folgt $\mu_0 q_0(x) = 0 \implies \mu_0 = 0$.
	
	Für $x=x_1$ folgt $\mu_1 \underbrace{q_1(x)}_{\neq 0} = 0$, also $\mu_1 = 0$. Induktives Vorgehen zeigt $\mu_j=0 \forall j$.
\end{proof}

\begin{theorem}[Eindeutigkeit + Existenz]
	Betrachte Lagrange-Interpolation zu Stützstellen $a \leq x_0 < ... < x_n \leq b$ und Funktionswerten $y_0, ..., y_n \in \mathbb{K}$. Dann existiert ein eindeutiges $p \in \mathbb{P}_n$ mit $p(x_j) = y_j \forall j$. Dieses wird gegeben durch $p = \sum_{j=0}^{n} y_j L_j$. Ist $\{q_0, ..., q_n\} \subseteq \mathbb{P}_n$ eine Basis von $\mathbb{P}_n$ und $p = \sum_{j=0}^{n} \lambda_j q_j$, so löst $\lambda = (\lambda_0, ..., \lambda_n)$ das lineare Gleichungssystem
	\begin{align*}
		\underbrace{\left(\begin{matrix}
			q_0(x_0) & ... & q_n(x_0)\\
			\vdots & & \vdots\\
			q_0(x_n) & ... & q_n(x_n)
		\end{matrix}\right)}_{=: A}\lambda = 
		\left(\begin{matrix}
			y_0\\ \vdots\\ y_n
		\end{matrix}\right)
	\end{align*}
	
	Die Matrix $A$ ist regulär, d.h. $\lambda$ ist die eindeutige Lösung.
\end{theorem}

\begin{proof}
	Da $L_j(x_k) = \delta_{jk} \forall j,k$ ist offensichtlich, dass $p=\sum_{j=0}^{n} \mu_j L_j$ genau dann das Interpolationsproblem löst, wenn $\mu_j=y_j \forall j$. $\implies$ Eindeutigkeit + Existenz
	
	Def. Lösungsoperator $\mathcal{P}: \mathbb{K}^{n-1} \rightarrow \mathbb{P}_n$ durch $(\mathcal{P}y)(x_j) = y_j \forall j=0, ..., n \forall y \in \mathbb{K}^{n+1}$
	
	$\implies$ wohldef, bijektiv
	
	Def. Auswertungsoperator $\mathcal{A}: \mathbb{P}_n \rightarrow \mathbb{K}^{n+1}, p \mapsto (p(x_0), ..., p(x_n))$
	
	$\implies$ wohldef, linear
	
	$\mathcal{P} \circ \mathcal{A} =$ Identität, $\mathcal{A} \circ \mathcal{P} =$ Identität,
	
	$\implies \mathcal{A} = \mathcal{P}^{-1}, \mathcal{P} = \mathcal{A}^{-1}$
	
	$\implies A$ ist die darstellende Matrix $\mathcal{A}$. $\implies A$ ist regulär, da $\mathcal{A}$ bijektiv, linear.
\end{proof}

\begin{remark}
	Die Konditionszahl $cond(A)$ der sogenannten \textbf{Vandermonde-Matrix} $A$ hängt stark von der Wahl der Basis ab. Für die Lagrange-Polynome wäre $A$ die Identität. Für die Monome ist $cond(A)$ in der Regel indiskutabel schlecht (hängt an der Wahl der $x_j$). Die Basiswahl beeinflusst auch die Besetzungsstruktur der Matrix.
\end{remark}

\begin{example}
	Die Newton-Basis führt auf eine untere Dreiecksmatrix
	\begin{align*}
		\left(\begin{matrix}
			1 & 0 & ... & 0\\
			1 & q_1(x_1) & ... & 0\\
			\vdots & \vdots & \ddots & \vdots\\
			1 & q_1(x_n) & ... & q_n(x_n)\\
		\end{matrix}\right)
	\end{align*}
	d.h. das lineare GLS kann in $\mathcal{O}(n^2)$ statt $\mathcal{O}(n^3)$ gelöst werden.
\end{example}

\begin{lemma}[Horner-Schema]
	Sei $p(x) = \sum_{j=0}^{n} \lambda_j \underbrace{q_j(x)}_{\prod_{k=0}^{j-1}(x-x_k)}$
	
	Für einen Auswertungspunkt $x \in \mathbb{R}$ betrachte
	\begin{itemize}
		\item $y = \lambda_n$
		\item for $k=n-1:-1_0$
		\item $y = (x-x_k)y + \lambda_k$
		\item end
	\end{itemize}
	
	$\implies$ Der Algorithmus berechnet in $3n$ Operationen den Funktionswert $y=p(x)$.
\end{lemma}

\begin{theorem}[Interpolationsfehlerdarstellung]
	Sei $f \in \mathcal{C}^{n+1}[a,b], 0 \leq m \leq n, p \in \mathbb{P}_n$ mit $p(x_j) = f(x_j) \forall j=0, ..., n$, wobei $a \leq x_0 < ... < x_n \leq b$
	\begin{align*}
		\implies f^{(m)}(x) - p^{(m)}(x) = \frac{f^{(n+1)(\xi)}}{(n+1-m)!} \prod_{l=0}^{n-m}(x-\zeta_l),
	\end{align*}
	wobei $\xi = \xi(m,x)$ und $\zeta_l = \zeta_l(m,x,x_0,..,x_n)$ in $[a,b]$
	
	Für $m=0$ gilt $\zeta_l = x_l \forall l$.
\end{theorem}

\begin{proof}
	$e := f-p \in \mathcal{C}^{n+1}[a,b]$
	
	$\implies e$ hat mindestens $n+1$ Nullstenen (bei $x_l$) $\implies e'$ hat mindestens $n$ Nullstellen $\implies e^{(m)}$ hat mindestens $n+1-m$ Nullstellen $a < \zeta_0 < ... < \zeta_{n-m} < b$
	
	o.B.d.A. $x \notin \{\zeta_0, ..., \zeta_{n-m}\}$
	
	Def. $F(y) := e^{(m)}(x)w(y) - e^{(m)}(y) w(x)$ mit $w(x) := \prod_{l=0}^{m-n}(y-\zeta_l)$
	
	$\implies F$ hat $n+2-m$ Nullstellen $\implies F^{(n+1-m)}$ hat mind. $1$ Nullstelle $\xi$
	
	\begin{align*}
		0 = F^{(n+1-m)(\xi)} = \underbrace{e^{(m)}(x)}_{=f^{(m)}(x) - p^{(m)}(x)} \underbrace{w^{(n+1-m)}(\xi)}_{=(n+1-m)!} - \underbrace{e^{(n+1)}(\xi)}_{=f^{n+1}(\xi)} \underbrace{w(x)}_{=\prod_{l=0}^{n-m}(x-\zeta_l)}
	\end{align*}
\end{proof}

\begin{corollary}[Interpolationsfehler-Abschätzung]
	Seien $f \in \mathcal{C}^{n+1}[a,b]$ reell- oder komplexwertig, $a \leq x_0 < ... < x_n \leq b, p \in \mathbb{P}_n$ mit $p(x_j)=f(x_j) \forall j=0, ..., n, 0 \leq m \leq n$
	\begin{align*}
		\implies ||f^{(m)}-p^{(m)}||_{L^\infty(a,b)} \leq C_\mathbb{K} \frac{||f^{(n+1)}||_{L^\infty(a,b)}}{(n+1-m)!} (b-a)^{n+1-m}
	\end{align*}
	mit $C_{\mathbb{K}}=1$ für reelwertiges $f$, $C_{\mathbb{K}} = 2$ für komplexwertiges $f$.
\end{corollary}

\begin{proof}
	klar für $\mathbb{K} = \mathbb{R}$.
	
	Für $\mathbb{K} = \mathbb{C}$, betrachte $Re(f), Im(f) \in \mathcal{C}^{n+1}[a,b]$.
\end{proof}

\begin{remark}
	Aus der Fehlerabschätzung und der Konvergenz der Exponentialreihe $exp(y) = \sum_{k=0}^{\infty}\frac{y^k}{k!}$ folgt, dass der Interpolationsfehler ''schnell'' konvergiert, sofern sich die Ableitungen $||f^{(k)||_{L^\infty(a,b)}}$ gut verhalten (z.B. $||f^{(k)||_{L^\infty(a,b)}} \leq M < \infty$).
\end{remark}

\begin{remark}
	Für $m=1, a = x$ und $b = x+h$ folgt
	\begin{align*}
		|f'(x)-p'(x)| \leq ||f'-p'||_{L^\infty(x,x+h)} \leq C_{\mathbb{K}} \frac{||f^{(n+1)}||_{L^\infty(x,x+h)}}{n!} h^n
	\end{align*}
	d.h. besser als die Differenzquotienten aus Kapitel 1.
\end{remark}

\subsection{Cebysev-Knoten}

\begin{definition}
	Sei $f:\mathbb{R} \rightarrow \mathbb{R}$ eine Funktion und $x \in \mathbb{R}$. Man nennt x eine \textbf{n-fache Nullstelle von $f$}, gdw. $f(x)=0$ und $f$ is lokal um $x$ $(n-1)$-mal diffbar mit $f^{(k)}(x)=0 \forall k=1, ..., n-1$. Wir schreiben $n(f,x) \in \mathbb{N}_0$ für die Vielfachheit.
\end{definition}

\begin{lemma}
	Sei $p \in \mathbb{P}_n$ mit Nullstellen $x_1 < ... < x_k$ und $N := \sum_{j=1}^{n}n(f,x_j) > n$.
	
	$\implies p=0$, d.h. ein nicht-triviales Polynom vom Grad $n$ hat $\leq n$ Nullstellen, wobei diese mit Vielfachheit gezählt werden.
\end{lemma}

\begin{proof}
	Induktion nach $n$.
	
	Ind.anf.: $n=0$, d.h. $p$ ist konstant mit mind. einer Nullstelle $\implies p=0 \checkmark$
	
	Ind.hyp: Die Aussage gelte für alle Polynome $q \in \mathbb{P}_{n-1}$.
	
	$p \in \mathbb{P}_n$ hat Nullstellen $x_1 < ... < x_k$ und $N = \sum_{j=1}^{k}n(p, x_j) > n$
	
	$\implies p \in \mathbb{P}_{n-1}$ hat Nullstellen $\zeta_1 < ... < \zeta_{k-1}$ mit $x_j < \zeta_j < x_{j+1}$ (nach MWS) und bei allen $x_j$ mit $n(p,x_j)>1$.
	
	Für die Nullstellen von $p'\in \mathbb{P}_{n-1}$ gilt also
	\begin{align*}
		\sum_{j=1}^{k-1} \underbrace{n(p',\zeta_j)}_{\geq 1} + \sum_{j=1}^{k} \max\{n(p,x_j)-1, 0\}
		\geq -1 + \underbrace{\sum_{j=1}^{k} \underbrace{(\max\{n(p,x_j) - 1, 0\} + 1)}_{\geq n(p,x_j)} }_{=N > n} > n-1
	\end{align*}
	$\implies p' = 0 \implies p$ konstant $\implies p=0$.
\end{proof}

\begin{remark}
	Aus der linearen Algebra wissen wir, dass sich jedes Polynom $p \in \mathbb{P}_n$ mit Nullstelle $x_0$ in der Form $p(x) = q(x)(x-x_0)$ schreiben lässt mit $q \in \mathbb{P}_{n-1}$, sog. \textbf{Polynomdivision}.
\end{remark}

Ferner gilt der \textbf{Fundamentalsatz der Algebra}: Für jedes $p \in \mathbb{P}_n$ existieren $x_1, ..., x_n \in \mathbb{C}$ und $\lambda \in \mathbb{C}$ mit $p(x) = \lambda \prod_{j=1}^{n}(x-x_j)$. Offensichtlich ist diese Aussage viel stärker als ''mein Lemma''.

\textbf{Ziel}: Für $m=0$ gilt für alle $x \in [a,b]$
\begin{align*}
	|f(x)-p(x)| \leq C_{\mathbb{K}} \frac{||f^{(n+1)}||_{L^\infty(a,b)}}{(n+1)!} \prod_{l=0}^{n} |x-x_l|,
\end{align*}
wenn $f$ glatt und $p \in \mathbb{P}_n$ Lagrange-Interpolationspolynom zu $x_j$.

Nun wollen wir die $x_j$ so wählen, dass $\max_{x\in [a,b]} \prod_{l=0}^{n} |x-x_l|$ minimal wird.

\begin{definition}
	Für $n \in \mathbb{N}_0$ definiere die \textbf{Cebysev-Polynome (der ersten Art)} durch $T_n(t) := \cos(n \arccos t)$ auf $[-1, 1]$.
\end{definition}

\begin{lemma}
	\begin{enumerate}
		\item $T_n(\cos(\Phi)) = \cos(n \Phi) \forall 0 \leq \Phi \leq \pi \forall n \in \mathbb{N}_0$
		\item Auf $[-1, 1]$ gilt $T_0(t) = 1$, $T_1(t) = t$, $T_{n+1}(t) = 2t T_n(t) - T_{n-1}(t) \forall n \in \mathbb{N}$
		\item $T_n \in \mathbb{P}_n[-1, 1]$ mit Leitkoeffizient $2^{n-1}$ für $n \geq 1$
		\item $||T_n||_{L^\infty(-1, 1)} = 1$
		\item $T_n$ hat in $[-1, 1]$ genau $n+1$ lokale Extrema $T_n(s_j^{(n)}) = (-1)^j$ mit $s_j^{(n)} = \cos \left(\frac{j\pi}{n}\right)$ für $j=0, ..., n$
		\item $T_n$ hat in $[-1, 1]$ genau $n$ einfache Nullstellen $T_n(t_j^{(n)}) = 0$, $t_j^{(n)} = \cos \left(\frac{(2j-1)}{n}\frac{\pi}{2} \right)$ für $j=1, ..., n$
	\end{enumerate}
\end{lemma}

\begin{proof}[nur die sog. Drei-Term-Rekursion in (2)]
	Whl: Additionstheorem des Cosinus: $\cos(x)+\cos(y) = 2\cos\left(\frac{x+y}{2}\right) \cos\left(\frac{x-y}{2}\right)$
	
	$t=\cos(\Theta), x:=(n+1)\Theta, y:=(n-1)\Theta$
	\begin{align*}
		\implies \frac{x+y}{2} = n \Theta, \frac{x-y}{2} = \Theta\\
		\implies T_{n+1}(t) + T_{n-1}(t) = \cos(\underbrace{(n+1)\Theta}_{x}) + \cos(\underbrace{(n-1)\Theta}_{y}) = 2 \underbrace{\cos\left(\underbrace{\frac{x+y}{2}}_{n\Theta}\right)}_{T_n(t)} \underbrace{\cos\left(\underbrace{\frac{x-y}{2}}_{\Theta}\right)}_{=t}
	\end{align*}
\end{proof}

\begin{theorem}[Optimalität der Cebysev-Knoten]
	Betrachte die affine Transformation $\Psi:[-1, 1] \rightarrow [a,b], \Psi(t)=\frac{1}{2} \{(a+b) + t(b-a)\}$.
	
	Seien $t_1^{(n+1)}, ..., t_{n+1}^{(n+1)}$ die Nullstellen von $T_{n+1}$.
	\begin{align*}
		\implies \min_{x_0, ..., x_n \in [a,b]} \max_{x\in [a,b]} \prod_{j=0}^{n} |x-x_j| = \max_{x\in [a,b]} \prod_{j=0}^{n} |x - \Psi(t_{j+1}^{(n+1)})| = \left(\frac{b-a}{2}\right)^{n+1} \frac{1}{2^n}.
	\end{align*}
	Die $\Psi(t_{j+1}^{(n+1)})$ für $j=0, ..., n$ heißen \textbf{Cebysev-Knoten in [a,b]}.
\end{theorem}

\begin{proof}
	\begin{enumerate}
		\item zz: $\max_{t \in [-1, 1]} \prod_{j=0}^{n} |t-t_{j+1}^{(n+1)}| = \frac{1}{2^n}$
		
		Lemma (iii) + (vi) $\implies T_{n+1}(t) = 2^n \prod_{j=0}^{n} (t-t_{j+1}^{(n+1)})$
		
		Lemma (iv) $\implies 1 = ||T_{n+1}||_{L^\infty(-1, 1)} = \max_{t \in [-1, 1]} 2^n \prod_{j=0}^{n} |t-t_{j+1}^{(n+1)}|$
		
		\item zz. $\frac{1}{2^n} \leq \inf_{t_0, ..., t_n \in [-1, 1]} \max_{t \in [-1, 1]} \prod_{j=0}^{n} |t-t_j|$ (dann folgt die Behauptung für $[a, b] = [-1, 1]$)
		
		Annahme: Ex. $t_0, ..., t_n \in [-1, 1]$ mit $w(t) := \prod_{j=0}^{n} (t-t_i)$ erfüllt
		\begin{align*}
			||w||_{L^\infty(-1, 1)} = \max_{t \in [-1, 1]} \prod_{j=0}^{n} |t-t_j| \lneq \frac{1}{2^n}.
		\end{align*}
		
		Definiere $p := \underbrace{\frac{1}{2^n} T_{n+1}}_{\in \mathbb{P}_{n+1}} - \underbrace{w}_{\in \mathbb{P}_{n+1}} \in \mathbb{P}_n$. Ferner $\frac{1}{2^n}T_{n+1}(s_j^{(n+1)}) = \frac{(-1)^j}{2^n}$ und $|w(s_j^{n+1})| < \frac{1}{2^n}$.
		
		$\implies p$ hat $n+1$ Vorzeichenwechsel $\implies n+1$ Nullstellen $\implies p=0 \implies w = \frac{1}{2^n} T_{n+1}$ \lightning
		
		\item klar: $\Psi$ ist Bijektion von $[-1, 1]$ auf $[a, b]$
		
		$\Psi(t) - \Psi(t_{j+1}^{(n+1)}) = \frac{1}{2} \{(t - t_{j+1}^{(n+1)})(b-a)\}$
		
		\begin{align*}
			\implies \max_{x\in [a,b]} \prod_{j=0}^{n} |x-\Psi(t_{j+1}^{(n+1)})| = \max_{t \in [-1, 1]} \prod_{j=0}^{n} |\Psi(t) - \Psi(t_{j+1}^{(n+1)})| =\\
			\left(\frac{b-a}{2}\right)^{n+1} \underbrace{\max_{t \in [-1, 1]} \prod_{j=0}^{n} |t-t_{j+1}	{(n+1)}|}_{=\frac{1}{2^n}}
		\end{align*}
	\end{enumerate}
\end{proof}

\subsection{Lebesgue-Konstante}

\begin{theorem}
	Seien $a \leq x_0 < ... < x_n \leq b$ Stützstellen mit zugehörigen Lagrange-Polynomen $L_0, ..., L_n \in \mathbb{P}_n$.
	
	Def. $I_n : \mathcal{C}[a,b] \rightarrow \mathbb{P}_n, I_nf := \sum_{j=0}^{n} f(x_j) L_j$.
	
	$\implies I_n$ ist eine lineare Projektion auf $\mathbb{P}_n$ mit Operatornorm
	\begin{align*}
		||I_n|| := \sup_{f \in \mathcal{C}[a, b], f\neq 0} \frac{||I_nf||_{L^\infty(a,b)}}{||f||_{L^\infty(a,b)}} = \max_{x\in [a,b]} \sum_{j=0}^{n} |L_j(x)| =: \Lambda(x_0, ..., x_n).
	\end{align*}
	
	Die Zahl $\Lambda(x_0, ..., x_n)$ heißt \textbf{Lebesgue-Konstante}.
\end{theorem}

\begin{proof}
	$I_n$ wohldef., linear \checkmark
	
	Für $p \in \mathbb{P}_n$ gilt $I_np = p$, da Polynominterpolation eine eindeutige Lsg. hat.
	
	Für $f \in \mathcal{C}[a,b]$ mit $f\neq 0$ gilt
	\begin{align*}
		\frac{||I_nf||_{L^\infty(a,b)}}{||f||_{L^\infty(a,b)}} = \max_{x\in [a,b]} \left|\sum_{j=0}^{n} \frac{f(x_j)}{||f||_{L^\infty(a,b)}} L_j(x)\right| \leq \Lambda(x_0, ..., x_n).
	\end{align*}
	
	Um Gleichheit zu zeigen, wähle $x \in [a,b]$ mit $\sum_{j=0}^{n}|L_j(x)| = \Lambda(x_0, ..., x_n)$. Wähle $f \in \mathcal{C}[a, b]$ als Polygonzug mit $||f||_{L^\infty(a,b)} \leq 1$ und $f(x_j) = sign L_j(x)$. $\implies$ Gleichheit bei obiger Abschätzung.
\end{proof}

\begin{remark}
	Derselbe Beweis zeigt für $\tilde{I}_n: \mathbb{K}^{n+1} \rightarrow \mathbb{P}_k, \tilde{I}_n(y_0, ..., y_n) := \sum_{j=0}^{n} y_j L_j$, dass $||\tilde{I}_n(y_0, ..., y_n)||_{L^\infty(a,b)} \leq \Lambda \max_{j=0,...,n} |y_j|$ mit Gleichheit für spezielle $y_j = sign(L_j(x))$, wenn $x \in [a, b]$ mit $\sum_{j=0}^{n} |L_j(x)| = \max_{\tilde{x}\in [a,b]} \sum_{j=0}^{n} |L_j(x)|$.
	
	$\implies \tilde{I}$ ist der (lineare) Lösungsoperator der Pol.int.
	\begin{align*}
		\implies ||\tilde{I}_n(y_0, ..., y_n) - I_n(\tilde{y}_0, ..., \tilde{y}_n)||_{L^\infty(a,b)} \leq \Lambda \max_{j=0,...,n} |y_j - \tilde{y}_i|
	\end{align*}
	d.h. $\Lambda$ ist die abstrakte Konditionszahl der Polynominterpolation.
\end{remark}

Abschließend einige Bemerkungen zum Bestapproximationsproblem.

\begin{lemma}
	$X$ normierter Raum, $Y \leq X$ endlich-dim. Teilraum, $x\in X$
	
	$\implies$ Ex. $y \in Y$ mit $||x-y||_X = \min_{\tilde{y} \in Y} ||x-\tilde{y}||_X$
\end{lemma}

\begin{proof}
	Wähle Folge $(y_n)_{n\in \mathbb{N}} \subseteq Y$ mit
	\begin{align*}
		\lim\limits_{n\rightarrow\infty} ||x-y_n||_X = \inf_{\tilde{y} \in Y} ||x-\tilde{y}||_X\\
		\implies ||y_n||_X \leq \underbrace{||x-y_n||_X}_{\text{glm. beschränkt wegen Konvergenz}} + ||x||_X \implies \sup_{n\in \mathbb{N}} ||y_n||_X \leq M < \infty
	\end{align*}
	
	Da $Y$ endl.-dim., gilt der Satz von Bolzano-Weierstraß, d.h. $(y_n)_{n\in \mathbb{N}}$ hat eine konvergente Teilfolge. O.B.d.A. ex. $y \in Y$ mit $||y-y_n||_X \rightarrow 0$ für $n \rightarrow \infty$.
	
	$||x-y||_X = \lim\limits_{n\rightarrow\infty} ||x-y_n||_X = \inf_{\tilde{y} \in Y} ||x-\tilde{y}||_X$.
\end{proof}

\begin{remark}
	Mit Satz über Lebesgue-Konstante und dem Lemma gilt für alle $q \in \mathbb{P}_n$
	\begin{align*}
		\underbrace{||f-I_nf||_{L^\infty(a,b)}}_{\geq \min_{q \in \mathbb{P}_n} ||f-q||_{L^\infty(a,b)}} \leq ||f-q||_{L^\infty(a,b)} + \underbrace{||I_n(f-q)||_{L^\infty(a,b)}}_{\leq \Lambda ||f-q||_{L^\infty(a,b)}}\\
		\implies \min_{q \in \mathbb{P}_n} ||f-q||_{L^\infty(a,b)} \leq ||f-I_nf||_{L^\infty(a,b)} \leq (1+\Lambda) \min_{q \in \mathbb{P}_n} ||f-q||_{L^\infty(a,b)}
	\end{align*}
\end{remark}

\begin{remark}
	Nach Satz von Weierstraß gilt $\lim\limits_{n\rightarrow\infty} \min_{p \in \mathbb{P}_n} ||f-p||_{L^\infty(a,b)} = 0 \forall f \in \mathcal{C}[a,b]$.
	
	Nach Satz von Faber gilt allerdings, dass es für jede Folge von Stützstellen $(x_0^{(n)}, ..., x_n^{(n)})_{n\in\mathbb{N}}$ eine Funktion $f\in \mathcal{C}[a,b]$ mit der Eigenschaft, dass $||f-I_n^{(n)}f||_{L^\infty(a,b)}$ divergiert!
	
	Insbesondere muss also $\Lambda_n^{(n)} \rightarrow \infty$ gelten!
\end{remark}

\begin{remark}
	Für äquidistante Stützstellen divergiert $\Lambda_n$ exponentiell schnell. Für Cebysev-Knoten gilt allerdings $\Lambda_n = \mathcal{O}(\log n)$.
\end{remark}

\begin{remark}
	Der \textbf{Remez-Algorithmus} berechnet (in unendlich vielen Schritten) ein Polynom $q \in \mathbb{P}_n$ mit $||f-q||_{L^\infty(a,b)} = \min_{p \in \mathbb{P}_n} ||f-p||_{L^\infty(a,b)} \forall f\in\mathcal{C}[a,b]$.
	
	Startwert ist dafür der Cebysev-Interpoland.
	
	Der \textbf{Alternantensatz von Cebysev} zeigt, dass das Bestapprox.polynom $q \in \mathbb{P}_n$ bzgl. $||.||_{L^\infty(a,b)}$ in der Tat eindeutig ist.
\end{remark}

\subsection{Auswertung von Interpolationspol.}

\begin{theorem}[Neville-Verfahren]
	Seien $a \leq x_0 < ... < x_n \leq b$ Stützstellen mit Funktionswerten $y_j \in \mathbb{K}$ und $p \in \mathbb{P}_n$ mit $p(x_j) = y_j \forall j=0, ..., n, x \in [a, b]$ Auswertungspunkt.
	
	Für $j,m \in \mathbb{N}_0$ mit $j + m \leq n$, definiere $p_{j,m} \in \mathbb{P}_m$ als eind. Int.polynom mit $p(x_k) = y_k \forall k=j, ..., j+m$
	\begin{align*}
		p(x) = p_{0, n}(x)\\
		p_{j,0}(x) = y_j\\
		p_{j,m}(x) = \underbrace{\frac{(x-x_j) p_{j+1, m-1}(x) - (x-x_{j+m}) p_{j,m-1}}{x_{j+m}-x_j}}_{=:q(x), q \in \mathbb{P}_m}
	\end{align*}
\end{theorem}

\begin{proof}
	$q(x) = y_j, q(x_{j+m}) = y_{j+m}, q(x_k) = y_k, k=j+1, ..., n-m+1 \implies q=p_{j,m}$
\end{proof}

Dieser Satz führt auf das induktive \textbf{Neville-Schema}
\begin{align*}
	\begin{matrix}
		y_0 &= & p_{0,0}(x) & \rightarrow & p_{0,1}(x) & \rightarrow & ... & p_{0,n}(x) = p(x)\\
		                 &  & & \nearrow    &            & & & \\
		y_1 &= & p_{1,0}(x) & \rightarrow & p_{1,1}(x) & \nearrow    & & \\
		y_2 &= & p_{2,0}(x) & \nearrow & & & \\
		&\vdots & & & & & \\
		y_{n-1} &= & p_{n-1,0}(x) & \rightarrow & p_{n-1,1}(x) & & \\
		y_n &= & p_{n,0}(x) & \nearrow & & & \\
	\end{matrix}
\end{align*}

\begin{remark}
	\begin{itemize}
		\item Das Neville-Verfahren ist ein sog. \textbf{Einschritt-Verfahren}, d.h. eine ''neue Spalte'' nur mit Hilfe der vorausgegangenen Spalte berechnet.
		\item Wenn man ''von oben nach unten rechnet'', ist kein zusätzlicher Speicher nötig. In diesem Fall sollte man die ''Diagonale'' speichern.
		\item Man kann im Neville-Verfahren dann leicht einen neuen Punkt $(x_{n+1}, y_{n+1})$ hinzunehmen und erhält $p_{0,n+1}(x)$, indem man nur die neue Diagonale rechnet.
	\end{itemize}
\end{remark}

\begin{algorithm}[Neville]
	Input: Stützstellen $a \leq x_0 < ... < x_n \leq b$, Funktionswerte $y_0, ..., y_n \in \mathbb{K}$, Auswertungspunkt $x \in \mathbb{R}$
	
	\begin{itemize}
		\item for $m=1:n$
		\item for $j=0:n-m$
		\item $y_j = \frac{(x-x_j)y_{j+1} - (x-x_{j+m})y_j}{x_{j+m}-x_j}$
		\item end
		\item end
	\end{itemize}
	
	Output: $y_0 = p(x)$, wobei $p \in \mathbb{P}_n$ mit $p(x_j) = y_j \forall j$
	
	klar: Speicherbedarf $n+1$ (überschreiben von $y$-Vektor), Arithmetischer Aufwand $\frac{7}{2}n(n+1)$.
\end{algorithm}

\begin{definition}
	Sei $p = \sum_{j=0}^{n} \lambda_j x^j \in \mathbb{P}_n$. Dann bezeichnet man $\lambda_n$ als \textbf{führenden Koeffizienten von $p$ bzgl. $\mathbb{P}_n$}.
	
	Falls $j=0$ oder ($\lambda_j \neq 0$ und $\lambda_k = 0 \forall k > j$), so bezeichnet man $\lambda_j$ als \textbf{Leitkoeffizient von $p$}.
\end{definition}

\begin{theorem}[Newtons Dividierte Differenzen]
	Seien $a \leq x_0 < ... < x_n \leq b$ Stützstellen, $y_j \in \mathbb{K}, p \in \mathbb{P}_n$ mit $p(x_j) = y_j \forall j=0, ..., n$. Für $j, m \in \mathbb{N}_0$ mit $j+m \leq n$ definiere
	\begin{align*}
		y_{j,0} := y_j & y_{j,m} := \frac{y_{j+1,m-1}-y_{j,m-1}}{x_{j+m}-x_j}
	\end{align*}
	$\implies$
	\begin{enumerate}
		\item $y_{j,m}$ ist der führende Koeff. von $p_{j,m} \in \mathbb{P}_m$ aus dem Neville-Verfahren.
		\item Mit $\lambda_j := y_{0,j}$ gilt $p(x) = \sum_{j=0}^{n} \lambda_j \underbrace{\prod_{k=0}^{j-1}(x-x_k)}_{=q_j \in \mathbb{P}_j}$ d.h. die dividierten Differenzen geben die Koeffizienten des Int.pol. bzgl. Newton-Basis.
	\end{enumerate}	
\end{theorem}

\begin{proof}
	$q_k := p_{0,k} - p_{0,k-1} \in \mathbb{P}_n$ mit führendem Koeff. $y_{0,k}$ und Nullstellen $x_0, ..., x_{n-1}$
	
	$\implies q_k = y_{0,k} \prod_{j=0}^{k-1}(x-x_j)$ nach Pol.div.
	\begin{align*}
		\implies p = p_{0,k} = p_{0,0} + \sum_{k=1}^{n} \underbrace{(p_{j,k} - p_{0,k-1})}_{=q_k} = y_{0,0} + \sum_{k=1}^{n} y_{0,k} \prod_{j=0}^{k-1} (x-x_j) = \sum_{k=0}^{n} \underbrace{y_{0,k}}_{=\lambda_k} \prod_{j=0}^{k-1} (x-x_j)
	\end{align*}
\end{proof}

Schema der dividierten Differenzen
\begin{align*}
	\begin{matrix}
		y_0 &=      & y_{0,0} & \searrow & & & \\
		y_1 &=      & y_{1,0} & \rightarrow & y_{0,1} & & \\
		    &\vdots &         & \searrow & & & \\
		    &\vdots &         & \rightarrow & y_{1,1} & \rightarrow & y_{0,2} \\
		    &\vdots &         &            & &  &\ddots\\
		y_{n-1} &=  & y_{n-1,0} & \searrow & & & \\
		y_n &=      & y_{n,0} & \rightarrow & y_{n-1,1} & \rightarrow & ... & y_{0,n}\\
	\end{matrix}
\end{align*}

$\implies$ arithmetischer Aufwand $3 \frac{n(n+1)}{2}$, um alle $y_{0,j}$ zu berechnen.

\begin{remark}
	\begin{itemize}
		\item Die dividierten Differenzen sind ein Einschrittverfahren.
		\item Wenn man den $y$-Vektor überschreibt, braucht man keinen zusätzlichen Speicher.
		\item Das Verfahren löst das Vandermonde-System für die Newton-Basis, aber die Matrix aufzustellen.
		\item Die Auswertung von $p(x)$ erfolgt mit Horner-Schema und Aufwand $3n$ pro $x \in \mathbb{K}$.
	\end{itemize}
\end{remark}

\begin{algorithm}
	Input: Stützstellen $x_0 < ... < x_n$, Funktionswerte $y_0, ..., y_n \in \mathbb{K}$
	
	\begin{itemize}
		\item for $m=1:n$
		\item for $j = n-m:-1:0$
		\item $y_{j,m} := \frac{y_{j+m} - y_{j,m-1}}{x_{j+m} - x_j}$
		\item end
		\item end
	\end{itemize}
	
	Output: Koeffizienten des Interpol.pl. $p \in \mathbb{P}_n$ $y_0, ..., y_n$ bzgl. Newton-Basis.
\end{algorithm}

\begin{remark}
	Will man das Interpolationspolynom $p(x)$ an $N$ Stellen auswerten, so gilt für den Gesamtaufwand: Aufwand(Neville) = $\frac{7}{2}Nn(n+1)$, Aufwand(Div. Diff. + Horner) = $\underbrace{\frac{3}{2}n(n+1)}_{\text{div. Diff.}} + \underbrace{3Nn}_{\text{Horner}}$.
	
	Es gilt immer: Aufwand(Div. Diff. + Horner) $\leq$ Aufwand(Neville). Wenn man sich den Fortpflanzungsfehler anschaut dann sieht man aber, dass Neville weniger anfällig ist für Auslöschung.
	
	In der Praxis verwendet man deshalb Neville für kleine $N$ und Div. Diff. + Horner für große $N$.
\end{remark}

\subsection{Hermite-Polynominterpolation}

\begin{theorem}[Wohlgestelltheit]
	Gegeben seien Stützstellen $a \leq x_0 < ... < x_n \leq b$, Funktionswerte $y_j^(k) \in \mathbb{K}$ für $j=0, ..., n$ und $k=0, ..., n_j \in \mathbb{N}_0$ (Lagrange $n_j = 0 \forall j$), Def $N := \left(\sum_{j=0}^{n}(n_j + 1)\right)-1$
	
	$\implies$ Ex. eind. $p \in \mathbb{P}_N$ mit $p^{(k)}(x_j) = y_j^{(k)} \forall j=0, ..., n, \forall k=0, ..., n_j$, wobei $p^{(0)} = p$.
\end{theorem}

\begin{proof}
	Betrachte den Auswertungsoperator $\mathcal{A}: \mathbb{P}_N \rightarrow \mathbb{K}^{N+1}, \mathcal{A}p := (p(x_0), ..., p^{(n_0)}(x_0), p(x_1), ..., p^{(n_1)}(x_1), ..., p^{(n_n)}(x_n))$
	
	klar: $\mathcal{A}$ ist linear und $\dim \mathbb{P}_N = N+1$
	
	$\implies \mathcal{A}$ ist $\underbrace{\text{bijektiv}}_{\text{=Behauptung}}$, gdw. $\mathcal{A} \underbrace{\text{injektiv}}_{\text{zu zeigen!}}$ $\underbrace{\text{(oder $\mathcal{A}$ ist surj)}}_{\text{''schwierig''}}$.
	
	Sei $p \in \mathbb{P}_N$ mit $\mathcal{A}p = 0$, d.h. $x_j$ eine $(n_j+1)$-fache Nullstelle von $p \forall j$. $\implies p \in \mathbb{P}_N$ hat $\sum_{j=0}^{n}(n_j+1) = N+1$ viele Nst. (bzgl. Vielfachheit) $\implies p = 0$. 
\end{proof}

\begin{remark}
	\begin{itemize}
		\item Der vorausgegangene Beweis ist das ''normale Beweisprinzip'' für lineare Interpolationsaufgaben. Klar: Man kann die Interpolationsaufgabe insb. lineares Gleichungssystem (äquivalent) formulieren.
		\item Neville-Verfahren und dividierte Differenzen lassen sich auch für das Hermite-Interpolationsproblem formulieren.
		\item Analog zu Lagrange (dieselbe Basis) kann man Fehlerdarstellung und Feherabschätzung beweisen, z.B.
		\begin{align*}
			|f(x)-p(x)| \leq C_{\mathbb{K}} \frac{||f^{(N+1)}||_{L^\infty(a,b)}}{(N+1)!} \prod_{j=0}^{n} |x-x_j|^{n_j+1}
		\end{align*}
		$C_{\mathbb{C} = \sqrt{2}}$ (vorher $2$)
	\end{itemize}
\end{remark}

\subsection{Spline-Interpolation}

Die Polynominterpolation erfordert hohe Glätte an $f$, um Fehlerabschätzung zu kriegen. Alternativ kann man deshalb stückweise Polynome betrachten (sog. Splines), um Verfahren und Fehlerkontrolle zu haben, falls $f$ nicht so glatt ist.

\begin{example}[affiner Interpolationspline]
	Zu Stützstellen $a = x_0 < x_1 < ... < x_n = b$ und $f \in \mathcal{C}[a,b]$ ist $s \in \mathcal{C}[a,b]$ mit
	\begin{itemize}
		\item $s|_{[x_{j-1}, x_j]} \in \mathbb{P}_1 \forall j=1, ..., n$
		\item $s(x_j) = f(x_j) \forall j=0, ..., n$
	\end{itemize}
	
	$\implies$ Offensichtlich eindeutig $s(x) = f(x_{j-1}) \frac{x-x_j}{x_{j-1}-x_j} + f(x_j) \frac{x-x_{j-1}}{x_j-x_{j-1}} \forall j \forall x \in [x_j-1, x_j]$
\end{example}

\begin{lemma}
	Zu $f \in \mathcal{C}[a, b] \cap \mathcal{C}^2[x_{j-1}, x_j] \forall j=1, ..., n$ sei $s \in \mathcal{C}[a, b]$ der affine Interpolationsspline.
	
	Def $h:[a,b] \rightarrow \mathbb{R}_{>0}, h|_{[x_{j-1}, x_j]} := x_j - x_{j-1}$ \textbf{lokale Netzweite}
	
	$\implies ||f-s||_{L^\infty(a,b)} \leq \frac{C_{\mathbb{K}}}{8} ||h^2 f''||_{L^\infty(a,b)}$
\end{lemma}

\begin{proof}
	Sei $x \in [x_{j-1}, x_j]$
	\begin{align*}
		\implies |f(x) - s(x)| \leq C_{\mathbb{K}} \frac{||f''||_{L^\infty(x_{j-1}, x_j)}}{2} \underbrace{|(x-x_{j-1})(x-x_j)}_{\text{maximal für }x = \frac{x_{j-1} + x_j}{2}}\\
		\implies ||f-s||_{L^\infty(x_{j-1}, x_j)} \leq C_{\mathbb{K}} \frac{||f''||_{L^\infty(x_{j-1}, x_j)}}{2} \cdot \frac{(x_j - x_{j-1})^2}{9} = \frac{C_{\mathbb{K}}}{8} ||h^2 f''||_{L^\infty(x_{j-1}, x_j)}
	\end{align*}
\end{proof}

\begin{lemma}
	Zu $f \in \mathcal{C}[a, b]$ und $s \in \mathcal{C}[a, b]$ affiner Int.spline
	\begin{align*}
		\implies (i) ||f-s||_{L^2(a,b)} \leq ||h f'||_{L^2(a,b)}, \text{ sofern } f \in \mathcal{C}^1[x_{j-1}, x_j] \text{ für alle } j\\
		(ii) ||f-s||_{L^2(a,b)} \leq ||h^2f||_{L^2(a,b)}, \text{ sofern } f \in \mathcal{C}^2[x_{j-1}, x_j] \text{ für alle } j
	\end{align*}
\end{lemma}

\begin{proof}
	zz: (i) elementweise für $I_j = [x_{j-1}, x_j]$
	\begin{align*}
		F := f-s \in \mathcal{C}^1[x_{j-1}, x_j], h_j := x_j - x_{j-1}\\
		F(x_{j-1}) = 0 \implies \int_{I_j} |F(x)|^2 = \int_{I_j} \left|\int_{x_{j-1}}^{x_j} F' dx \right|^2 \leq h_j^2 ||F'||_{L^2(I_j)}^2\\
		\implies ||F||_{L^2(I_j)} \leq h_j ||F'||_{L^2(I_j)}\\
		F(x_{j-1}) = 0 = F(x_j) \implies \int_{I_j}F' dx = 0
	\end{align*}
	$s'|_{I_j}$ konstant
	\begin{align*}
		\implies s'|_{I_j} = \frac{1}{h_j} \int_{I_j}f' dx \implies <f', s'>_{L^2(I_j)} = ||s'||_{L^2(I_j)}^2\\
		\implies ||F'||_{L^2(I_j)}^2 = ||f'||_{L^2(I_j)}^2 - 2 Re<f', s'>_{L^2(I_j)} + ||s'||_{L^2(I_j)}^2 = ||f'||_{L^2(I_j)}^2 - ||s'||_{L^2(I_j)}^2\\
		\implies ||f-s||_{L^2(I_j)} = ||F||_{L^2(I_j)} \leq h_j ||F'||_{L^2(I_j)} \leq h_j ||f'||_{L^2(I_j)} = ||hf'||_{L^2(I_j)}
	\end{align*}
	
	zz: (ii) elementweise
	\begin{align*}
		Re F(x_{j-1}) = 0 = ReF(x_j) \implies Re F'(\zeta) = 0 \text{ für } x_{j-1} < \zeta < x_j\\
		|Re F'(x)| = | \int_{J} Re F'' dt| \leq h_j^{\frac{1}{2}} ||Re F''||_{L^2(I_j)}
	\end{align*}
	analog für $Im F'$
	\begin{align*}
		\implies \int_{I_j}|F'(x)|^2 \leq h_j^2 ||F''||_{L^2(I_j)}^2 \implies ||F'||_{L^2(I_j)} \leq h_j ||F''||_{L^2(I_j)}\\
		\implies ||f-s||_{L^2(I_j)} \leq h_j ||F'||_{L^2(I_j)} \leq h_j^2 ||\underbrace{F''}_{= f'' \text{ auf } I_j}||_{L^2(I_j)} = ||h^2 f''||_{L^2(I_j)}
	\end{align*}
	
	(3) $||f-s||_{L^2(a,b)}^2 = \sum_{j=1}^{n} ||f-s||_{L^2(I_j)}^2 \leq \sum_{j=1}^{n} ||h^2 f''||_{L^2(I_j)}^2 = ||h^2 f''||_{L^2(a,b)}^2$
\end{proof}

\begin{definition}
	Es sei $\Delta = (x_0, ..., x_n)$ eine \textbf{Zerlegung von $[a,b]$}, d.h. $a = x_0 < ... < x_n = b$. Zu gegebenen $p,q \in \mathbb{N}_0$ heißt $s:[a,b]\rightarrow\mathbb{K}$ \textbf{Spline vom Grad $p$ mit Glattheit $q$}, gdw. $s \in \mathcal{C}^q[a,b]$ mit $s|_{[x_{j-1}, x_j]} \in \mathbb{P}_p \forall j=1, ..., n$.
	
	Schreibweise $s \in \mathbb{S}_q^p(\Delta)$ bzw. $s \in \mathbb{S}^p(\Delta)$, falls $q=p-1$.
\end{definition}

\begin{remark}
	Die wichtigsten Beispiele sind \textbf{affine Splines} $\mathbb{S}^1(\Delta)$, \textbf{quadratische Splines} $\mathbb{S}^2(\Delta)$, \textbf{kubische Splines} $\mathbb{S}^3(\Delta)$.
\end{remark}

\begin{remark}
	Für Wohlgestelltheit von Interpolationsaufgaben muss man $\dim \mathbb{S}_q^p(\Delta)$ bestimmen.
\end{remark}

\begin{example}
	$\dim \mathbb{S}_1^p(\Delta) = \underbrace{n}_{\text{Anz. Intervalle}}\underbrace{(p+1)}_{=\dim \mathbb{P}_p} \widehat{=}$ global unstetige stückweise Polynome
	
	$\dim \mathbb{S}_0^p(\Delta) = n(p+1) - (n-1) \widehat{=}$ global stetige stw. Polynome
\end{example}

$\implies$ Bestimmung von Basen (und Dimension) von $\mathbb{S}_q^p(\Delta)$ ist komplizierter als bei Polynomräumen, lässt sich aber über sogn. B-Splines bewerkstelligen.

\begin{remark}
	Bei Splines hat man mehrere Möglichkeiten, um den Fehler $||f-s||$ zu verringern
	\begin{itemize}
		\item \textbf{h-Methode}, d.h. die Zerlegung wird verfeinert
		\item \textbf{p-Methode}, d.h. man erhöht den Polynomgrad
		\item \textbf{r-Methode}, d.h. man erhält $\#\Delta=n$, aber man verschiebt die Stützstellen.
	\end{itemize}
	Zusätzlich kann man alles lokal kombinieren, z.B.
	\begin{itemize}
		\item \textbf{hp-Methode}: Verfeinere $\Delta$, wo $f$ unglatt ist, und erhöhe $p$, wo $f$ glatt ist.
	\end{itemize}
\end{remark}

\begin{theorem}
	Sei $\Delta = (x_0, ..., x_n)$ Zerlegung von $[a,b]$
	
	$\implies \dim \mathbb{S}^p(\Delta) = n+p$ und $\mathcal{B}:=\{x^j, \max\{x-x_k, 0\}^p | j=0, ..., p, k=1, ..., n-1\}$ ist eine Basis.
\end{theorem}

\begin{proof}
	klar: $\mathcal{B} \subseteq \mathbb{S}^p(\Delta)$
	
	(1) zz. $\mathcal{B}$ ist lin. unabh.
	
	Seien $\lambda_j, \mu_j \in \mathbb{K}$ mit $0 = \sum_{j=0}^{p} \lambda_j x^j + \sum_{k=1}^{n-1} \mu_k q_k$
	
	klar: Auf $[a, b]$ gilt $q_k|_{[x_0, x_1]} = 0 \forall k=1, ..., n-1$ $\implies \lambda_j = 0 \forall j=0, ..., p$, da Monome lin. unabh. auf $[x_0, x_1]$.
	
	klar: Auf $[x_1, x_2]$ gilt $q_k|_{[x_1, x_2]} = 0 \forall k=2, ..., n-1$ $\implies \mu_1 q_1 = 0$ auf $[x_1, x_2]$, $q_1(x_2) \neq 0 \implies \mu_1 = 0$
	
	Induktives Vorgehen zeigt $\mu_k = 0 \forall k=1, ..., n-1$
	
	(2) zz: $\mathbb{S}^p(\Delta) \subseteq [\mathcal{B}]$
	
	Sei $s \in \mathbb{S}^p(\Delta)$. Sei $z_1 \in \mathbb{P}_p$ mit $s|_{[x_0, x_1]} = z_1|_{[x_0, x_1]}$.
	
	Für $k=2, ..., n$ definiere sukzessive
	\begin{align*}
		z_k := z_{k-1} + \frac{s(x_k) - z_{k-1}(x_k)}{q_{k-1}(x_n)} q_{n-1} \in span(B)
	\end{align*} 
	zz. $z_n = s$
	
	Betrache Residuum $r := s - z_n$
	
	klar: $r(x_k) = 0 \forall k=0, ..., n$
	
	Auf $[x_0, x_1]$ gilt $q_k|_{[x_0, x_1]} = 0 \forall k = 1, ..., n-1$ $\implies z_n|_{[x_0, x_1]} = z_1|_{[x_0, x_1]} = s|_{[x_0, x_1]} \implies r|_{[x_0, x_1]} = 0$.
	
	Auf $[x_1, x_2]$ gilt $r \in \mathbb{P}_p$ mit $r(x_1) = 0 = r(x_2)$. $r \in \mathcal{C}^{p-1}[a,b]$, insb. $(p-1)$-mal stetig diffbar in $x_1$, d.h. $x_1$ ist $p$-fache Nst. von $r$. $\implies p+1$ viele Nst. auf $[x_1, x_2] \implies r = 0$ auf $[x_1, x_2]$.
	
	Dasselbe Vorgegen auf allen Intervallen zeigt $r=0$ auf $[a, b]$. 
\end{proof}

\begin{remark}
	Sei $\Delta = (x_0, ..., x_1)$ Zerlegung von $[a, b]$ und $s \in \mathbb{S}^p(\Delta)$ mit $s(x_j) = y_j \forall j=0, ..., n$. Dann sind nur $n+1$ Bedingungen fixiert, aber $\dim \mathbb{S}^p(\Delta) = n+p$, d.h. es fehlen noch $p-1$ Bedingungen für Wohlgestelltheit.
	
	Diese werden in der Regel als Randbedingungen formuliert, d.h. an Bedingungen an Ableitungen von $s$ in $a$ und $b$. Um diese symmetrisch zu stellen, müssen wir annehmen, dass $p-1 = 2r$ gerade ist.
	
	\begin{itemize}
		\item (H) \textbf{Hermite-Randbedingungen} (oder \textbf{vollständige Randbedingungen}): Es werden $s^(j)(a), s^(j)(b)$ für $j = 1, ..., r$ zusätzlich vorgegeben.
		
		\item (N) \textbf{Natürliche Randbedingungen}: $r \leq n$ und $s^{(j)}(a) = 0 = s^{(j)}(b)$ für $j=r+1, ..., 2r$
		
		\item (P) \textbf{Periodische Randbedingungen}: $s^{(j)}(a) = s^{(b)}$ für $j=1, ..., 2r$
	\end{itemize}
\end{remark}

\begin{theorem}
	\begin{enumerate}
		\item Zu $p-1 = 2r$ und Randbedingungen (H), (P), (N) existiert (jeweils) ein eindeutiger Interpolationsspline $s \in \mathbb{S}^p(\Delta)$, der diese Randbedingung und $s(x_j) = y_j \forall j=0, ..., n$ wobei $\Delta = (x_0, ..., x_n)$ und $y_j \in \mathbb{K}$ gegeben.
		
		\item Erfüllt $g \in \mathcal{C}^{(r+1)}[a, b]$ dieselben Interpolations- und Randbedingungen wie $s$, so gilt
		\begin{align*}
			||g^{(r+1)} - s^{(r+1)}||_{L^2(a,b)}^2 = ||g^{(r+1)}||_{L^2(a,b)}^2 - ||s^{(r+1)}||_{L^2(a,b)}^2
		\end{align*}
		d.h. der Spline erfüllt die Minimaleigenschaft
		\begin{align*}
			||s^{(r+1)}||_{L^2(a,b)} \leq ||g^{(r+1)}||_{L^2(a,b)}
		\end{align*}
		
		\item Falls $s$ (N) erfüllt, so muss $g$ nur die Interpolationsbedingungen erfüllen!
	\end{enumerate}
\end{theorem}

\begin{remark}
	Für $p=2$ minimieren kubische Interpolationssplines also die Krümmungsenergie $||s''||_{L^2(a,b)}^2$. Daher kommt auch der Name Spline (dt: Biegestab)
\end{remark}

\begin{proof}
	(1) zz. (i) unter der Veraussetzung, dass (ii) gilt.
	
	Seien $s, \tilde{s} \in \mathbb{S}^p(\Delta)$ mit derselben Interpolations- und Randbedingungen. Aus (ii) folgt dann
	\begin{align*}
		||\tilde{s}^{(r+1)} - s^{(r+1)}||_{L^2(a,b)} = 0 \implies \rho := \tilde{s} - s \in \mathbb{P}_r
	\end{align*}
	
	(H) $\rho(a) = 0$ und $\rho^{(j)}(a) = 0$ für $j=1, ..., r$ $\implies a$ ist $(r+1)$-fache Nullstelle $\implies \rho = 0$
	
	(N) $\rho(x_j) = 0$ für $j=0, ..., n$ und $r \leq n$ $\implies$ mehr als $r+1$ Nullstellen $\implies \rho = 0$
	
	(P) $\rho^{(j)}(a) = \rho^{(j)}(b)$ für $j = 1, ..., 2r$. $\rho \in \mathbb{P}_r \implies \rho^{(r-1)} \in \mathbb{P}_1$ und $\rho^{(r-1)}(a) = \rho^{(r-1)}(b) \implies \rho^{(r-1)} \in \mathbb{P}_0 \implies \rho \in \mathbb{P}_{r-1}$. Induktiv $\rho \in \mathbb{P}_0$ mit $\rho(a) = 0 \implies \rho = 0$.
	
	Jetzt folgt mit Standardargumentation, dass der lineare Auswertungsoperator $\mathcal{A}: \mathbb{S}^p(\Delta) \rightarrow \mathbb{K}^{n+p}$ bijektiv ist und aus Dimensionsgründen bijektiv. Insbesondere ist der Lösungsoperator $\mathcal{L} = \mathcal{A}^{-1}$ bijektiv.
	
	(ii) Betrachte $|x-y|^2 = |x|^2 - 2 Re x \bar{y} + |y|^2 = |x|^2 - |y|^2 - 2 Re (x-y)\bar{y}$
	
	$\implies ||g^{(r+1)} - s^{(r+1)}||_{L^2(a,b)}^2 = ||g^{(r+1)}||_{L^2(a,b)}^2 - ||s^{(r+1)}||_{L^2(a,b)}^2 - 2 Re \underbrace{\int_{a}^{b} (g^{(r+1)} - s^2{(r+1)}) \overline{s^{(r+1)}} dx}_{\text{zz: }=0}$
	
	Wh: $\int_{a}^{b}F'G + FG' dx = [FG]_a^b$
	
	\begin{align*}
		\int_{a}^{b} (g^{(r+1)} - s^{(r+1)}) \overline{s^{(r+1)}} dx = \left[(g^{(r)} - s^{(r)}) \overline{s^{(r+2)}} dx \right]_a^b - \int_{a}^{b} (g^{(r)} - s^{(r)}) \overline{s^{(r+2)}} dx =\\
		\left[(g^{(r)} - s^{(r)}) \overline{s^{(r+2)}} dx \right]_a^b - \left[(g^{(r-1)} - s^{(r-1)}) \overline{s^{(r+2)}} dx \right]_a^b + \int_{a}^{b} (g^{(r-1)} - s^{(r-1)}) \overline{s^{(r+3)}} dx =\\
		\left(\sum_{j=0}^{r-1} (-1)^j \left[(g^{(r-j)} - s^{(r-j)}) \overline{s^{(r+1+j)}}\right]_a^b \right) + (-1)^j \int_{a}^{b} (g' - s') \underbrace{\overline{s^{(2r+1)}}_{=\overline{s^(p)} \leftarrow \text{ ist konstant auf }[x_{j-1}, x_j]}} dx
	\end{align*}
	
	\begin{align*}
		\int_{a}^{b} (g' - s') \overline{s^{(p)}} dx = \sum_{j=1}^{n} \underbrace{\int_{x_{j-1}}^{x_j} (g' - s') \underbrace{\overline{s^{(p)}}}_{\in \mathbb{K}} dx}_{= \overline{s^{(p)}}_{[x_{j-1}, x_j]} \underbrace{\int_{x_{j-1}}^{x_j} (g' - s') dx}_{=[g-s]_{x_{j-1}}^{x_j} = 0} }
	\end{align*}
	
	(H) $(g^{(r-j)} - s^{(r-j)})(\frac{a}{b}) = 0 \implies \sum(.) = 0$
	
	(N) $\overline{s^{(r+1+j)}}(\frac{a}{b})=0 \implies \sum (.) = 0$
	
	(P) $[g^{(r-j)}]_a^b = 0, [s^{(r-j)}]_a^b = 0 \implies \sum(.) = 0$
\end{proof}

\begin{remark}
	Man kann mit Hilfe der sog. B-Splines Basen von allgemeineren Spline-Räumen konstruieren:
	
	Sei $(t_j)_{j \in \mathbb{Z}}$ monoton steigend mit $\lim\limits_{j \rightarrow \pm \infty} t_j = \pm \infty$.
	
	Def $B_{j, 0} := \Xi_{[t_j, t_{j+1})}$ für $p = 0$, $B_{j,p} := w_{j,p} B_{j,p-1} + (1-w_{j,p})B_{j+1,p-1}$ für $p \geq 1$ mit $w_{j,p}(x) = \frac{x-t_j}{t_{j+p} - t_j}$ für $t_j < t_{j+p}$ und $w_{j,p}(x) = 0$ für $t_j = t_{j+p}$
	
	\begin{itemize}
		\item Man kann zeigen, dass für $t_j = x_j$ mit $\Delta=(x_0, ..., x_n)$ Zerlegung von $[a,b]$ gilt $\mathbb{S}^p(\Delta) = span\{B_{j,p}|_{[a,b]}: j=p, ..., n-1\}$ und diese $B_{j,p}$ bilden auch eine Basis (da $n+p$ viele).
		
		\item Ferner haben die $B_{j,p}$ ''gute'' Eigenschaften, z.B. $B_{j,p} > 0, supp(B_{j,p}) = [t_j, t_{j+p+1}], \sum_{j \in \mathbb{Z}} B_{j,p} = 1$.
		
		\item Die Vielfachheit eines Knotens $t_j = ... = t_{j+k}$ reduiert die Differenzierbarkeit der Splines bei $t_k$ um $k$, d.h. $\mathcal{C}^{p-(k+1)}$-Differenzierbarkeit.
	\end{itemize}
\end{remark}

Übung: Sei $s \in \mathbb{S}^p(\Delta)$ interpolierend $s(x_j) = y_j$ mit natürlichen Randbedingungen ($\rightsquigarrow$ ''naives'' Gleichungssystem mit Basis aus Satz wobei $(n+3)\times(n+3)$ Gleichungssysteme mit vollbesetzter Vandermonde-Matrix). Auf $[x_{j-1}, x_j]$ machen wir den Ansatz $s|_{[x_{j-1, x_j}]} = a_0^{(j)} + a_1^{(j)} (x-x_j) + a_2^{(j)} (x-x_j)^2 + a_3^{(j)}(x-x_j)^3$ mit unbekannten Koeffizienten $a_0^{(j)}, a_1^{(j)}, a_2^{(j)}, a_3^{(j)} \forall j=1, ..., n$.
\begin{align*}
	\implies a_0^{(j)} = y_j, a_1^{(j)} = \frac{y_j - y_{j-1}}{h_j} + \frac{h_j}{3} (2 a_2^{(j)} + a_2^{(j-1)}), a_3^{(j)} = \frac{a_2^{(j)} - a_2^{(j-1)}}{3h_j}, h_j := x_j - x_{j-1} \text{ und } a_2^{(j)} = 0 
\end{align*}

TODO Matrix 07 27:32

\begin{remark}
	Für interpolierende Splines kann man Fehlerabschätzungen der Form $||f-s||_{L^\infty(a,b)} \leq C h^{p+1} ||f^{(p+1)}||_{L^\infty(a,b)}$ wobei $h := \max_j (x_j - x_{j-1})$. Aber $C > 0$ hängt an der Glattheit der Splines. Das ist technisch zu beweisen und hängt (natürlich) auch an den Randbdg.
\end{remark}

\subsection{Diskrete und schnelle Fourier-Transformation}

Das Ziel dieses Abschnitts ist die Präsentation und Analyse des FFT-Algorithmus (''fast Fourier transform''), der in der Numerik einer der wichtigsten und vielfaltigsten Algorithmen ist. Unsere ''Motivation'' ist die einfachste Anwendung.

\begin{definition}
	Wir bezeichnen $\mathbb{T}_n := \{ \sum_{j=0}^{n-1} \lambda_j \exp(ijx) | \lambda_0, ..., \lambda_{n-1} \in \mathbb{C}\}$ als Raum der \textbf{trigonometrischen Polynome}.
\end{definition}

\begin{theorem}
	\begin{enumerate}
		\item $\dim \mathbb{T}_{n-1} = n$
		
		\item Zu Stützstellen $0 \leq x_0 < ... < x_{n-1} \lneq 2\pi$ und Funktionswerten $y_j \in \mathbb{C}$ ex. eind. $p \in \mathbb{T}_{n-1}$ mit $p(x_j) = y_j \forall j=0, ..., n-1$
	\end{enumerate}
\end{theorem}

\begin{proof}
	klar: $\dim \mathbb{T}_{n-1} \leq n$
	
	Sei $p(x) = \sum_{j=0}^{n-1} \lambda_j \exp(ijx) \in \mathbb{T}_n$ mit $p(x_j) = 0 \forall j=0, ..., n-1$
	
	zz: $\lambda_j = 0 \forall j=0, ..., n-1$ (dann lin. unabh. und Eind. des Int.pol.)
	
	Def $z_k := \exp(ix_k) \in \mathbb{C}, \exp(ijx_k) = z_k^j$ 
	
	$\implies 0 = p(x_k) = \sum_{j=0}^{n-1} \lambda_j z_k^j =: \tilde{p}(z_n), \tilde{p} \in \mathbb{P}_{n-1} \implies \tilde{p} = 0$, da $\tilde{p} n$ Nullstellen hat $\implies \lambda_j = 0 \forall j=0, ..., n-1$.
\end{proof}

Im restlichen Abschnitt betrachten wir \textbf{äquidistante Stützstellen} (oder \textbf{uniforme Stützstellen}) $x_k = \frac{2\pi k}{n}$ für $k=0, ..., n-1$. Dies führt auf zusätzliche Struktur der Vandermonde-Matrix, die durch FFT genutzt werden kann.

\begin{theorem}
	Seien $x_k = \frac{2\pi k}{n}, y_k \in \mathbb{C}$ gegeben für $k=0, ..., n-1$. Sei $p \in \mathbb{T}_n$ das eind. trig. Int.pol. mit $p(x_j) = y_j \forall k$.
	
	Sei $p(x) = \sum_{j=0}^{n-1} \lambda_j \exp(ijx)$ mit Koeff. $\lambda_j \in \mathbb{C}$.
	
	Def $w_n := \exp(- \frac{2\pi i}{n})$ \textbf{$n$-te Einheitswurzel} und $V_n \in \mathbb{C}^{n\times n}, V_n := (w_n^{jk})_{j,k=0}^{n-1}$ \textbf{Fourier-Matrix} (oder: \textbf{DFT-Matrix})
	
	$\implies$
	\begin{enumerate}
		\item $\frac{1}{n}V_n y = \lambda$ mit $y=(y_0, ..., y_{n-1})$, d.h. $\lambda_j = \frac{1}{n} \sum_{k=0}^{n-1} w_n^{jk} y_k$
		\item $\frac{1}{\sqrt{n}} V_n$ ist symmetrisch und orthogonal, d.h. $\left(\frac{1}{\sqrt{n}}V_n\right)^{-1} = \frac{1}{\sqrt{n}}\overline{V_n}$.
		\item Insb. ist $W=\overline{V_n}$ die Vandermonde-Matrix der trigonometrischen Interpolation
	\end{enumerate}
\end{theorem}

\begin{proof}
	(iii) $\lambda$ ist Lösung des Vandermonde-Systems $W\lambda = y$ mit
	\begin{align*}
		W = \left(\begin{matrix}
			p_0(x_0) & \cdots & p_{n-1}(x_0)\\
			\vdots & & \vdots\\
			p_0(x_{n-1}) & \cdots & p_{n-1}(x_{n-1})
		\end{matrix}\right) \text{ mit } p_j = \exp(ijx)
	\end{align*}
	
	$W_{jk} = p_k(x_j) = \exp(ikx_j) = \exp\left(\frac{2\pi i}{n} jk\right) = w_n^{-jk} = (\overline{w_n})^{jk} \implies W = \overline{V_n} = W$ symmetrisch.
	
	Beweis (i), (ii): Sei $W^{(k)} = (W_{jk})_{j=0}^{n-1}$ $k$-te Spalte von $W$.
	\begin{align*}
		W^{(k)} \cdot W^{(k)} = \sum_{j=0}^{n-1} W_{jk} \overline{W}_{jk} = \sum_{j=0}^{n-1} \underbrace{|W_{jk}|^2}_{=|w_n|^{2jk}=1} = n
	\end{align*}
	Für $k+l$ gilt
	\begin{align*}
		W^{(k)} \cdot W^{(l)} = \sum_{j=0}^{n-1} W_{jk} \overline{W_{jl}} = \sum_{j=0}^{n-1} \underbrace{w_n^{j(l-k)}}_{=(w_n^{(l-k)})^j=0} = \frac{1-(w_n^{(l-k)})^n}{1-w_n^{(l-k)}}\\
		(w_n^{(l-k)})^n = (w_n^n)^{l-k} = 1
	\end{align*}
	$\implies \frac{1}{\sqrt{n}}W$ ist symmetrisch und orthogonal und $W^{-1} = \frac{1}{\sqrt{n}} \left(\frac{1}{\sqrt{n}}W\right)^{-1} = \frac{1}{n} \bar{W} = \frac{1}{n} V_n$.
\end{proof}

\begin{definition}
	Die Abbildung $\mathcal{F}:\mathbb{C}^n \rightarrow \mathbb{C}^n, \mathcal{F}_n(x) = V_n x$ bezeichnet man als \textbf{diskrete Fourier-Transformation (DFT) der Länge $n$}. Die Inverse $\mathcal{F}_n^{-1} \mathbb{C}^n \rightarrow \mathbb{C}^n, \mathcal{F}_n^{-1}(x) = \frac{1}{n}\bar{V}_nx$ heißt \textbf{diskrete Fourier-Rücktransformation}.
\end{definition}

\begin{remark}
	Die Skalierung von $\mathcal{F}_n$ ist in der Literatur uneinheitlich. Manchmal $\frac{1}{n}V_n, \frac{1}{\sqrt{n}}V_n$, \textbf{hier} $\mathcal{F}_n(x) = V_n x$.
\end{remark}

\begin{theorem}
	Für $p \in \mathbb{N}$ und $n=2^p$ und $m = \frac{n}{2}$ sei $w_n := \exp(- \frac{2\pi i}{n})$. Betrachte Permutation $\sigma_n : \mathbb{C}^n \rightarrow \mathbb{C}^n, \sigma_n(x) = (\underbrace{x_1, x_3, ..., x_{n-1}}_{\text{ungerade Indizes}}, \underbrace{x_2, x_4, ..., x_n}_{\text{gerade Indizes}})$.
	
	Für $x \in \mathbb{C}^n$ definiere $a,b \in \mathbb{C}^m$ durch $a_j := x_j + x_{j+m}, b_j = (x_j - x_{j+m})w_n^{j-1}$ für $j=1, ..., m = \frac{n}{2}$.
	
	$\implies \sigma_n(\mathcal{F}_n(x)) = (\mathcal{F}_m(a), \mathcal{F}_m(b))$, d.h. Auswertung von $\mathcal{F}_n(x)$ wird auf Fourier-Trafo halber Länge $\mathcal{F}_m(a), \mathcal{F}_m(b)$ zurückgeführt (+ Vertauschen).
\end{theorem}

\begin{proof}
	\begin{align*}
		\mathcal{F}_n(y_0, ..., y_{n-1}) = \left(\sum_{l=0}^{n-1} w_n^{jl} y_l | j = 0, ..., n-1\right)\\
		\mathcal{F}_n(x_1, ..., x_n) = \left(\sum_{l=0}^{n-1} w_n^{(j-1)l} x_{l+1} | j = 1, ..., n\right)\\
	\end{align*}
	
	(1) zz: $(\mathcal{F}_n(x))_{2j-1} = (\mathcal{F}_m(a))_j$
	
	klar: $w_n^2 = w_m, w_m^m = 1$
	\begin{align*}
		\implies (\mathcal{F}_n(x))_{2j-1} = \sum_{l=0}^{n-1} w_n^{(2j-2)l} x_{l+1} = \sum_{l=0}^{m-1} ( \underbrace{w_n^{2(j-1)l}}_{=w_m^{(j-1)l}} x_{l+1} + \underbrace{w_n^{2(j-1)(l+m)}}_{=w_m^{(j-1)l} \underbrace{ w_m^{2(j-1)m}}_{=1}} x_{l+m+1}) =\\
		\sum_{l=0}^{m-1} w_m^{(j-1)l} \underbrace{(x_{l+1} + x_{l+m+1})}_{=a_{l+1}} = (\mathcal{F}_m(a))_j
	\end{align*}
	
	(2) zz: $(\mathcal{F}_n(x))_{2j} = (\mathcal{F}_m(b))_j$
	
	klar: $w_n^{(2j-1)l} = w_n^{2(j-1)l}w_n^l = w_m^{(j-1)l} w_n^l$, $w_n^{(2j-1)(l+m)} = w_n^{2(j-1)l} w_n^l \underbrace{w_n^{(2j-1)m}}_{=(w_n^{n/2})^(2j+1) = -1} = - w_m^{(j-1)l} w_n^l$
	\begin{align*}
		\implies (\mathcal{F}_n(x))_{2j} = \sum_{l=0}^{n-1} w_n^{(2j-1)}x_{l+1} = \sum_{l=0}^{n-1} \underbrace{(w_n^{(2j-1)l} x_{l+1} + w_n^{(2j-1)(l+m)} x_{l+m+1})}_{= w_m^{(j-1)l}} \underbrace{(x_{l+1} - x_{l+m+1}) w_n^l}_{=b_{l+1}} = (\mathcal{F}_m(b))_j
	\end{align*}
	
\end{proof}

\begin{remark}
	In Matrixform kann man die FFT als Faktorisierung schreiben:
	\begin{align*}
		V_n = P_n \left(\begin{matrix}
			V_{n/2} & 0\\
			0 & V_{n/2}
		\end{matrix}\right) \left(\begin{matrix}
		I_{n/2} & I_{n/2}\\
		D_{n/2} & - D_{n/2}
		\end{matrix}\right)
	\end{align*}
	mit $P_n = (e_1, e_3, ..., e_{n-1}, e_2, e_4, ..., e_n) \in \mathbb{R}^{n\times n}$, $I_{n/2}$ Identität, $D_{n/2} = diag(w_n^0, ..., w_n^{n-1}) \in \mathbb{C}^{n \times n}$
\end{remark}

\begin{corollary}
	Als \textbf{Fast-Fourier-Transformation (FFT)} bezeichnet man die rekursive Berechnung von $\mathcal{F}_n(x)$ mit $n=2^p$ mittels der Rekursion aus dem Satz. Die gesamte Rekursion behötigt weniger als $\frac{3}{2}n \log_2 n$ arithmetische Operationen plus die Berechnung von $w_n^l$ für $l = 0, ...,  n-1$.
\end{corollary}

\begin{proof}
	(1) Wegen $w_{n/2}^l = w_n^{2l}$ reicht es, alle $w_n^l$ für $l=0, ..., n-1$ und das maximale $n$ zu berechnen.
	
	(2) Beweis des arithm. Aufwands durch Induktion nach $p$.
	\begin{itemize}
		\item $A(p) = $ Anzahl Additionen/Subtraktionen
		\item $M(p) = $ Anzahl Multiplikationen
	\end{itemize}
	
	Beh: $A(p) = p^{2^p}, M(p) \leq \frac{1}{2} p^{2^p}$ (dann fertig, da Aufwand$(\mathcal{F}_n(x)) = A(p) + M(p) \leq \frac{2}{3} p^{2^p} = \frac{3}{2} n \log_2n$)
	
	Ind.anf. $p=1, n=2, ..., m-1 \implies A(p) = 2, M(p) = 0$ \checkmark
	
	Ind.schritt: Die Aussage gelte für $p$, dann $A(p+1) = 2 \underbrace{A(p)}_{=p^{2^p}} + 2 \cdot 2^p = (p+1) 2^{p+1}$ \checkmark. $M(p+1) = 2 \underbrace{M(p)}_{\leq \frac{1}{2} p^{2^p}} + 2^p \leq p^{2^p} + 2^p = \frac{1}{2} (p+1)2^{p+1}$ \checkmark
\end{proof}

\begin{remark}
	Die FFT ist eine schnelle Matrix-Vektor-Multiplikation für vollbesetzte Matrizen, die eine gewisse Struktur haben, ohne die Matrix jeweils voll aufzubauen, d.h. Speicherbedarf $\mathcal{O}(n)$.
\end{remark}

UE: $A \in \mathbb{K}^{n \times n}$ \textbf{zirkulant}, d.h.
\begin{align*}
	A = \left(\begin{matrix}
		a_0 & a_{n-1} & \cdot & a_1\\
		a_1 & a_0 & \ddots & \vdots\\
		\vdots & \ddots & \vdots\\
		a_{n-1} & a_{n-2} & \cdots & a_0
	\end{matrix}\right)
\end{align*}

$\implies V_n A V_n^{-1} = D = diag(p(1), p(w_n), ..., p(w_n^{n-1}))$ mit $p(x) = \sum_{j=0}^{n} a_j x^j$

$\implies$ klar: $A$ ist regulär, gdw. $p(w_n^j) \neq 0 \forall j=0, ..., n-1$

Sei $A$ zusätzlich regulär, $b \in \mathbb{K}^k$, Ziel: Löse $Ax=b$

$\implies A = V_n^{-1}DV_n, A^{-1} = V_n^{-1}D^{-1}V_n \implies x = V_n^{-1}D^{-1}V_nb = \frac{1}{n} \overline{V_n}D^{-1}V_nb$, $\overline{V_n\bar{y}}$

Frage: Wie berechnet man die Diagonale von $D$ effizient?

naiv: $\mathcal{O}(n^2)$, clever: $V_nA=DV_n$, insb. $V_na = D(1, ..., 1)^T = diag(D) \implies$ FFT liefert $diag(D)$.

\begin{remark}
	\begin{itemize}
		\item Die FFT verdankt ihren Namen dem Zusammenhang mit der Fourier-Transformation: Für $f \in L^2[0, 2\pi]$ definiere $\hat{f}(k) := \frac{1}{2\pi} \int_{0}^{2\pi}f(t)e^{-ikt}dt$ und $S_n(t) := \sum_{k=-n}^{n} \hat{f}(k) e^{ikt}$
		
		$\implies \lim_n ||f-S_n||_{L^2(0, 2\pi)} = 0$ und $\frac{1}{2\pi} ||f||_{L^2(0,2\pi)}^2 = \sum_{k=-\infty}^{\infty}|\hat{f}(k)|^2$
		
		\item Die trig. Interpolation ist auch für $p(x) := \sum_{k=-n}^{n} c_k \exp(ikt)$ mit $p(x_j) = y_j \forall j=-n, ..., n$ eind. lösbar, denn $p(x) = \sum_{l=0}^{2n} c_{l-n} \underbrace{\exp(i(l-n)x)}_{=\exp(ilk)\exp(-inx) \neq 0}$
	\end{itemize}
\end{remark}






\end{document}
