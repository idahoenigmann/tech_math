\chapter{Data and Algorithm}
\label{chapter:data_and_algorithm}

This chapter describes the CAP Sleep Database used and the algorithm implemented. The algorithm uses FFT, PCA and k-nearest neighbor to estimate the sleep stage from a recorded EEG signal.

\section{Data}
\label{sec:data}

We use the CAP Sleep Database\cite{Terzano2001}\cite{Goldberger2000} which provides 16 recordings of patients without pathology. Two of these recordings have corrupted files and one did not have the data points used for this analysis and thus could not be analyzed. The table~\ref{tab:meta_data_of_recordings} shows some meta data of the recordings.

\begin{table}
	\centering
	\begin{tabular}{c|c|c|c|c|c}
		File Name & Age & Gender & Length of Recording & Channel Name & Sampling Frequency \\
		\hline
		n1  & 37 & F & 577 minutes & C4-A1  & 512 Hz \\
		n2  & 34 & M & 735 minutes & C4-A1  & 512 Hz \\
		n3  & 35 & F & 551 minutes & C4-A1  & 512 Hz \\
		n4  & 25 & F & 596 minutes & C4-A1  & 200 Hz \\
		n5  & 35 & F & 524 minutes & C4-A1  & 512 Hz \\
		n6  & 31 & M & 527 minutes & C3-A2  & 128 Hz \\
		n7  & 31 & M & 492 minutes & C3-A2  & 128 Hz \\
		n8  & 42 & F & 501 minutes & C3-A2  & 200 Hz \\
		n9  & 31 & M & 532 minutes & C3-A2  & 128 Hz \\
		n10 & 23 & M & 490 minutes & C4-A1  & 512 Hz \\
		n11 & 28 & F & 527 minutes & C4-A1  & 512 Hz \\
		n12 & 29 & M & 495 minutes & C3-A2  & 100 Hz \\
		n16 & 41 & F & 513 minutes & C4-A1  & 100 Hz \\
	\end{tabular}
	\caption{Meta data of the 13 recordings from the CAP Sleep Database.}
	\label{tab:meta_data_of_recordings}
\end{table}

The given dataset describes multiple channels of EEG data recorded during sleep. We are interested in the channels recording activity in the central area, as these are the two channels most of the recordings shared. The figure~\ref{fig:head_placement} shows where the sensors were placed.

\begin{figure}
	\centering
	\includegraphics[width=0.7\linewidth]{figs/head_placement}
	\caption{Scheme of sensor placement in recordings of EEG data.}
	\label{fig:head_placement}
\end{figure}

The data was recorded in different frequencies. To reduce data size and standardize all channels we resample to 200 Hz.

Very low and very high frequencies often stem from other sources, such as the measurement equipment, breathing of the patients or measuring inaccuracies. To get rid of these noises we apply a low pass filter of 30 Hz and a high pass filter of 0.5 Hz.

\section{Algorithm}
\label{sec:algorithm}

We now describe the general structure of the algorithm from reading the data files to the clustering.

First we split the recorded data into 30 second sections. For each of these sections we get the data as a function of amplitude over time. An example segment is shown in the upper figure of~\ref{fig:example_30s_segment}. We apply a Fourier transformation as described in section~\ref{sec:fourier_transformation} and get a function of amplitude over frequency. This gives a frequency based representation of the data. As the sleep stages are characterized by different frequencies and amplitudes, as described in section~\ref{sec:eeg_data_and_sleep_stages}, this will help us distinguishing them. The output of such an FFT can be seen in the lower figure of~\ref{fig:example_30s_segment}. The pseudo code for this processing is described in algorithm~\ref{alg:process_eeg_data}.

\begin{algorithm}
	\caption{EEG data processing}\label{alg:process_eeg_data}
	\begin{algorithmic}
		\For{each patient}
		\State Read EEG data from file
		\State Apply a 30 Hz low pass and a 0.5 Hz high pass filter
		\State Resample with frequency 200 Hz
		\State Get one of the channels C4-A1 or C3-A2 (in this order)
		\State Section out the part of data where the patient is asleep
		\For{each 30 second segment of the data}
		\State Do a Fourier transformation (FFT)
		\EndFor
		\State Save all the outputs from the FFT to the patients file
		\EndFor
	\end{algorithmic}
\end{algorithm}


\begin{figure}
	\centering	
	\begin{subfigure}[b]{\textwidth}
		\begin{tikzpicture}
			\begin{axis}[xlabel=Time (seconds), ylabel=Amplitude, width=\textwidth, height=0.3\textwidth]
				\addplot+ [no marks, color=black] table[col sep=comma] {figs/example_30s_segment1.csv};
			\end{axis}
		\end{tikzpicture}
	\end{subfigure}
	\vskip\baselineskip
	\begin{subfigure}[b]{\textwidth}
		\begin{tikzpicture}
			\begin{axis}[xlabel=Frequency (Hz), ylabel=Amplitude, width=\textwidth, height=0.3\textwidth]
				\addplot+ [no marks, color=black] table[col sep=comma] {figs/example_30s_segment2.csv};
			\end{axis}
		\end{tikzpicture}
	\end{subfigure}
	
	
	\caption{A 30 second segment of the data. The top shows the recorded signal of the EEG. The bottom shows the output of the Fourier transformation.}
	\label{fig:example_30s_segment}
\end{figure}

We now have high dimensional data points, as each 30 second segment is represented by the amplitudes for each of the distinct frequencies output by the FFT. Before we can start searching for cluster in the data we want to find a lower dimensional representation. This is were PCA can be used. From the output of algorithm~\ref{alg:pca_of_eeg_data} we get principal components, which reveal the frequencies were the most variance is shown. Under our assumptions this variation is caused by the different sleep stages each recording goes through. We show a visual representation of the data in the first three principal components in figure~\ref{fig:pca_output_3d}.

\begin{algorithm}
	\caption{Apply PCA to the EEG data}\label{alg:pca_of_eeg_data}
	\begin{algorithmic}
		\For{each patient}
			\State Read FFT output from patients file
			\State Read the sleep stages from another file
		\EndFor
		\State Concatenate FFT data of all patients
		\State Concatenate sleep stages of all patients
		\State Do PCA on the combined FFT data
		\State Transform data according to the principal components from the PCA
		\State Visualize the data in the first three axis, with the color corresponding to the sleep stage
	\end{algorithmic}
\end{algorithm}

\begin{figure}
	\centering
	\includegraphics[width=\linewidth]{figs/pca_output_3d}
	\caption{Three dimensional representation of the data. Each 30 second segment is represented by a dot in the color corresponding to the sleep stage this segment is assigned to. Awake is represented by red, orange is REM, yellow is S1, green S2 and blue S3.}
	\label{fig:pca_output_3d}
\end{figure}

If we have a new recording and want to know the sleep stages we have to follow similar steps. First the recording has to be split into 30 second segments. These have to be transformed by a FFT. The output can be converted to the PCA basis by multiplying with the matrix of principal components. Lastly we can get a estimate of the sleep stage by using the k nearest neighbor algorithm. The pseudo code is shown in algorithm~\ref{alg:gues_sleep_stage}.

\begin{algorithm}
	\caption{Get estimate for sleep stage}\label{alg:gues_sleep_stage}
	\begin{algorithmic}
		\Require{EEG recording of sleep}
		\For{each 30s segment}
			\State Do FFT on the segment
			\State Multiply FFT output by principal component matrix
			\State Reduce dimensions
			\State Do k nearest neighbor
			\State Save the result of k nearest neighbor
		\EndFor
		\State \Return results of k nearest neighbor
	\end{algorithmic}
\end{algorithm}
