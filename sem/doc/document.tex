\documentclass[a4paper]{IEEEtran}
\usepackage{amsmath}
\usepackage{amssymb}
\usepackage{amsthm}
\usepackage{bbm}
\usepackage{enumerate}
\usepackage[shortlabels]{enumitem}
\usepackage{algorithm}
\usepackage{algpseudocode}
\usepackage{graphicx}
\usepackage{subcaption}

\newtheorem{definition}{Definition}
\newtheorem{lemma}{Lemma}
\newtheorem{theorem}{Theorem}

%opening
\title{Using PCA on EEG Data to Distinguish Sleep~Stages}
\author{\IEEEauthorblockN{Ida Hönigmann}
	\IEEEauthorblockA{\\Technische Universität Wien, Austria\\
		Email: e12002348@student.tuwien.ac.at}}

\begin{document}

\maketitle

\begin{abstract}
[TODO]
\end{abstract}

\section{Introduction}
\label{sec:introduction}



\section{Study of Literature}
\label{sec:study_of_literature}

\section{Mathematical Basics}
\label{sec:mathematical_basics}


\section{Principal Component Analysis}
\label{sec:principal_component_analysis}



\section{Methodology}
\label{sec:methodology}



\section{Sleep Stages and EEG Data}
\label{sec:sleep_stages_and_eeg_data}

\section{Data and Algorithm}
\label{sec:data_and_algorithm}

\begin{enumerate}
	\item subdivide eeg signals in the temporal domain
	\item apply fft transforming into frequency domain
	\item pca
	\item achive dimensinality reduction
	\item classification of sleep stages
	\item visulisation
\end{enumerate}

\section{Results}
\label{sec:results}

\section{Conclusion}
\label{sec:conclusion}

\bibliographystyle{plain}
\bibliography{document}

\end{document}
