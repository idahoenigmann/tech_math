\chapter{Introduction}
\label{chapter:introduction}

[TODO general introduction]


\section{EEG Data and Sleep Stages}
\label{sec:eeg_data_and_sleep_stages}

An electroencephalogram (EEG) records the variations in potential in the brain. A bipolar EEG measures the potential between two electrodes placed on the head. In comparison a unipolar measures the potential between a electrode on the head and one on some other part of the body. In this work we are only concerned with bipolar EEG measurements.

Ganong~\cite[chapter~11]{Ganong1997} describes typical patterns observed in EEG data of a sleeping person. He describes the EEG patterns associated with rapid eye movement~(REM) sleep and non-REM~(NREM) sleep. NREM sleep is further partitioned into three stages, termed Stage~1 (S1) to Stage~3 (S3)\footnote{some authors use four stages}. The EEG data of these stages is characterized by

\begin{enumerate}[label={S\arabic*:}]
	\item low amplitude, high frequency
	\item appearance of sleep spindles (bursts of higher amplitude, lower frequency waves)
	\item high amplitude, low frequency
\end{enumerate}

\noindent
In REM sleep the EEG data is that of high frequency and low amplitude patterns, resembling the data observed in alert humans.

Example EEG data of these different sleep stages can be seen in Figure~\ref{fig:different_sleep_stages}.

\begin{figure}
	\centering

	\begin{subfigure}[b]{\textwidth}
		\begin{tikzpicture}
			\begin{axis}[width=\textwidth, height=0.8\textwidth, ytick=\empty, xtick=\empty, clip=false]
				\addplot+ [no marks, color=black] table[col sep=comma] {figs/example_stage_awake.csv};
				\addplot+ [no marks, color=black] table[col sep=comma] {figs/example_stage_S1.csv};
				\addplot+ [no marks, color=black] table[col sep=comma] {figs/example_stage_S2.csv};
				\addplot+ [no marks, color=black] table[col sep=comma] {figs/example_stage_S3.csv};
				\addplot+ [no marks, color=black] table[col sep=comma] {figs/example_stage_REM.csv};
				
				\node[above,black] at (-15.0,420.0) {awake};
				\node[above,black] at (-15.0,315.0) {REM};
				\node[above,black] at (-15.0,220.0) {S1};
				\node[above,black] at (-15.0,125.0) {S2};
				\node[above,black] at (-15.0,30.0) {S3};
			\end{axis}
		\end{tikzpicture}
	\end{subfigure}
	
	\caption{Short data segments of the the different sleep stages.}
	\label{fig:different_sleep_stages}
\end{figure}