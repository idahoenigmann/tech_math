\documentclass[a4paper]{IEEEtran}
\usepackage{amsmath}
\usepackage{amssymb}
\usepackage{amsthm}
\usepackage{enumerate}
\usepackage[shortlabels]{enumitem}

\newtheorem{definition}{Definition}
\newtheorem{lemma}{Lemma}
\newtheorem{theorem}{Theorem}

%opening
\title{Using PCA on EEG Data to Distinguish Sleep~Stages}
\author{\IEEEauthorblockN{Ida Hönigmann}
	\IEEEauthorblockA{\\Technical University Vienna, Austria\\
		Email: e12002348@student.tuwien.ac.at}}

\begin{document}

\maketitle

\begin{abstract}
[TODO]
\end{abstract}

\section{Introduction}
\label{sec:introduction}

[TODO general introduction]

\subsection{EEG Data and Sleep Stages}

Ganong~\cite{Ganong1997} describes typical patterns observed in electroencephalogram (EEG) data of a sleeping person. He describes the EEG patterns associated with rapid eye movement~(REM) sleep and non-REM~(NREM) sleep.

NREM sleep is further partitioned into four (although some only use three) stages, termed Stage~1 (S1) to Stage~4 (S4). Example EEG data of these different sleep stages can be seen in Figure~\ref{pic:TODO} [TODO image]. The EEG data of these stages is characterized as follows:

\begin{enumerate}[label={S\arabic*:}]
\item low-amplitude, high-frequency
\item appearance of sleep spindles (bursts of higher amplitude, lower frequency waves)
\item increased amplitude, lower frequency
\item maximal amplitude, minimal frequency
\end{enumerate}

In REM sleep the EEG data is that of high frequency and low amplitude patterns, resembling the data observed in alert humans.

\section{Study of Literature}
\label{sec:study_of_literature}

%[Note:
%
%first work on pca \cite{Pearson1901} and \cite{Hotelling1933}
%
%given paper \cite{Jolliffe2016}
%
%when does pca fail? \cite{Shlens2014} and \cite{Tenenbaum2000} (non-linear method)
%
%book containing sleep phases eeg \cite{Ganong1997}
%
%Review Paper on Sleep Stage Classification Methods \cite{Boostani2017}
%
%papers trying to solve similar problem \cite{Tautan2021} and \cite{Putilov2015} and \cite{Metzner2023}
%
%competition using similar data set \cite{Ghassemi2018}
%
%winner of competition \cite{Howe2018}]
%\\
%\\
%\\
A substantial body of scientific research has been devoted to exploring Principal Component Analysis (PCA).
The foundation of this method was laid by Pearson~\cite{Pearson1901} and Hotelling~\cite{Hotelling1933}.

An introduction to PCA, as well as a good overview on how to derive the formula used to compute the Principal Components (PC) is given by Shlens~\cite{Shlens2014}.
Recent applications and variants of PCA are explored by Jolliffe et. al.~\cite{Jolliffe2016}.

Shlens discusses the limitations of PCA, as well as examples in which PCA fails~\cite{Shlens2014}, such as the requirement of linearly dependent data.
Tenenbaum proposes a non-linear method to combat this problem\cite{Tenenbaum2000}.

Generally speaking the variables must not have third or higher order dependencies\footnote{e.g. $\mathbb{E}[x_ix_jx_k] \neq 0$ for some $i, j, k$ assuming mean-free variables} between them. In some cases it is possible to reduce a problem with higher order dependencies to a second order one by applying a non-linear transformation beforehand. This method is called kernel PCA\cite{Shlens2014}.

Another method for combating this problem is Independent Component Analysis (ICA) which is discussed by Naik~et.~al.\cite{Naik2011}.
\\
\\
The given problem of distinguishing sleep stages given some EEG data has been investigated by use of PCA, as well as neural networks. Some of these works are summarized below.

A review of different methods in the preprocessing, feature extraction and classification is given by Boostani et. al.\cite{Boostani2017}. They find that using a random forest classifier and entropy of wavelet coefficients as feature gives the best results.

Tăuţan et. al.\cite{Tautan2021} compare different methods of dimensionality reduction on EEG data, such as PCA, factor analysis and autoencoders. They conclude that PCA and factor analysis improves the accuracy of the model.

Putilov\cite{Putilov2015} used PCA to find boundaries between Stage~1, Stage~2 and Stage~3. Changes in the first two PC were related to changes between the Stage~1 and Stage~2, while changes in the fourth PC exhibited a change in sign at the boundary of Stage~2 and Stage~3. This suggests that changes between Stage~1 and Stage~2 are easier to detect that ones between Stage~2 and Stage~3.

Metzner et. al.\cite{Metzner2023} try to rediscover the different human-defined sleep stages. They find that using PCA on the results makes clusters apparent. These clusters could then be used as a basis for a redefinition of sleep stages.

The PhysioNet/Computing in Cardiology Challange 2018 was a competition using a similar data\cite{Ghassemi2018}. The goal was to identify arousal during sleep from EEG, EOG, EMG, ECG and SaO2 data given. The winning paper of this competition describes the use of a dense recurrent convolutional neural network (DRCNN) consisting of multiple dense convolutional layers, a bidirectional long-short term memory layer and a softmax output layer\cite{Howe2018}.
\\
\\
As shown in this section, the utilization of PCA to analyze EEG data has been used with success.

\section{Mathematical Basics}
\label{sec:mathematical_basics}

We define mathematical notation, which will be used in Section~\ref{sec:principal_component_analysis} to define the PCA.

\subsection{Covariance}
Assume we have two sets of $n$ observations of variables with mean 0. Let us call the first list of observations $\mathbf{a} = (a_1, ..., a_n)$ and the second $\mathbf{b} = (b_1, ..., b_n)$.

\begin{definition}[covariance]
Let us define the \textit{covariance} of $\mathbf{a} \in \mathbb{R}^n$ and $\mathbf{b} \in \mathbb{R}^n$ as
\begin{align*}
	\sigma_{\mathbf{ab}} := \frac{1}{n} \sum_{i=1}^{n}a_ib_i = \frac{1}{n}\mathbf{a}\cdot\mathbf{b}^T.
\end{align*}
\end{definition}

From the definition it is obvious that the covariance is symmetric, $\sigma_{\mathbf{ab}} = \sigma_{\mathbf{ba}}$. In the special case $\mathbf{a} = \mathbf{b}$ the covariance $\sigma_{\mathbf{aa}}$ is called \textit{variance} $\sigma_{\mathbf{a}}^2$.

\begin{definition}[covariance matrix]
Generalizing to $m$ variables $\mathbf{X} = [\mathbf{x_1}, ..., \mathbf{x_m}]$, each having been observed $n$ times, gives us the \textit{covariance matrix}.

\begin{align*}
	\mathbf{C_X} := \left(\begin{matrix}
		\sigma_{\mathbf{x_1x_1}}	& \cdots & \sigma_{\mathbf{x_1x_m}}	\\
		\vdots						& \ddots & \vdots					\\
		\sigma_{\mathbf{x_mx_1}}	& \cdots & \sigma_{\mathbf{x_mx_m}}	\\
	\end{matrix}\right) = \frac{1}{n} \mathbf{X}\mathbf{X}^T
\end{align*}
\end{definition}

The covariance matrix is a symmetric $m\times m$ matrix.

\subsection{Diagonalizable Matrix}

\begin{definition}[Diagonalizable Matrix]
	A square matrix $\mathbf{A}$ is called \textit{diagonalizable}, if there exists a invertable matrix $\mathbf{P}$ and a diagonal matrix $\mathbf{D}$ such that $\mathbf{A} = \mathbf{P}\mathbf{D}\mathbf{P}^{-1}$.
\end{definition}

\begin{definition}[Symmetric matrix]
	A square matrix $\mathbf{A}$ is called \textit{symmetric}, if $\mathbf{A}^T = \mathbf{A}$.
\end{definition}

\begin{theorem}
	\label{th:symmetric_matrix_diagonalizable}
	Every symmetric matrix is diagonalizable.
\end{theorem}

The proof of this theorem requires some preparation, which we will do now.

\begin{definition}[Eigenvalues and Eigenvectors]
	Let $\mathbf{A}$ be a real $m\times m$ matrix. $\lambda \in \mathbb{R}$ is called a \textit{eigenvalue} with \textit{eigenvector} $\mathbf{v} \in \mathbb{R}^m\setminus\{\mathbf{0}\}$ if
	\begin{align}
		\label{eq:def_eigenvalue}
		\mathbf{Av} = \lambda \mathbf{v}.
	\end{align}
\end{definition}

\begin{lemma}
	\label{lem:existence_eigenvalues}
	Every square $m\times m$ matrix has $m$ (not necessarily unique) eigenvalues.
\end{lemma}

\begin{proof}
	We can rewrite equation~\ref{eq:def_eigenvalue} as
	\begin{align*}
		(\mathbf{A} - \lambda \mathbf{I})\mathbf{v} = \mathbf{0}
	\end{align*}
	
	This allows us to interpret $(\mathbf{A}-\lambda \mathbf{I})$ as a function, which takes vectors $\mathbf{v} \in \mathbf{R}^m$. For $\lambda$ to be a eigenvalue of $\mathbf{A}$ with eigenvector $\mathbf{v}$ it has to satisfy $\mathbf{v} \in \ker(\mathbf{A} - \lambda \mathbf{I})$ and $\mathbf{v} \neq 0$. From this we gather that all $\lambda$ with $\ker(\mathbf{A} - \lambda \mathbf{I}) \neq \{\mathbf{0}\}$ are eigenvalues. We know that this holds if and only if $\det(\mathbf{A} - \lambda \mathbf{I}) = 0$. The determinant is a polynomial of degree $m$ which can be expressed in the form $(\lambda - \lambda_1)...(\lambda - \lambda_m)$ with $\lambda_1, ..., \lambda_m \in \mathbb{C}$. These $\lambda_1, ..., \lambda_m$ are the $m$ eigenvalues we wanted to find.
\end{proof}

\begin{lemma}
	A symmetric matrix has real eigenvalues.
\end{lemma}

\begin{proof}
	Let $\bar{.}$ denote the complex conjugate. Define a complex dot product
	\begin{align*}
		(\mathbf{u}, \mathbf{v}) := \sum_{i=1}^{m} u_i \bar{v_i}
	\end{align*}
	This dot product has the following properties for all $\mathbf{A} \in \mathbb{C}^{m\times m}, \mathbf{u}, \mathbf{v} \in \mathbb{C}^m, \lambda \in \mathbb{C}$
	\begin{itemize}
		\item $(\mathbf{Au}, \mathbf{v}) = (\mathbf{u}, \mathbf{A}^T\mathbf{v})$,
		\item $(\lambda \mathbf{u}, \mathbf{v}) = \lambda(\mathbf{u}, \mathbf{v})$,
		\item $(\mathbf{u}, \lambda \mathbf{v}) = \bar{\lambda} (\mathbf{u}, \mathbf{v})$
		\item $(\mathbf{u}, \mathbf{u}) = 0 \iff \mathbf{u} = 0$
	\end{itemize}
	
	Let $\mathbf{A}$ be a symmetric matrix with eigenvalue $\lambda \in \mathbb{C}$.
	
	From this it follows that for all $\mathbf{u} \in \mathbb{C}^m$
	\begin{align*}
		\lambda (\mathbf{u}, \mathbf{u}) = (\lambda \mathbf{u}, \mathbf{u}) = (\mathbf{Au}, \mathbf{u}) = (\mathbf{u}, \mathbf{A}^T\mathbf{u}) =\\
		(\mathbf{u}, \mathbf{Au}) =	(\mathbf{u}, \lambda\mathbf{u}) = \bar{\lambda} (\mathbf{u}, \mathbf{u}).
	\end{align*}
	
	We derive that $\lambda = \bar{\lambda}$ and thus $\lambda \in \mathbb{R}$.
\end{proof}

Are the corresponding eigenvectors real? From the proof of lemma~\ref{lem:existence_eigenvalues} we know that the eigenvector $\mathbf{v}$ of eigenvalue $\lambda$ is from the $\ker(\mathbf{A} - \lambda\mathbf{I})$. Both the matrix $\mathbf{A}$ and $\lambda$ are real, so $\mathbf{v}$ must be as well.

\begin{lemma}
	\label{lem:symmetric_matrix_eigenvector_orthogonal}
	The eigenvectors of a symmetric matrix with distinct eigenvalues are orthogonal.
\end{lemma}

\begin{proof}
	Let $\lambda_1, \lambda_2$ be two distinct eigenvalues with eigenvectors $\mathbf{v}_1, \mathbf{v}_2$ of the matrix $\mathbf{A}$.
	
	\begin{align*}
		\lambda_1 \mathbf{v}_1 \cdot \mathbf{v}_2 = (\lambda_1\mathbf{v}_1)^T \mathbf{v}_2 = (\mathbf{Av}_1)^T \mathbf{v}_2 = \mathbf{v}_1^T \mathbf{A}^T \mathbf{v}_2 =\\
		\mathbf{v}_1^T \mathbf{A} \mathbf{v}_2 = \mathbf{v}_1^T (\lambda_2 \mathbf{v}_2) = \lambda_2 \mathbf{v}_1 \cdot \mathbf{v}_2
	\end{align*}
	
	This shows that $(\lambda_1 - \lambda_2) \mathbf{v}_1 \cdot \mathbf{v}_2 = 0$ and as $\lambda_1$ and $\lambda_2$ are distinct, $\mathbf{v}_1$ and $\mathbf{v}_2$ must be orthogonal.
\end{proof}

[TODO what if the eigenvalues are not distinct?]

Now we have everything we need to prove theorem~\ref{th:symmetric_matrix_diagonalizable}. 

\begin{proof}[Proof of Theorem~\ref{th:symmetric_matrix_diagonalizable}]
	Let $\mathbf{A} \in \mathbb{R}^{m\times m}$ be a symmetric matrix. From lemma~\ref{lem:existence_eigenvalues} we know that eigenvalues $\lambda_1, ..., \lambda_m$ with corresponding eigenvectors $\mathbf{v}_1, ..., \mathbf{v}_m$ exist.
	
	Define the following matrices
	\begin{align*}
		\mathbf{D} := \left(\begin{matrix}
			\lambda_1 & 0 & \cdots & 0\\
			0 & \lambda_2 & \cdots & 0\\
			\vdots & \vdots & \ddots & \vdots\\
			0 & 0 & \cdots & \lambda_m
		\end{matrix}\right) &&
		\mathbf{E} := \left(\begin{matrix}
			 & & &\\
			\mathbf{v}_1 & \mathbf{v}_2 & \cdots & \mathbf{v}_m\\
			 & & &\\
		\end{matrix}\right)
	\end{align*}
	
	In order to show that $\mathbf{A} = \mathbf{EDE}^{-1}$ we calculate
	\begin{align*}
		\mathbf{AE} = \left(\begin{matrix}
			\mathbf{Av}_1 & \cdots & \mathbf{Av}_m
		\end{matrix}\right) = \left(\begin{matrix}
			\lambda_1\mathbf{v}_1 & \cdots & \lambda_m\mathbf{v}_m
		\end{matrix}\right) = \mathbf{ED}.
	\end{align*}
	
	From lemma~\ref{lem:symmetric_matrix_eigenvector_orthogonal} we know that the eigenvectors, and therefore the columns of $\mathbf{E}$, are orthogonal. From this it follows that $rank(\mathbf{E}) = m$ which gives us the existence of $\mathbf{E}^{-1}$.
	
	This shows that $\mathbf{A}$ is diagonalizable.
\end{proof}

\begin{lemma}
	If the columns of a matrix $\mathbf{A}$ are orthogonal, then $\mathbf{A}^{-1} = \mathbf{A}^T$.
\end{lemma}

\begin{proof}
	Let $(\mathbf{a}_i)_{i=1,...,m}$ be the columns of the matrix.
	
	\begin{align*}
		\forall i,j: \mathbf{a}_i^T\mathbf{a}_j = \begin{cases}
			1 & \text{if } i=j\\
			0 & \text{otherwise}
		\end{cases} \implies \mathbf{A}^T\mathbf{A} = \mathbf{I}
	\end{align*}
	
	This shows $\mathbf{A}^{-1} = \mathbf{A}^T$.
\end{proof}

\section{Principal Component Analysis}
\label{sec:principal_component_analysis}

Combining the concepts in section~\ref{sec:mathematical_basics} we derive the ideas and implementation of PCA.

Assume we have gathered observations of different variables as part of an experiment. If we have $n$ variables, each having been observed $m$ times, we can create a $m \times n$ matrix of this data. What can we do to get more insight and find underlying patterns? For $n = 2$ we could try to plot the data, with the first variable as the $x$-axis and the second as the $y$ axis. [TODO plot as img?]

For larger values of $n$ this gets increasingly difficult\footnote{For $n = 3$ we can construct a 3 dimensional visualization. For higher dimensionality we have to use projection. Depending on the chosen projection the interpretation changes, which makes it difficult to interpret.}. PCA tries to solve this problem by transforming the data in such a way that the most interesting structure is in the first few axis of the transformed $n$ dimensional space. This makes it easy to look at a low dimension representation of the data, without loosing much information.



What happens if we skip the step in which we transform the mean to zero? [TODO]

\section{Sleep Stages and EEG Data}
\label{sec:sleep_stages_and_eeg_data}

\section{Data and Algorithm}
\label{sec:data_and_algorithm}

\begin{enumerate}
	\item subdivide eeg signals in the temporal domain
	\item apply fft transforming into frequency domain
	\item pca
	\item achive dimensinality reduction
	\item classification of sleep stages
	\item visulisation
\end{enumerate}

\section{Results}
\label{sec:results}

\section{Conclusion}
\label{sec:conclusion}

\bibliographystyle{plain}
\bibliography{document}

\end{document}
