\documentclass[a4paper]{IEEEtran}

%opening
\title{Using PCA on EEG Data to Differentiate Sleep~Stages}
\author{\IEEEauthorblockN{Ida Hönigmann}
	\IEEEauthorblockA{\\Technical University Vienna, Austria\\
		Email: e12002348@student.tuwien.ac.at}}

\begin{document}

\maketitle

\begin{abstract}

\end{abstract}

\section{Introduction}

\section{Study of Literature}
%The foundation of Pricipal Component Analysis (PCA) was layed by Pearson\cite{Pearson1901} and expanded on by Hotelling\cite{Hotelling1933}. Recent applications and variants of PCA are explored by Jolliffe et. al.\cite{Jolliffe2016}.

%Limitations of PCA are given by data that is non-linearly dependant. One such example is mentioned by Shlens\cite{Shlens2014}. Tenenbaum proposes a non-linear method to combat this problem\cite{Tenenbaum2000}.


first work on pca \cite{Pearson1901} and \cite{Hotelling1933}

given paper \cite{Jolliffe2016}

when does pca fail? \cite{Shlens2014} and \cite{Tenenbaum2000} (non-linear method)

book containing sleep phases eeg \cite{Ganong1997}

Review Paper on Sleep Stage Classification Methods \cite{Boostani2017}

papers trying to solve similar problem \cite{Tautan2021} and \cite{Putilov2015} and \cite{Metzner2023}

competition using similar data set \cite{Ghassemi2018}

winner of competition \cite{Howe2018}

\section{Mathematical Basics}

\section{Principal Component Analysis}

\section{Sleep Stages and EEG Data}

\section{Data and Algorithm}
\begin{enumerate}
	\item subdivide eeg signals in the temporal domain
	\item apply fft transforming into frequency domain
	\item pca
	\item achive dimensinality reduction
	\item classification of sleep stages
	\item visulisation
\end{enumerate}



\section{Results}

\section{Conclusion}


\bibliographystyle{plain}
\bibliography{document}

\end{document}
